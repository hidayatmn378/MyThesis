%\vspace{1.5pc}
\vspace{1.5pc}
\section[Latar Belakang]{Latar Belakang}
\begin{spacing}{1.5}
	Samudra Hindia adalah samudra terbesar ketiga di dunia, meliputi sekitar 19.8\% dari total volume lautan \shortcite{Eakins2010} dan merupakan lautan yang sangat berpengaruh bagi ekosistem di Bumi. Wilayah perairan Samudra Hindia mencakup Teluk Benggala (\textit{Bay of Bengal} (BoB)), Laut Andaman, Selat Malaka, dan perairan Aceh. Dengan cakupan wilayah yang begitu luas, Samudra Hindia merupakan penyumbang besar bagi sistem iklim dunia dan oleh karena itu sangat penting untuk dapat diprediksi. Pengembangan model kelautan bertujuan untuk menggambarkan iklim global secara akurat dan sesuai dengan hasil observasi atau pengamatan. Namun, variabilitas spasial dan temporal perlu dipahami untuk memperoleh hasil prediksi yang lebih baik. Pemanasan matahari dan kekuatan angin bervariasi dalam ruang dan waktu yang tercermin dalam variabilitas lapisan campuran laut dan suhu permukaan laut. Oleh karena itu, fokus utama dari tesis ini adalah mengkaji peran gaya atmosfer lokal pada variabilitas lapisan campuran dan akibatnya pada suhu permukaan laut.
	
	Beberapa studi observasional dan pemodelan laut telah dilakukan untuk mempelajari pengaruh interaksi atmosfer-laut terhadap variabilitas suhu permukaan laut (SST), salinitas permukaan laut (SSS), klorofil-a (Chl-a), kedalaman lapisan campuran (MLD) dan sirkulasi arus pada wilayah perairan Samudra Hindia, diantaranya adalah, \shortciteNP{Kantha2019} yang meneliti tentang pencampuran turbulen di lapisan atas laut untuk domain BoB utara. Pencampuran ini dipengaruhi oleh lapisan dangkal yang menutupi perairan asin teluk, yang dihasilkan dari arus besar air tawar yang bersumber dari sungai-sungai besar yang mengalir dari anak benua Asia. Faktor lain yang berpengaruh adalah curah hujan di lapisan atas laut selama musim panas. Lokasi BoB berbatasan dengan laut Arab. Perbedaan yang terjadi pada kedua wilayah ini tampak jelas di musim dingin. \textit{Upwelling} dan konveksi musim dingin meningkatkan biomassa fitoplankton di Laut Arab, tetapi sangat lemah atau bahkan tidak ada di BoB. Demikian juga masukan nutrisi melalui aliran sungai ke BoB tidak cukup untuk meningkatkan fitoplankton \shortcite{Jyothibabu2021}. BoB memiki keunikan akibat instrusi air tawar dari curah hujan yang tinggi selama musim panas sebagai hasil penetrasi insolasi matahari di kolom air \cite{Kantha2019}. \shortciteNP{Srivastava2018} mensimulasikan model tanpa gaya angin dekat permukaan laut yang mengakibatkan nilai SST di wilayah tersebut meningkat di semua musim. Disisi lain, simulasi tanpa input gaya radiasi gelombang pendek justru mendapatkan hasil yang berlawanan. Ditemukan juga bahwa pengaruh gaya fluks air tawar pada nilai SST di wilayah tersebut sangat kecil. Nilai SSS di laut Arab dan BoB menurun tanpa adanya gaya angin dekat permukaan dan radiasi gelombang pendek yang masuk, sedangkan di BoB utara meningkat tanpa adanya gaya fluks air tawar \shortcite{Srivastava2018}.
	
	BoB memiliki adveksi lateral kuat untuk air laut dengan nilai salinitas rendah. Adveksi ini berdampak pada pembentukan stratifikasi lapisan atas laut yang kuat (stratifikasi salinitas) dan memodifikasi lapisan campuran sehingga mempengaruhi kondisi SST dan SSS \shortcite{Buckley2020}. Dalam simulasi yang dilakukan di BoB utara selama peristiwa badai Siklon Titli pada tahun 2018, tercatat bahwa terjadi pendinginan SST (sekitar $2.0 - 2.5^\circ$C) dan peningkatan salinitas permukaan laut ($\sim$ 1 PSU) di sisi kanan jalur topan. Kondisi SST yang tinggi, TCHP (\textit{tropical cyclone heat potential}) dan kedalaman lapisan isotermal yang dalam adalah faktor laut pemicu intensifikasi badai Siklon Titli \shortcite{Akhter2022}. 
	
	\shortciteNP{Parida2022}, mengkaji tentang interaksi biofisik di BoB antara variable Chl-a dan SST serta hubungannya dengan pengaruh radiasi panas. Dari hasil observasi ditemukan bahwa kombinasi variabel Chl-a dan SST menentukan laju radiasi panas di kolom air. Dalam penelitian \shortciteNP{Chowdhury2021}, konsentrasi maksimum Chl-a di bawah permukaan (\textit{subsurface chlorophyll-a maximum} (SCM)) lebih tinggi selama musim panas dan awal musim gugur dibandingkan musim lainnya, terutama di sepanjang wilayah pesisir dan bagian barat BoB. Dalam periode ini, input nutrisi sungai, intrusi nutrisi air dari Laut Arab, dan \textit{upwelling} pesisir adalah tiga faktor dominan yang mempengaruhi konsentrasi Chl-a di permukaan dan SCM. Pendangkalan lapisan termoklin dipengaruhi oleh tekanan angin positif. Hal ini meningkatkan pasokan nutrisi dan secara signifikan meningkatkan konsentrasi Chl-a pada SCM di sepanjang sisi barat teluk.
	
	Iklim di BoB didominasi oleh monsun. Monsun timur laut (\textit{northeast monsoon}) terjadi antara bulan November hingga Februari, yang merupakan pertanda musim dingin (\textit{winter season}). Sebaliknya, monsun barat daya (\textit{southwest monsoon}) terjadi antara bulan Juni hingga September, yang merupakan pertanda musim panas (\textit{summer season}) \shortcite{Gadgil1984,Goswami2016}. Dalam kaitannya dengan MLD, diketahui bahwa angin memiliki dampak secara langsung. Disisi lain, presipitasi menunjukkan dampak tidak langsung terhadap MLD. Presipitasi membutuhkan waktu untuk mempengaruhi pendalaman dan pendangkalan MLD. Waktu yang diperlukan untuk presipitasi adalah dua bulan sebelum terjadi perubahan MLD \shortcite{Ikhwan2022}. Dalam penelitian \shortciteNP{Kumar2019}, pendinginan SST dan pengaruh terhadap MLD di BoB dikaji lebih lanjut selama peristiwa badai Roanu pada tahun 2016. Nilai SST selama peristiwa ini mencapai $1.5-2^\circ$C, sedangkan MLD bernilai $\sim$ 35 m. 
	
	Perbedaan nilai MLD secara signifikan terlihat di bagian selatan perairan yang dipisahkan oleh garis lintang pantai timur India (\textit{East India Coastal Current} (EICC)), area yang didominasi oleh aktivitas pusaran antisiklon. Pendangkalan lapisan campuran dan peningkatan stratifikasi dengan pengaruh \textit{relative wind} (RW) berkaitan dengan dominasi isopiknal oleh kecepatan Ekman ke atas yang tidak normal. Hal ini diakibatkan oleh interaksi arus permukaan antisiklonik dan angin monsun barat daya \shortcite{Seo2019}. Studi sebelumnya menunjukkan bahwa suhu lapisan campuran atau SST di BoB selama musim panas ditentukan oleh fluks udara-laut \shortcite{Jain2021}. 
	
	Dampak angin kencang dirasakan pada kedalaman yang lebih besar untuk suhu dibandingkan dengan salinitas di seluruh domain. Hal ini terlihat dari distribusi vertikal yang berbeda di BoB utara daripada di BoB selatan. Peningkatan pencampuran yang disebabkan oleh angin kencang menurunkan (atau meningkatkan) suhu laut bagian atas sebesar $0.2^\circ$C dan salinitas sebesar 0.3 PSU, dan melemahkan stratifikasi dekat-permukaan. Selain itu, angin kencang meningkatkan aktivitas pusaran air, memperkuat arus batas barat (\textit{Western Boundary Current} (WBC)) di musim semi dan meningkatkan \textit{upwelling} pantai selama musim semi dan musim panas di sepanjang pantai timur India \shortcite{Jana2018}. Berdasarkan inversi termal, rata-rata profil BoB barat laut memiliki lapisan campuran (MLD 10.30 m) dan lapisan isotermal (ILD 8.40 m) yang lebih dalam dibandingkan profil di BoB timur laut. Lapisan penghalang di BoB barat laut juga lebih tebal (2.79 m) daripada di BoB timur laut (1.05 m). Salah satu alasan yang mungkin untuk perbedaan ini adalah masuknya air tawar besar-besaran di BoB barat laut, karena air tawar mengurangi salinitas (27 PSU di BoB barat laut dan 35 PSU di BoB timur laut) dan menghasilkan MLD dan ILD yang lebih dangkal \shortcite{Masud-Ul-Alam2022}. 
		
	Masuknya air tawar yang besar berkaitan erat dengan MLD yang dangkal, pembentukan lapisan penghalang yang tebal, sirkulasi yang kuat dan pembalikan suhu \shortcite{Dandapat2020}. Korelasi parsial menunjukkan bahwa fluks panas bersih (Qnet) adalah kontributor utama pendalaman MLD di BoB utara, sedangkan tekanan angin mengontrol pendalaman MLD di BoB selatan. Variabilitas musiman menunjukkan pendalaman MLD selama monsun musim panas dan musim dingin dan pendangkalan MLD selama pra dan pasca monsun di BoB \shortcite{Sadhukhan2021}. Perubahan MLD yang diamati di BoB dengan jelas membatasi wilayah utara-selatan yang berbeda dengan $15^\circ$LU sebagai garis lintang pembatas. Bagian utara dari garis lintang ini menunjukkan bahwa MLD tetap dangkal ($\sim$20 m) selama hampir sepanjang tahun (tanpa menunjukkan perbedaan musim yang berarti). Perbedaan musim yang kurang terlihat menunjukkan bahwa air salinitas rendah di BoB utara, mengontrol pendangkalan MLD. Sebaliknya, bagian selatan menunjukkan variabilitas semi-tahunan yang kuat dengan pendalaman MLD selama musim panas dan musim dingin. Pendangkalan MLD terjadi selama musim semi dan musim gugur. Pendangkalan MLD di musim semi dan musim gugur dihasilkan dari pemanasan primer dan sekunder yang terkait dengan peningkatan radiasi matahari yang masuk dan kecepatan angin yang lebih lambat selama periode ini. Pendalaman MLD selama musim panas dihasilkan dari dua proses: peningkatan kekuatan angin dan intrusi air salinitas tinggi yang berasal dari Laut Arab \shortcite{Narvekar2006}.
	
	Penelitian tentang MLD di wilayah Samudra Hindia, khususnya laut Andaman dan perairan Aceh sendiri, masih jarang dilakukan. \shortciteNP{Ikhwan2022} meneliti tentang MLD di Laut Andaman dengan menggunakan data salinitas dari model CMEMS. Sinyal musiman digambarkan dengan data angin, presipitasi, temperatur, dan salinitas selama 26 tahun untuk mengidentifikasi jumlah musim MLD dalam setahun. Dari hasil penelitian diperoleh bahwa perbedaan kedalaman lapisan campuran di Laut Andaman dipengaruhi oleh angin dan presipitasi. Disisi lain, \shortciteNP{Yunita2021} mengkaji tentang MLD berdasarkan temperatur dan angin permukaan laut di perairan Aceh utara pada tahun 2017. Dengan membandingkan hasil data output model CMEMS dengan data Aqua MODIS, hasil penelitian menunjukkan bahwa kedua model relatif sama dengan variasi hingga $2^\circ$C. Sebagai tambahan, Verifikasi data SST CMEMS bulan Februari, April, Agustus dan Oktober menunjukkan hasil yang cukup baik dengan nilai korelasi r = 0,8523. Analisis yang dilakukan menunjukkan bahwa MLD terdalam terjadi pada bulan Februari, Agustus, dan Oktober. Lebih lanjut, MLD di perairan utara Aceh adalah 68-91 meter, dan perairan Sabang dan Krueng Raya 68-79 meter. 
	
	Dari beberapa penelitian yang telah disebutkan di atas, kajian mengenai kontribusi parameter meteorologi: \textit{2m air temperature, 2m specific humidity, convective precipitation rate, sea level pressure, wind stress U}, dan \textit{wind stress V} terhadap variabilitas MLD menggunakan data output model resolusi tinggi untuk jangka panjang belum pernah dilakukan sebelumnya, oleh karena itu penelitian ini bertujuan untuk menginvestigasi MLD berdasarkan parameter meteorologi yang telah disebutkan sebelumnya. 
	
	\section[Rumusan Masalah]{Rumusan Masalah}
	Pada latar belakang, telah diuraikan penelitian-penelitian terkait MLD dan pentingnya MLD dalam menggambarkan iklim global. Telah dijelaskan pula hal-hal yang dilakukan dalam penelitian ini. Fokus penelitian tesis ini adalah mengkaji pengaruh parameter meteorologi terhadap kedalaman lapisan campuran (\textit{Mixed Layer Depth}) di perairan Samudra Hindia.
	
	Subpertanyaan berikut akan berkontribusi pada perumusan jawaban atas masalah utama.
	\begin{itemize}
		\item Bagaimana analisis kedalaman lapisan campuran (MLD) di wilayah Samudra Hindia dalam 12 bulan pada tahun 2021? 
		\item Bagaimana analisis model iklim untuk parameter-parameter meteorologi \textit{2m air temperature, 2m specific humidity, convective precipitation rate, sea level pressure, wind stress U}, dan \textit{wind stress V} selama 20 tahun, tahun 2002 - 2021?
		\item Bagaimana hubungan parameter meteorologi terhadap analisis kedalaman lapisan campuran (MLD) di wilayah Samudra Hindia?
	\end{itemize}	
	\section[Tujuan Penelitian]{Tujuan Penelitian}
	
	Tujuan penelitian tesis ini adalah menyelidiki pengaruh parameter meteorologi terhadap kedalaman lapisan campuran (\textit{Mixed Layer Depth}) di Samudra Hindia dengan cara menjawab beberapa masalah terkait,
	
	\begin{itemize}
		\item Analisis kedalaman lapisan campuran (MLD) di wilayah Samudra Hindia dalam 12 bulan pada tahun 2021.
		\item Analisis model iklim untuk parameter-parameter meteorologi \textit{2m air temperature, 2m specific humidity, convective precipitation rate, sea level pressure, wind stress U}, dan \textit{wind stress V} selama 22 tahun, tahun 2000 - 2021.
		\item Hubungan parameter meteorologi terhadap analisis kedalaman lapisan campuran (MLD) di wilayah Samudra Hindia.
	\end{itemize}
	
	\section[Urgensi dan Kebaruan Penelitian]{Urgensi dan Kebaruan Penelitian}

	Interaksi antara laut dan atmosfer, adalah proses yang kompleks dan nonlinier. Angin yang merupakan salah satu parameter meteorologi, yang bertiup di atas permukaan laut misalnya, mentransfer momentum dan energi mekanik ke air, menghasilkan gelombang dan arus. Lautan pada gilirannya mengeluarkan energi sebagai panas, dengan emisi radiasi elektromagnetik, dengan konduksi, dan dalam bentuk laten, dengan penguapan. Fluks panas dari laut menyediakan salah satu sumber energi utama untuk gerakan atmosfer. Sumber energi untuk atmosfer ini dipengaruhi oleh turbulensi pada antarmuka udara/laut, dan oleh distribusi spasial pusat-pusat transfer energi tinggi dan rendah yang dipengaruhi oleh arus laut. Kopling ini terjadi melalui proses yang pada dasarnya terjadi pada skala kecil. Kekuatan kopling ini tergantung pada Interaksi antara laut dan atmosfer. Dengan demikian, semua parameter meteorologi, yang berhubungan langsung dengan laut, akan memiliki kontribusi terhadap lapisan MLD, dan tentu saja memiliki skala geografis dan temporal pada rentang yang luas \shortcite{Inniss2017}.
	
	Lapisan campuran dapat didasarkan pada parameter yang berbeda (misalnya, temperature), dan dapat mewakili rata-rata selama interval waktu yang berbeda (misalnya, hari, bulan) \shortcite{DeBoyerMontegut2004}. Dari penelitian sebelumnya diketahui bahwa MLD memiliki dampak terhadap iklim, yakni: dalam hal pengaturan respons rata-rata \textit{sea surface temperature} tropis (\textit{tropical mean SST}) terhadap pemanasan global \shortcite{Yeh2009}, pendangkalan MLD di musim dingin \shortcite{Richards2021}, dan pengaruh pada besarnya pemanasan permukaan laut (\textit{sea surface warming}) \shortcite{Hwang2017}. \textit{Upwelling} pantai memainkan peran penting dalam iklim regional melalui interaksi antara udara dan lapisan campuran (ML) \shortcite{Bessa2020,Toualy2022}. Oleh karena itu, kajian MLD lebih lanjut diperlukan untuk memahami pengaruh dan interaksi yang dapat terjadi pada MLD.
	
	Dalam penelitian ini, kajian MLD difokuskan pada temperature laut. Diketahui bahwa MLD, SST, kecepatan angin, dan variable lainnya dapat digunakan untuk menentukan perubahan musiman, sedangkan MLD rata-rata tidak berubah \shortcite{Muller-Karger2015}. Beberapa penelitian lain telah mengkaji tentang MLD dengan menggunakan temperature laut, diantaranya adalah, penelitian tentang MLD dengan menggunakan temperature dengan metode ambang batas (\textit{threshold method}) $\Delta t=0.2^\circ C$ \shortcite{Jeong2019}. Dengan ambang batas yang sama, dapat diketahui siklus musiman dan pola spasial (\textit{seasonal cycle and spatial patterns}) dari MLD \shortcite{Cai2021}. Kriteria \textit{threshold} yang digunakan dapat bervariasi, juga dapat di bandingkan dengan metode gradient. Penentuan MLD dengan temperature, dengan kriteria ambang batas  $1^\circ C$ merupakan metode yang terbaik dalam wilayah spesifik setelah dibandingkan dengan beberapa nilai kriteria dan nilai gradient \shortcite{Nahavandian2022}. Disisi lain, estimasi optimal penetrasi pencampuran turbulen diperoleh dengan menggunakan kriteria \textit{threshold}, $0.8^\circ C$ \shortcite{Kara2000}.
	
	Kajian pengaruh parameter meteorology terhadap MLD telah sering dilakukan sebelumnya. Beberapa penelitian diantaranya yang terkait hal ini adalah, \citeNP{Toualy2022} mengidentifikasi variabel angin dalam variasi MLD, dimana ML yang lebih dangkal dan laut produktif yang tinggi diamati selama periode pendinginan sementara MLD yang lebih dalam dan lautan yang kurang produktif terjadi selama musim panas. \citeNP{Kantha2019}, menggunakan data \textit{mooring Woods Hole Oceanographic Institution} (WHOI), yaitu: \textit{wind speed, wind direction, air temperature, relative humidity, precipitation, sea surface temperature}, dan beberapa data lain untuk menggerakkan model pencampuran satu dimensi, berdasarkan model turbulensi penutupan momen kedua, untuk mengeksplorasi variabilitas intra-tahunan di lapisan atas. Lebih lanjut, diketahui juga bahwa angin memiliki dampak langsung terhadap MLD dimana kekuatan angin mempengaruhi secara simultan kondisi MLD. Disisi lain, presipitasi menunjukkan dampak tidak langsung pada MLD. Curah hujan membutuhkan waktu untuk mengumpulkan efek untuk mengubah keadaan MLD. Waktu yang diperlukan untuk presipitasi adalah dua bulan sebelum terjadi perubahan MLD \shortcite{Ikhwan2022}. 
	
	Sejauh pengamatan kami, studi secara detail terkait 6 parameter meteorologi dan dampaknya terhadap lapisan vertikal di wilayah perairan Samudra Hindia belum pernah dilakukan sebelumnya. Oleh karena itu, dirasa penting untuk melakukan penelitian ini guna mengetahui pengaruh paramater meteorologi terhadap kedalaman lapisan campuran (MLD).

	\section[Manfaat Penelitian]{Manfaat Penelitian}
	
	Penelitian ini diharapkan mampu memberikan kontribusi ilmiah dan memperkaya pengetahuan tentang kedalaman lapisan campuran atau MLD. Hal ini karena MLD berperan penting secara iklim fisik dalam hal menentukan interval kisaran temperatur di wilayah laut dan pesisir. Sebagai tambahan, panas yang tersimpan dalam lapisan campuran menyediakan sumber panas yang mendorong variabilitas global seperti El Ni$\tilde{n}$o. MLD juga berperan dalam menentukan tingkatan rata-rata cahaya yang dapat dilihat oleh organisme laut seperti fitoplankton. Selain itu, dari periodesitas model iklim yang diperoleh akan bermanfaat untuk tujuan fishing ground, mitigasi perubahan iklim dan bencana hidro-oseanografi, tata ruang dan konservasi
	laut, dan sumber energi terbarukan. 

	\section[Sistematika Penulisan]{Sistematika Penulisan}

	Tesis ini tersusun atas 5 bab. Bab pertama menjelaskan pendahuluan tentang latar belakang mengapa penelitian ini dilakukan, background masalah yang mendasari, tujuan penelitian, manfaat penelitian, serta kebaruan dari penelitian. Bab kedua berisikan tinjauan pustaka menyangkut ulasan singkat materi penelitian. Bab ketiga membahas tentang metode penelitian yang dilakukan, data yang yang digunakan, serta diagram alir (\textit{flowchart}) dari penelitian. Bab keempat membahas hasil dan pembahasan penelitian. Terakhir, bab kelima membahas tentang kesimpulan dari penelitian.
	
\end{spacing}