%\vspace{1.5pc}
\vspace{1.5pc}
\section[Latar Belakang]{Latar Belakang}
\begin{spacing}{1.5}
	Samudera Hindia adalah samudera terbesar ketiga di dunia, meliputi sekitar 19.8\% dari total volume lautan \shortcite{Eakins2010} dan merupakan lautan yang sangat berpengaruh bagi ekosistem di Bumi. Cakupan wilayah dari Samudera Hindia termasuk di dalamnya Teluk Benggala (\textit{Bay of Bengal} (BoB)), Laut Andaman, Selat Malaka, dan Perairan Aceh. Dengan cakupan wilayah yang begitu luas, Samudera Hindia merupakan penyumbang besar bagi sistem iklim dunia dan oleh karena itu sangat penting untuk dapat diprediksi. Pengembangan model kelautan berusaha untuk menggambarkan iklim global dengan cukup baik disertai dengan pengamatan. Namun, variabilitas spasial dan temporal perlu dipahami untuk prediksi yang lebih baik. Pemanasan matahari dan kekuatan angin bervariasi dalam ruang dan waktu yang akan tercermin dalam variabilitas lapisan campuran laut dan suhu permukaan. Oleh karena itu, peran gaya atmosfer lokal pada variabilitas lapisan campuran dan, karenanya, pada suhu permukaan laut menjadi fokus dari tesis ini.
	
	Beberapa studi observasional dan pemodelan telah dilakukan untuk mempelajari pengaruh interaksi atmosfer-laut terhadap variabilitas suhu permukaan laut (SST), salinitas permukaan laut (SSS), klorofil-a (chl-a), kedalaman lapisan campuran (MLD) dan sirkulasi pada wilayah perairan Samudera Hindia, diantaranya adalah,	pencampuran turbulen di lapisan atas BoB utara dipengaruhi oleh lapisan dangkal yang menutupi perairan asin teluk, yang dihasilkan dari arus besar air tawar dari sungai-sungai besar yang mengalir dari anak benua Asia dan dari curah hujan di atas teluk selama musim panas \shortcite{Kantha2019}. Karena BoB juga berbatasan dengan laut Arab, perbedaan sering terjadi pada musim dingin, yaitu upwelling dan konveksi musim dingin, yang meningkatkan biomassa fitoplankton di Laut Arab, tetapi sangat lemah atau bahkan tidak ada di BoB. Demikian pula, masukan nutrisi melalui aliran sungai ke BoB tidak cukup untuk meningkatkan stok fitoplankton di luar perairan \shortcite{Jyothibabu2021}. BoB memiki keunikan akibat instrusi air tawar dari curah hujan yang tinggi selama musim panas sebagai hasil penetrasi insolasi matahari di kolom air \cite{Kantha2019}. \shortciteNP{Srivastava2018} mensimulasikan model tanpa gaya angin dekat permukaan, hasilnya adalah SST (\textit{Sea Surface Temperature}) wilayah tersebut sangat meningkat di semua musim, sedangkan, tanpa adanya gaya radiasi gelombang pendek yang masuk, mereka mendapatkan hasil yang benar-benar berlawanan. Ditemukan bahwa pengaruh pemaksaan fluks air tawar pada SST wilayah tersebut sangat kecil. Ditemukan juga bahwa SSS (\textit{Sea Surface Salinity}) laut Arab dan BoB menurun tanpa adanya gaya angin dekat permukaan dan radiasi gelombang pendek yang masuk, sedangkan di BoB utara meningkat tanpa adanya gaya fluks air tawar \shortcite{Srivastava2018}.
	
	Adveksi lateral yang kuat dari air salinitas rendah mengarah pada pengembangan stratifikasi laut atas yang kuat (stratifikasi salinitas), yang dapat berdampak signifikan pada evolusi SST dan SSS dengan memodifikasi pencampuran di dekat permukaan. Fluks udara-laut tidak cukup untuk mensimulasikan evolusi SST dengan benar di BoB utara, dan bahwa penghitungan adveksi air tawar diperlukan untuk mengurangi kesalahan dalam SST \shortcite{Buckley2020}. Pendinginan SST yang nyata (sekitar $2.0 - 2.5^\circ$C) dan peningkatan salinitas permukaan laut ($\sim$ 1 psu) di sisi kanan jalur topan. SST yang tinggi, TCHP (\textit{tropical cyclone heat potential}) dan kedalaman lapisan isotermal yang dalam adalah kekuatan pemicu samudera utama untuk mengintensifkan siklon Titli \shortcite{Akhter2022}. 
	
	Pengaruh radiasi panas terhadap lapisan permukaan batas BoB tergantung pada variabel biologis (Chl-a atau Klorofil-a) dan fisik (panas). Pemanasan biologis $10\; \text{Wm}^{-2}$ akan menghasilkan pemanasan tambahan $0,008^\circ \text{C jam}^{-1}$ di laut bagian atas yang menunjukkan dampak signifikan dari peningkatan konsentrasi chl-a \shortcite{Parida2022}. Konsentrasi maksimum klorofil-a di permukaan dan di bawah permukaan atau \textit{subsurface chlorophyll-a maximum} (SCM) lebih tinggi selama musim panas dan awal musim gugur dibandingkan musim lainnya, terutama di sepanjang wilayah pesisir dan bagian barat BoB. Selama musim panas dan awal musim gugur, masukan nutrisi sungai, intrusi air bergizi dari Laut Arab, dan upwelling pesisir adalah tiga pendorong dominan yang mengendalikan konsentrasi klorofil-a di permukaan dan SCM. Pengangkatan termoklin yang diinduksi oleh tegangan angin positif meningkatkan pasokan nutrisi dan dengan demikian secara signifikan meningkatkan konsentrasi klorofil-a di SCM di sepanjang sisi barat teluk selama paruh kedua tahun ini. Selama musim semi, kedalaman eufotik yang dalam memainkan peran penting dalam mengendalikan konsentrasi dan kedalaman SCM \shortcite{Chowdhury2021}.
	
	Kekuatan angin mempengaruhi secara simultan kondisi MLD. Presipitasi menunjukkan dampak tidak langsung pada MLD. Curah hujan membutuhkan waktu untuk mengumpulkan efek untuk mengubah keadaan MLD. Waktu yang diperlukan untuk presipitasi adalah dua bulan sebelum terjadi perubahan MLD \shortcite{Ikhwan2022}. Pendinginan terutama disebabkan oleh pencampuran hangat ($32^\circ$C), tutupan segar yang terbentuk selama bulan-bulan sebelumnya dari angin sepoi-sepoi dan langit cerah, yang menyumbang sekitar setengah dari pendinginan. Fluks panas udara-laut memainkan peran sekunder, terhitung sekitar seperempat dari pendinginan. Kedalaman pencampuran didiagnosis dengan dua ukuran: kedalaman lapisan campuran tradisional dan "kedalaman pencampuran" yang didefinisikan sebagai kedalaman terdalam yang tidak stabil terhadap ketidakstabilan geser. Kedalaman pencampuran kira-kira dua kali ($\sim$ 65 m) dari kedalaman lapisan campuran ($\sim$ 35 m), yang menggambarkan pentingnya "lapisan transisi" di antara mereka. Lapisan campuran diratifikasi kembali menjadi 2 lapisan dalam sehari setelah badai berakhir dengan gelombang frekuensi mendekati inersia yang ditimbulkan oleh badai Roanu meningkatkan laju pencampuran diapiknal pada kedalaman lapisan transisi \shortcite{Kumar2019}. 
	
	MLD secara signifikan terdampar di selatan garis lintang pantai timur India (EICC) yang terpisah, area yang didominasi oleh aktivitas pusaran antisiklon. Lapisan campuran yang lebih dangkal dan stratifikasi yang ditingkatkan dengan efek relative wind (RW) dikaitkan dengan dominasi isopiknal oleh kecepatan Ekman ke atas yang tidak normal, yang dengan sendirinya dihasilkan oleh interaksi arus permukaan antisiklonik dan angin monsun barat daya yang berlaku. Bagian barat daya BoB ini merupakan titik panas untuk pertukaran momentum antara sirkulasi permukaan dan angin monsun, sehingga merupakan area potensial untuk pengukuran lapangan terfokus untuk energetika sirkulasi laut dan interaksi udara-laut \shortcite{Seo2019}. 	MLD yang sebenarnya tidak hanya bergantung pada kecepatan angin, tapi salinitas juga berperan di teluk utara. Namun, ada perubahan yang dapat diabaikan dalam SST bahkan ketika MLD berubah secara signifikan karena termoklin dalam memisahkan perubahan MLD dan SST. Sebaliknya, termoklin yang lebih dangkal di teluk barat membatasi potensi MLD, yang menyebabkan perubahan SST yang lebih besar. Gelombang Rossby upwelling (atau downwelling) pada dasarnya mengkondisikan laut bagian atas dengan mengurangi (atau meningkatkan) potensi MLD. Variasi SST melemah hanya ketika termoklin semakin dalam selama peristiwa downwelling, yang terjadi kemudian di teluk barat karena gelombang Rossby merambat ke barat \shortcite{Jain2021}. 
	
	Dampak angin kencang dirasakan pada kedalaman yang lebih besar untuk suhu daripada salinitas di seluruh domain; namun, dampaknya diwujudkan dengan distribusi vertikal yang berbeda di bagian utara daripada di bagian selatan Teluk. Seperti yang diharapkan, pencampuran yang ditingkatkan yang disebabkan oleh angin yang lebih kuat menurunkan (atau meningkatkan) suhu laut bagian atas (salinitas) sebesar $0.2^\circ$C (0.3 psu), dan melemahkan stratifikasi dekat-permukaan. Selain itu, angin yang lebih kuat meningkatkan aktivitas pusaran air, memperkuat arus batas barat musim semi dan meningkatkan upwelling pantai selama musim semi dan musim panas di sepanjang pantai timur India \shortcite{Jana2018}. Berdasarkan inversi termal, rata-rata profil BoB barat laut memiliki lapisan campuran yang lebih dalam (MLD 10.30 m) dan lapisan isotermal (ILD 8.40 m) dibandingkan profil di BoB timur laut. Lapisan penghalang di BoB barat laut juga lebih tebal (2.79 m) daripada di BoB timur laut (1.05 m). Salah satu alasan yang mungkin untuk perbedaan ini adalah masuknya air tawar besar-besaran di BoB barat laut, karena air tawar mengurangi salinitas (27 PSU di BoB barat laut versus 35 PSU di BoB timur laut) dan menghasilkan MLD dan ILD yang lebih dangkal \shortcite{Masud-Ul-Alam2022}.  Stratifikasi dan lapisan depan lapisan campuran berkembang dalam skala waktu yang relatif singkat, kemungkinan sebagai respons terhadap kekuatan atmosfer yang kuat baik yang terkait dengan siklon tropis, kondisi monsun timur laut yang berkelanjutan, atau kombinasi keduanya \shortcite{Shroyer2020}. 
		
	Masuknya air tawar yang besar berkaitan erat dengan MLD yang dangkal, pembentukan lapisan penghalang yang tebal, dan sirkulasi yang kuat dan pembalikan suhu \shortcite{Dandapat2020}. 	Korelasi parsial menunjukkan bahwa fluks panas bersih (Qnet) adalah kontributor utama pendalaman MLD pada BoB utara, sedangkan tekanan angin mengontrol pendalaman atas BOB selatan. Variabilitas musiman menunjukkan pendalaman MLD selama monsun musim panas dan musim dingin dan pendangkalan selama pra dan pasca monsun di atas BoB \shortcite{Sadhukhan2021}. Perubahan yang diamati pada MLD dengan jelas membatasi rezim utara-selatan yang berbeda dengan $15^\circ$LU sebagai garis lintang pembatas. Utara dari garis lintang ini MLD tetap dangkal ($\sim$20 m) hampir sepanjang tahun tanpa menunjukkan musim yang berarti. Kurangnya musim menunjukkan bahwa air salinitas rendah, yang selalu ada di teluk utara, mengontrol stabilitas dan MLD. Penyegaran musim dingin yang diamati didorong oleh curah hujan musim dingin dan debit sungai terkait, yang didorong ke lepas pantai di bawah sirkulasi yang berlaku. Stratifikasi yang dihasilkan begitu kuat sehingga bahkan pendinginan $4^\circ$C pada suhu permukaan laut (SST) selama musim dingin tidak dapat memulai pencampuran konvektif. Sebaliknya, wilayah selatan menunjukkan variabilitas semi-tahunan yang kuat dengan MLD yang dalam selama musim panas dan musim dingin dan MLD yang dangkal selama musim semi dan musim gugur. MLD dangkal di musim semi dan musim gugur dihasilkan dari pemanasan primer dan sekunder yang terkait dengan peningkatan radiasi matahari yang masuk dan angin yang lebih ringan selama periode ini. Lapisan campuran yang dalam selama musim panas dihasilkan dari dua proses: peningkatan kekuatan angin dan intrusi air salinitas tinggi yang berasal dari Laut Arab \shortcite{Narvekar2006}.
	
	Dari beberapa penelitian yang telah disebutkan di atas, kajian mengenai kontribusi parameter meteorologi: \textit{2m air temperature, 2m specific humidity, convective precipitation rate, sea level pressure, wind stress U}, dan \textit{wind stress V} terhadap variabilitas MLD menggunakan data output model resolusi tinggi untuk jangka panjang belum pernah dilakukan sebelumnya khususnya untuk wilayah perairan Aceh, oleh karena itu penelitian ini bertujuan untuk menginvestigasi MLD berdasarkan parameter meteorologi yang telah disebutkan sebelumnya. Analisis dengan model iklim ditekankan sebagai verifikasi untuk observasi MLD yang dilakukan pada sampel stasiun wilayah peneltian. Pada akhirnya, dari hasil analisis yang dilakukan akan diperoleh hubungan antara parameter meteorologi dan MLD.
	
	\section[Rumusan Masalah]{Rumusan Masalah}
	Pada latar belakang, telah diuraikan penelitian-penelitian terkait MLD dan mengapa MLD penting untuk menggambarkan iklim global. Telah dijelaskan pula secara ringkas mengenai hal-hal apa saja yang akan dilakukan dalam penelitian ini. Fokus dari penelitian tesis ini adalah menjawab masalah utama, yaitu
	
	Bagaimana pengaruh parameter meteorologi terhadap kedalaman lapisan campuran (\textit{Mixed Layer Depth}) di Perairan Aceh?
	
	Subpertanyaan berikut akan berkontribusi pada perumusan jawaban atas masalah utama.
	\begin{itemize}
		\item Bagaimana analisis kedalaman lapisan campuran (MLD) di wilayah perairan Aceh dalam 12 bulan pada tahun 2021? 
		\item Bagaimana analisis model iklim untuk parameter-parameter meteorologi \textit{2m air temperature, 2m specific humidity, convective precipitation rate, sea level pressure, wind stress U}, dan \textit{wind stress V} selama 22 tahun, tahun 2000 - 2021?
		\item Bagaimana hubungan parameter meteorologi terhadap analisis kedalaman lapisan campuran (MLD) di wilayah perairan Aceh?
	\end{itemize}
	\section[Tujuan Penelitian]{Tujuan Penelitian}
	
	Tujuan dari penelitian tesis ini adalah mencari tahu pengaruh parameter meteorologi terhadap kedalaman lapisan campuran (\textit{Mixed Layer Depth}) di Perairan Aceh dengan cara menjawab beberapa masalah terkait,
	
	\begin{itemize}
		\item Analisis kedalaman lapisan campuran (MLD) di wilayah perairan Aceh dalam 12 bulan pada tahun 2021.
		\item Analisis model iklim untuk parameter-parameter meteorologi \textit{2m air temperature, 2m specific humidity, convective precipitation rate, sea level pressure, wind stress U}, dan \textit{wind stress V} selama 22 tahun, tahun 2000 - 2021.
		\item Hubungan parameter meteorologi terhadap analisis kedalaman lapisan campuran (MLD) di wilayah perairan Aceh.
	\end{itemize}
	
	\section[Urgensi dan Kebaruan Penelitian]{Urgensi dan Kebaruan Penelitian}

	Sejauh pengamatan kami, studi secara detail terkait 6 parameter meteorologi dan dampaknya terhadap lapisan vertikal di wilayah perairan Aceh belum pernah dilakukan sebelumnya. Oleh karena itu, dirasa penting untuk melakukan penelitian ini guna mengetahui pengaruh paramater meteorologi terhadap kedalaman lapisan campuran (MLD).

	\section[Manfaat Penelitian]{Manfaat Penelitian}
	
	Penelitian ini diharapkan mampu memberikan kontribusi ilmiah dan memperkaya pengetahuan tentang kedalaman lapisan campuran atau MLD. Hal ini karena MLD berperan penting secara iklim fisik dalam hal menentukan interval kisaran temperatur di wilayah laut dan pesisir. Sebagai tambahan, panas yang tersimpan dalam lapisan campuran menyediakan sumber panas yang mendorong variabilitas global seperti El Ni$\tilde{n}$o. MLD juga berperan dalam menentukan tingkatan rata-rata cahaya yang dapat dilihat oleh organisme laut seperti fitoplankton. Selain itu, dari periodesitas model iklim yang diperoleh akan bermanfaat untuk tujuan fishing ground, mitigasi perubahan iklim dan bencana hidro-oseanografi, tata ruang dan konservasi
	laut, dan sumber energi terbarukan. 

	\section[Sistematika Penulisan]{Sistematika Penulisan}

	Tesis ini tersusun atas 5 bab. Bab pertama menjelaskan pendahuluan tentang latar belakang mengapa penelitian ini dilakukan, background masalah yang mendasari, tujuan penelitian, manfaat penelitian, serta kebaruan dari penelitian. Bab kedua berisikan tinjauan pustaka menyangkut ulasan singkat materi penelitian. Bab ketiga membahas tentang metode penelitian yang dilakukan, data yang yang digunakan, serta diagram alir (\textit{flowchart}) dari penelitian. Bab keempat membahas hasil dan pembahasan penelitian. Terakhir, bab kelima membahas tentang kesimpulan dari penelitian.
	
\end{spacing}