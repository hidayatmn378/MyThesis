\vspace{1.5pc}
\begin{spacing}{1.5}
\section[Kesimpulan]{KESIMPULAN}
\subsection[MLD]{MLD}

	Deskripsi siklus musiman MLD berbasis-temperature telah dikonstruksi dengan menggunakan data temperature dari HYCOM. Definisi MLD yang digunakan dalam penelitian mengikuti metode \textit{threshold} $0.1^\circ$C, dengan referensi dari permukaan laut. Kemudian dari parameter-parameter meteorology NCEP/NCAR, dibentuk model iklim untuk mengidentifikasi parameter yang mempengaruhi ketebalan MLD. Model iklim ini sangat berguna untuk mengestimasi puncak dan lembah dari parameter-parameter meteorologi \shortcite{Toharudin2021,Yates2020,Ikhwan2022,Haridhi2016}. 
	
	Nilai estimasi MLD di kedua lintang pada Table \ref{table:MLD_thickness} menunjukkan bahwa variasi MLD pada lintang $19^\circ$N lebih besar dibandingkan lintang $9^\circ$N. Terlebih, ketebalan MLD antara 2 lintang tidak menujukkan posisi dalam bulan yang sama. MLD yang dangkal pada lintang $9^\circ$N terletak pada bulan Maret and Desember, sementara itu MLD yang dangkal pada lintang $19^\circ$N terletak pada bulan Maret, Mei, and Juni. Di sisi lain MLD paling dangkal pada lintang $9^\circ$N terletak pada bulan Februari and November, sedangkan MLD untuk lintang $19^\circ$N terletak pada bulan Maret and April.
	
	Hasil penelitian ini menunjukkan bahwa peran musim sangat tinggi. Hal ini terlihat dari rata-rata nilai MLD batas bawah (MinX) dan batas atas (MaxX) yang ditunjukkan pada Tabel \ref{table:MLD_monsoon}, dimana MLD lebih tebal pada bulan-bulan monsun musim dingin (November-Februari) dibandingkan pada bulan-bulan monsun musim panas (Juni-September). Ini berlaku untuk kedua lintang: lintang $9^\circ$N dan lintang $19^\circ$N.
	
	Hasil model musiman menunjukkan bahwa 3 parameter yaitu temperature udara, kelembaban spesifik, dan tekanan angin terjadi ekstrim minimum pada bulan Desember dan Januari dan ekstrim maksimum pada bulan Juni dan Juli. Terkait dengan MLD, nilai minimum ekstrim dari 3 parameter ini memberikan kontribusi positif terhadap ketebalan MLD. Sementara nilai maksimum ekstrim membuat MLD menjadi lebih tipis di bulan-bulan musim panas. Ini berlaku untuk kedua garis lintang.
	
	Untuk 2 parameter lainnya, laju presipitasi konvektif dan tekanan permukaan laut yang ekstrim minimum terjadi pada bulan Juli dan Agustus, dan hal ini berkontribusi pada tipisnya lapisan MLD. Sementara itu, pada bulan Januari dan Februari, parameter tersebut terjadi pada tingkat maksimum yang ekstrem, dan ini berkontribusi pada MLD yang lebih tebal. Ini juga terjadi di kedua garis lintang.
	
	Model musiman untuk lima parameter meteorologi menunjukkan bahwa semua parameter yang mencapai ekstrim maksimum dan minimum hanya terjadi pada bulan Desember-Februari (musim dingin) dan Juni-Agustus (musim panas). Oleh karena itu, dapat disimpulkan bahwa semua parameter meteorologi mengikuti musim monsun.
	
\subsection[Analisis Chl-a, SST, dan SSS]{Analisis Chl-a, SST, dan SSS}
	Perbedaan yang mencolok antara klorofil-a, temperature permukaan laut, dan salinitas permukaan laut di dekat pantai khususnya di bagian utara domain dalam gambar \ref{fig:paper1_2}(a-c) menjadi landasan dalam menganalisis hubungan antara Chl-a - SST, Chl-a - SSS dan SST - SSS dengan menggunakan analisis-analisis statistik. Hasilnya adalah terdapat korelasi antara pasangan-pasangan variable tersebut. Korelasi sedang dan negative terjadi untuk pasangan Chl-a - SST, dan Chl-a - SSS dengan koefisien determinansi berturut-turut sebesar 25.11\% dan 45.98\% yang menunjukkan pengaruh Chl-a terhadap SST ataupun pengaruh Chl-a terhadap SSS. Sebaliknya, korelasi kuat dan positif ditunjukkan oleh pasangan SST - SSS dengan koefisien determinansi sebesar yang menunjukkan pengaruh SST terhadap SSS. Uji hipotesis yang dilakukan menunjukkan bahwa penarikan kesimpulan atas hubungan-hubungan korelasi ini dapat diterima. Lebih lanjut, dari analisis variansi yang diperoleh, ditemukan bahwa Chl-a berpengaruh nyata terhadap variasi SST dan SSS, SST berpengaruh nyata terhadap variasi SSS.
	
\subsection[\textit{Eddies} dan SSH]{\textit{Eddies} dan SSH}

	Dari penelitian yang dilakukan diperoleh hasil yaitu terjadi dua arus yang berlawanan arah jarum jam (\textit{counter-clockwise current}) di Perairan Aceh dan di bagian utara Selat Malaka pada Bulan April 2020. Karena wilayah tersebut terletak di \textit{Northern Hemisphere}, maka terjadi Ekman transport ke arah luar dari lingkaran \textit{eddies} sehingga terjadi kekosongan massa air pada lingkaran eddies tersebut. Karena kekosongan massa air, maka harus diisi dari bawah. Karakteristik air dari bawah yang dingin, menghasilkan SST yang dingin pula. Ini terkonfirmasi dari SST yang rendah dibandingkan daerah sekitarnya karena telah terjadi peristiwa \textit{upwelling}, pada dua \textit{eddies} yang dibahas sebelumnya.
	
\section[Saran]{SARAN}

	Kami merekomendasikan bahwa penyelidikan parameter meteorologi lainnya seperti fluks panas, radiasi gelombang panjang dan pendek, fluks air tawar, tutupan awan dan parameter lainnya harus dilanjutkan.

\end{spacing}