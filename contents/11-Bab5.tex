\vspace{1.5pc}
\begin{spacing}{1.5}
\section[Kesimpulan]{KESIMPULAN}
\subsection[MLD]{MLD}

	Deskripsi siklus musiman MLD berbasis-temperature telah dikonstruksi dengan menggunakan data temperature dari HYCOM. Definisi MLD yang digunakan dalam penelitian mengikuti metode threshold $0.1^\circ$C, dengan referensi dari permukaan laut. Kemudian dari parameter-parameter meteorology NCEP/NCAR, dibentuk model iklim untuk mengidentifikasi parameter yang mempengaruhi ketebalan MLD. Model iklim ini sangat berguna untuk mengestimasi puncak dan lembah dari parameter-parameter meteorologi \shortcite{Toharudin2021,Yates2020,Ikhwan2022,Haridhi2016}. 
	
	Nilai estimasi MLD di kedua latitude pada Table \ref{table:MLD_thickness} menunjukkan bahwa Variasi MLD pada Latitude $19^\circ$N lebih besar dibandingkan Latitude $9^\circ$N. Terlebih, ketebalan MLD antara 2 latitude tidak menujukkan posisi dalam bulan yang sama. MLD yang dangkal pada latitude $9^\circ$N terletak pada bulan Maret and Desember, sementara itu MLD yang dangkal pada latitude $19^\circ$N terletak pada bulan Maret, Mei, and Juni. Di sisi lain MLD paling dangkal pada latitude $9^\circ$N terletak pada bulan Februari and November, sedangkan MLD untuk latitude $19^\circ$N terletak pada bulan Maret and April.
	
	Hasil penelitian ini menunjukkan bahwa peran musim sangat tinggi. Hal ini terlihat dari rata-rata nilai MLD batas bawah (MinX) dan batas atas (MaxX) yang ditunjukkan pada Tabel \ref{table:MLD_monsoon}, dimana MLD lebih tebal pada bulan-bulan monsun musim dingin (November-Februari) dibandingkan pada bulan-bulan monsun musim panas (Juni-September). Ini berlaku untuk kedua lintang: lintang $9^\circ$N dan lintang $19^\circ$N.
	
	Hasil model musiman menunjukkan bahwa 3 parameter yaitu Temperature Udara, Kelembaban Spesifik, dan Tekanan Angin terjadi ekstrim minimum pada bulan Desember dan Januari dan ekstrim maksimum pada bulan Juni dan Juli. Terkait dengan MLD, nilai minimum ekstrim dari 3 parameter ini memberikan kontribusi positif terhadap ketebalan MLD. Sementara nilai maksimum ekstrim membuat MLD menjadi lebih tipis di bulan-bulan musim panas. Ini berlaku untuk kedua garis lintang.
	
	Untuk 2 parameter lainnya, Laju Presipitasi Konvektif dan tekanan permukaan laut yang ekstrim minimum terjadi pada bulan Juli dan Agustus, dan hal ini berkontribusi pada tipisnya lapisan MLD. Sementara itu, pada bulan Januari dan Februari, parameter tersebut terjadi pada tingkat maksimum yang ekstrem, dan ini berkontribusi pada MLD yang lebih tebal. Ini juga terjadi di kedua garis lintang.
	
	Model musiman untuk lima parameter meteorologi menunjukkan bahwa semua parameter yang mencapai ekstrim maksimum dan minimum hanya terjadi pada bulan Desember-Februari (musim dingin) dan Juni-Agustus (musim panas). Oleh karena itu, dapat disimpulkan bahwa semua parameter meteorologi mengikuti musim monsun.
	
%\subsection[Hasil Tambahan]{Hasil Tambahan}
	
\section[Saran]{SARAN}

	Kami merekomendasikan bahwa penyelidikan parameter meteorologi lainnya seperti fluks panas, radiasi gelombang panjang dan pendek, fluks air tawar, tutupan awan dan parameter lainnya harus dilanjutkan.

\end{spacing}