%\vspace{1.5pc}
\vspace{1.5pc}
%\section[State of the Art]{State of the Art}
\begin{spacing}{1.5}
	
	Bab ini menjelaskan lebih detail mengenai pustaka relevan dan tinjauan teori dalam penelitian ini. Hal ini bertujuan untuk mereview, mengupdate, mengkritik dan mensintesis literatur, melakukan meta-analisis literatur, melakukan konsepsi ulang dari topik yang direview, dan menjawab pertanyaan spesifik penelitian dari topik yang telah direview dalam literatur \shortcite{Torraco2016}. Struktur pembahasan studi relevan dan tinjauan teori selanjutnya dibagi dalam beberapa hal: pertama, akan dibahas mengenai persamaan gerak fluida dan Navier-Stokes dalam pemodelan laut, termasuk didalamnya grid C Arakawa, dan diskritisasi numerik atas persamaan Navier-Stokes serta kriteria kestabilan dari model. Terakhir, akan dibahas mengenai model iklim yang digunakan serta kedalaman lapisan campuran.
	
	
\end{spacing}
\vspace{-0.1pc}
\section[Persamaan Gerak Fluida]{Persamaan Gerak Fluida}
\begin{spacing}{1.5}
	
	Persamaan matematika yang mengatur aliran viskoelastik fluida berasal dari persamaan-persamaan hukum konservasi fisika yaitu konservasi massa, momentum dan persamaan konstitutif reologi \shortcite{Alves2021}. Penjabaran dari hukum-hukum tersebut menentukan bagaimana suatu persamaan model hidrodinamika dibuat. Salah satu persamaan fluida yang paling terkenal adalah persamaan Navier-Stokes yang terdiri dari persamaan momentum, persamaan kontinuitas, dan persamaan konservasi densitas \shortcite{Haditiar2020}. Persamaan Navier-Stokes digunakan untuk menggambarkan fluida yang mengalir dan dianggap memiliki pergerakan yang kontinu. Diketahui bahwa hasil pengamatan dari sebuah partikel fluida yang mengalir memiliki sifat-sifat fluida secara umum yaitu kecepatan, temperatur, tekanan dan densitas \shortcite{Rafiq2019,Das2018,Khan2019}. Sebuah partikel fluida diilustrasikan pada Gambar \ref{fig:cube}a, dan \ref{fig:cube}b. Komponen fluida seperti tekanan $p$, kecepatan $u$, dan densitas $\rho$ terletak pada pusat partikel yang bergantung terhadap waktu $(t)$ dan ruang $(x,y,z)$. Sehingga, komponen-komponen tersebut dapat ditulis dalam fungsi $p(x,y,z,t), u(x,y,z,t)$  dan $\rho(x,y,z,t)$. 
	
	Asumsikan bahwa partikel fluida yang diobservasi sangat kecil sehingga sifat fluida pada permukaan kubus dapat diekspresikan secara akurat dengan menggunakan dua suku pertama dari ekspansi deret Taylor, 
	\begin{equation*}
		\sum_{n=0}^{\infty}\frac{f^{n}(a)}{n!}(x-a)^n = f(a)+\frac{f'(a)}{1!}(x-a)+\dots
	\end{equation*} 
	Sebagai contoh, tekanan pada muka $W$ dan $E$, keduanya memiliki jarak $\frac{1}{2}\delta x$ dari posisi partikel di tengah sehingga diperoleh bentuk ekspresi,
	\begin{equation*}
		p-\frac{\partial p}{\partial x}\frac{1}{2}\delta x \quad \text{dan} \quad
		p+\frac{\partial p}{\partial x}\frac{1}{2}\delta x.
	\end{equation*}
	Hal yang sama dapat dilakukan untuk variabel yang lainnya.
	\begin{figure}[H]
		\centering
		\includegraphics[width=16cm]{contents/cube}
		\caption{(a) Ilustrasi partikel sebagai sifat fisis fluida, (b) Aliran massa jenis masuk dan keluar. Gambar direproduksi dari \protect\shortcite{versteeg2007introduction}}
		\label{fig:cube}
		\medspace
		\small
		Massa jenis dari partikel $\rho(x,y,z,t)$ pada gambar bagian (a) dapat diterjemahkan sebagai aliran yang masuk dan keluar. Pada gambar bagian (b), arah aliran massa jenis pada partikel pusat merupakan jumlahan dari aliran massa jenis masuk dan keluar. Dengan cara yang sama, dapat juga dilakukan untuk tekanan dan kecepatan. 
	\end{figure}
	
\end{spacing}
\vspace{-1pc}
\section[Persamaan Navier-Stokes 3 Dimensi]{Persamaan Navier-Stokes 3 Dimensi}
\begin{spacing}{1.5}
	\par Model sirkulasi laut atau \textit{Ocean General Circulation Models} (OGCM) menggunakan persamaan Navier-Stokes untuk memodelkan fenomena fisis yang terjadi di lautan. Persamaan gerak Navier-Stokes nonhidrostatik dalam model 3-D terdiri dari persamaan momentum, persamaan kontinuitas, dan persamaan konservasi densitas \shortcite{Haditiar2020}. Pada model Navier-Stokes dengan pendekatan nonhidrostatik, tekanan air laut (P) dipecah menjadi dua bagian utama, yaitu: tekanan hidrostatik (p) dan tekanan nonhidrostatik (q)
	\begin{equation}
		P = p+q.
	\end{equation}
	Tekanan p dihitung secara diagnostik dari densitas  dan percepatan gravitasi g seperti pada persamaan berikut 
	\begin{equation}
		\frac{\partial p}{\partial z} = -(\rho - \rho_o)g.
	\end{equation}
	Sedangkan tekanan q dihitung secara prognostik dalam persamaan momentum (implisit). Hal ini karena tekanan q bergantung terhadap sirkulasi arus.
	\par Persamaan momentum lengkap untuk model nonhidrostatik adalah sebagai berikut
	\begin{equation}\label{eq:momentum}
		\begin{aligned}
			\frac{\partial u}{\partial t} + \text{adv}(u)-fv &= \frac{-1}{\rho_o}\frac{\partial(p+q)}{\partial x}+\text{diff}(u) \\
			\frac{\partial v}{\partial t} + \text{adv}(v)+fu &= \frac{-1}{\rho_o}\frac{\partial(p+q)}{\partial y}+\text{diff}(v) \\
			\frac{\partial w}{\partial t} +\text{adv}(w) &= \frac{-1}{\rho_o}\frac{\partial q}{\partial z}+\text{diff}(w).
		\end{aligned}	
	\end{equation}
	\par Dengan $\text{adv}(\psi)=u\frac{\partial \psi}{\partial x}+v\frac{\partial \psi}{\partial y}+w\frac{\partial \psi}{\partial z}$ adalah persamaan adveksi dan $\text{diff}(\psi)=\frac{\partial}{\partial x}(A_{H} \frac{\partial \psi}{\partial x})+\frac{\partial}{\partial y}(A_{H} \frac{\partial \psi}{\partial y})+\frac{\partial}{\partial z}(A_{Z} \frac{\partial \psi}{\partial z})$ adalah persamaan difusi dengan $A_H$ dan $A_Z$ koefisien gesekan eddy horizontal dan vertikal. Kecepatan arus dalam sistem koordinat Cartesian 3-D didefinisikan dengan u,v, dan w. Waktu didefinisikan dengan t, parameter Coriolis dengan f, dan densitas air laut referensi dengan $\rho_o$.
	
	Konservasi volume diekspresikan oleh persamaan kontinuitas untuk fluida yang tak termampatkan,
	\begin{equation}\label{eq:kontinuitas}
		\frac{\partial u}{\partial t} + \frac{\partial v}{\partial t} + \frac{\partial w}{\partial t} = 0.
	\end{equation}
	Berdasarkan persamaan kontinuitas \ref{eq:kontinuitas}, tekanan dinamis pada lapisan permukaan dapat dihitung dengan persamaan berikut
	\begin{equation}\label{eq:tekanan}
		\frac{\partial q_s}{\partial t} = \rho_o g_i \times \left( \frac{(\partial \left(H \langle u \rangle \right)} {\partial x} + \frac{(\partial \left(H \langle v \rangle \right)} {\partial y}\right)
	\end{equation}
	dengan $q_s = \rho_o g \eta$. Disini $\rho_o$ adalah densitas air laut referensi, dan $\eta$ adalah elevasi permukaan laut, $H$ adalah total kedalaman laut, dan $<.>$ adalah operator rata-rata vertikal.
	\par Densitas air laut bergantung pada temperatur, salinitas, dan tekanan. Selanjutnya asumsikan bahwa air laut hanya bergantung linear terhadap temperatur dan salinitas, serta difusifitas \textit{eddy} untuk temperatur dan salinitas sama. Persamaan konservasi densitas diberikan oleh,
	\begin{equation}
		\frac{\partial \rho}{\partial t} + \text{adv}(\rho) = \text{diff}(\rho).
	\end{equation}
	
	Dalam aplikasinya, persamaan Navier-Stokes tidak hanya digunakan untuk memodelkan laut, tapi juga merambah ke bidang pemodelan cuaca \shortcite{Rohli2021}, aliran air dalam pipa \shortcite{Ouchiha2012} dan aliran udara di sekitar sayap pesawat \shortcite{Tulus2019}. Dalam bentuk persamaan lengkap dan simplifikasi, persamaan ini juga dapat digunakan untuk mendesain kereta api \shortcite{Croquer2020}, pesawat terbang \shortcite{Chau2021}, dan mobil \shortcite{Ambarita2018}. Terdapat juga studi tentang aliran darah \shortcite{Gill2021}, desain stasiun pembangkit listrik \shortcite{Yang2019}, dan analisis polusi udara \shortcite{Issakhov2022}. 

\section[Arakawa C grid]{Arakawa C grid}
	Diskritisasi grid di bidang horizontal dapat dibedakan menjadi grid persegi (\textit{rectiliniear}) Gambar \ref{fig:grid}a dan grid lengkung (\textit{curvlinear}) Gambar \ref{fig:grid}b, di bidang vertikal berupa grid level z (\textit{z-coordinates}) Gambar \ref{fig:grid}c dan grid level s (\textit{$\sigma$-coordinate}) Gambar \ref{fig:grid}d \shortcite{Delandmeter2019}.
	
	\begin{figure}[H]
		\centering
		\includegraphics[width=7cm]{contents/grid.jpg}
		\caption{Diskritisasi grid dalam Parcels. Di bidang horizontal: (a) grid persegi, (b) grid lengkung, di bidang vertikal: (c) grid level z, (d) grid level s \protect\shortcite{Delandmeter2019}}.
		\label{fig:grid}
	\end{figure}
	
	Dalam aplikasinya, beberapa software pemodelan laut mengimplementasikan grid bertingkat (\textit{staggered grid}) yang diperkenalkan oleh \shortciteNP{ARAKAWA1977}, yaitu grid A, B dan C. Lebih lanjut, antara grid A, dan grid C terdapat perbedaan fundamental yaitu letak penyimpanan simpul variabel (lihat Gambar \ref{fig:arakawa}), sedangkan grid B dapat dianggap sebagai peralihan dari grid A ke grid C dan perbedaan tipe model grid ini menjadi penting dikarenakan peningkatan kapasitas komputasi yang stabil di banyak pusat pemodelan iklim telah mengantarkan periode transisi untuk model laut global  \shortcite{Barham2018,Delandmeter2019}. 
	
	\begin{figure}[H]
		\centering
		\includegraphics[width=13cm]{contents/arakawa.jpg}
		\caption{Grid Arakawa: (a) Grid A dan (b) Grid C \protect\shortcite{Delandmeter2019}}
		\label{fig:arakawa}
		\medspace
		\small
		Grid A adalah satu-satunya \textit{unstaggered grid} dalam grid Arakawa dimana variabel-variabelnya (\textit{zonal velocity (u), meridional velocity (v), tracers (T)}) hanya terdapat pada titik sudut grid, berbeda dengan grid C yang berada di sisi dan tengah grid. $i$ dan $j$ adalah indeks yang merepresentasikan variabel kolom dan baris dimana variabel disimpan.
	\end{figure}
\end{spacing}
\vspace{-0.1pc}
\section[Model Iklim]{Model Iklim}
\begin{spacing}{1.5}
	Aplikasi deret waktu (\textit{time series}) banyak melibatkan data yang menunjukkan siklus musiman. Contoh yang paling umum digunakan adalah data cuaca. Dalam penelitian \shortciteNP{Haridhi2016}, model nonlinear regresi (Pers. \ref{eq:nrl}) digunakan untuk mengkarakterisasi hubungan antara SST (\textit{sea surface temperature}) dan ND (\textit{net deployment}) - penyebaran jaring nelayan pukat cincin tradisional. Untuk menvalidasi temuan ini, mereka menggunakan persamaan siklus musiman \citeA[p. 793]{crawley2012r} dan mencari korelasi antara data SST dan data meteorologi. Dilain hal, \shortciteNP{Ikhwan2022} dalam penelitiannya mengkaji tentang kedalaman lapisan campuran (MLD) di laut Andaman menggunakan data salinitas (SSS) dari model 3-D CMEMS (\textit{Copernicus Marine Environment Monitoring Service}). Model iklim digunakan untuk mengidentifikasi dan memvalidasi jumlah musim MLD dalam setahun. Persamaan nonregresi linear \shortcite{Haridhi2016} diformulasikan dalam bentuk ,
	\begin{equation}\label{eq:nrl}
		y = b_1 + b_2(\sin(b_3x+b_4))
	\end{equation}
	dengan $b_1$ adalah konstanta pergeseran vertikal, $b_2$ adalah amplitudo gelombang sinus, $b_3$ adalah frekuensi, x adalah variabel waktu, dan $b_4$ adalah fase.
	Persamaan untuk siklus musiman \shortcite[p. 793]{crawley2012r} diberikan oleh,
	\begin{equation}
		y = \alpha + \beta \sin(2\pi t)+\gamma \cos(2\pi t) + \epsilon
	\end{equation}
	dengan adalah $\alpha$ konstanta pergesaran vertikal, $\beta$ adalah amplitude dari gelombang sinus, $\gamma$ adalah amplitude dari gelombang kosinus, $t$ adalah waktu, dan $\epsilon$ adalah elemen residual yang mungkin mewakili komponen white-noise tidak beraturan dalam proses yang mendasari data.
	
	Misalkan sebuah titik bergerak dengan kecepatan konstan pada suatu lingkaran dengan jari-jari $\rho$ dan $t$ adalah waktu yang dihitung saat jari-jari terhubung dengan titik pusat pada sudut $\theta$ dibawah sumbu horizontal. Jika titik tersebut diproyeksikan pada sumbu horizontal maka jarak proyeksi dari titik pusat adalah 
	\begin{equation}\label{eq:MIK1}
		x = \rho \cos(\omega t-\theta)
	\end{equation}
	dengan $\rho$ adalah amplitudo, $\omega$ adalah kecepatan sudut atau frekuensi, dan $\theta$ adalah perpindahan fase. Gerakan proyeksi bolak-balik sepanjang sumbu horizontal digambarkan sebagai gerak harmonik sederhana.
	
	Kecepatan sudut diukur dalam radian per satuan periode, kuantitas $2\pi / \omega$ adalah periode siklus. Pergerakan fase, juga diukur dalam radian, menunjukkan sejauh mana fungsi kosinus telah berpindah oleh pergeseran sepanjang waktu. Jadi, alih-alih puncak fungsi terjadi pada waktu $t = 0$, seperti yang terjadi pada fungsi kosinus biasa, sekarang terjadi pada
	waktu $t = \theta/\omega$. Selanjutnya perhatikan bahwa $\cos(A-B) = \cos(A)\cos(B)+\sin(A)\sin(B)$, akibatnya persamaan \ref{eq:MIK1} dapat ditulis menjadi
	\begin{equation}
		\begin{aligned}
			x &= \rho \cos(\theta)\cos(\omega t) + \rho \sin(\theta)\sin(\omega t) \\
			&= \alpha\cos(\omega t) + \beta\sin(\omega t)
		\end{aligned}
	\end{equation}
	dengan 
	\begin{equation*}
		\alpha = \rho \cos(\theta), \quad 
		\beta = \rho \sin(\theta), \quad \text{dan} \quad
		\alpha^2 + \beta^2 = \rho^2
	\end{equation*}
\subsection[Siklus Teratur]{Siklus Teratur}
	Sebuah komponen siklus yang tersembunyi di bawah gerakan lain dapat diekstraksi dari urutan data dengan aplikasi langsung dari metode regresi linier. Persamaanya dapat dituliskan dalam bentuk
	\begin{equation}\label{eq:MIK2}
		y_t = \alpha c_t(\omega) + \beta s_t(\omega) + e_t; \quad t = 0, \dots, T-1
	\end{equation}
	dengan $c_t(\omega)=\cos(\omega t)$ dan $s_t(\omega)=\sin(\omega t)$. Untuk menghindari perlunya istilah intersep, nilai-nilai variabel dependen harus berupa deviasi terhadap nilai rata-rata. Dalam bentuk matriks, persamaan \ref{eq:MIK2} ditulis menjadi
	\begin{eqnarray}
		y = [c \quad s][\alpha \quad \beta]^T + e
	\end{eqnarray} 
	dengan $c = [c_0,\dots,c_N]^T, \; s = [s_0, \dots, s_N]^T$ dan $e = [e_0,\dots,e_N]^T$ adalah vektor dengan $N$ elemen. Parameter $\alpha, \beta$ dapat dicari dengan menjalankan regresi untuk nilai $\omega$ yang beragam dan  dengan memilih regresi yang memberikan nilai terendah untuk jumlah sisa kuadrat.
	
	Representasi siklus musim ideal dengan fungsi sinusoidal sederhana mungkin tidak masuk akal. Meskipun demikian, bentuk gelombang dengan sifat yang lebih rumit dapat disintesis dengan menggunakan serangkaian fungsi sinus dan kosinus yang frekuensinya merupakan kelipatan bilangan bulat dari frekuensi musiman dasar. Jika terdapat $s = 2n$ observasi per tahun, maka model umum untuk fluktuasi musiman akan terdiri dari frekuensi
	\begin{equation}
		\omega_j = \frac{2\pi j}{s}, \quad j = 0, \dots, n = \frac{s}{2},
	\end{equation}
	yang mana sama dengan ruang pada interval $[0, \pi]$. Deretan frekuensi seperti ini disebut juga sebagai skala harmonik.
	
	Fluktuasi model musiman terdiri dari kumpulan frekuensi harmonis yang berhubungan yang diformulasikan sebagai,
	\begin{equation}
		y_t = \sum_{j=0}^{n}{\alpha_j\cos(\omega_j t)+\beta_j\sin(\omega_j t)}+e_t,
	\end{equation}
	dengan $e_t$ adalah elemen residu yang dapat merepresentasikan ketidak beraturan komponen \textit{white-noise} dalam proses yang mendasari data.
	
\subsection[Siklus Tidak Teratur]{Siklus Tidak Teratur}
\end{spacing}
\vspace{-0.1pc}
\section[Kedalaman Lapisan Campuran]{Kedalaman Lapisan Campuran}
\begin{spacing}{1.5}
	Secara umum, suhu laut berada pada kisaran $-2^\circ$C sampai $30^\circ$C. Air terhangat cenderung berada pada air permukaan di daerah rendah, sedangkan air permukaan di daerah kutub jelas jauh lebih dingin. Dikarenakan pola arus permukaan, pada garis lintang yang setara, air di sisi timur cekungan laut lebih dingin daripada air di sisi barat. Meskipun air permukaan bisa sangat hangat, sebagian besar 
	
	\begin{figure}[H]
		\centering
		\includegraphics[width=15cm]{contents/mld_theory}
		\caption{(a) Profil suhu laut terbuka yang khas untuk wilayah lintang tengah, menunjukkan lapisan campuran, termoklin yang curam, dan suhu yang relatif stabil di kedalaman, (b) Profil suhu representatif untuk daerah tropis, lintang tengah, dan kutub (PW), dan (c) Di daerah beriklim sedang, lapisan campuran lebih dalam dan termoklin kurang menonjol di musim dingin dibandingkan dengan musim panas (PW) \protect\shortcite{webb2021introduction}}
		\label{fig:mld_theory}
	\end{figure}
	 
\end{spacing}
\vspace{-0.1pc}
%\section[Parameter Meteorologi]{Parameter Meteorologi}
%\begin{spacing}{1.5}
%
%\end{spacing}
