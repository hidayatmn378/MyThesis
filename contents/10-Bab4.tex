%\vspace{1.5pc}
\vspace{1.5pc}
\section[Aplikasi pada \textit{Bay of Bengal}(BoB)]{Aplikasi pada \textit{Bay of Bengal} (BoB)}
\begin{spacing}{1.5}
\vspace{-1pc}
\subsection[Sirkulasi Arus dan Angin]{Sirkulasi Arus dan Angin}
	
	Gambar 2 menunjukkan plot arus rata-rata (u dan v) di BoB di atas permukaan laut untuk bulan Februari dan Agustus 2021. Warna pada gambar mewakili elevasi permukaan air laut (dalam meter) dan vektor arah sebagai sirkulasi permukaan laut (dalam m/s). Pada bulan Februari (Gambar 2(a)), arus permukaan lebih kuat di sepanjang batas terbuka (\textit{open boundaries}) dan di beberapa area di mana pusaran (\textit{eddies}) terjadi. Dua pusaran tampak sangat menonjol pada koordinat ($82^\circ$E, $12^\circ$N) dan ($87^\circ$E, $19^\circ$N), yang ditandai dengan arah arus yang berbeda dan nilai elevasi permukaan yang kontras. Eddy pada koordinat ($82^\circ$E, $12^\circ$N) memiliki arah berlawanan jarum jam dan elevasi permukaan yang rendah, $0.3 - 0.4$ m. Sebaliknya, eddy pada koordinat ($87^\circ$E, $19^\circ$N) memiliki arah searah jarum jam dan elevasi permukaan yang tinggi, $0.6 - 0.7$ m.
	
	Pada bulan Agustus (Gambar 2(b)), arus permukaan yang diamati juga menunjukkan arus yang kuat di sepanjang batas terbuka dan di beberapa daerah di mana pusaran terjadi. Terlihat bahwa 2 pusaran yang sangat menonjol pada bulan Agustus berada pada koordinat ($83^\circ$E, $14^\circ$N) dan ($83^\circ$E, $16^\circ$N) yang juga ditandai dengan arah arus yang berbeda dan nilai elevasi permukaan yang kontras. Eddy pada koordinat ($83^\circ$E, $14^\circ$N) memiliki arah searah jarum jam dan elevasi permukaan yang tinggi, $0.6 - 0.7$ m. Di sisi lain, eddy pada koordinat ($83^\circ$E, $16^\circ$N) memiliki arah berlawanan jarum jam dan elevasi permukaan yang rendah, $0.3$ m.
	
	\begin{figure}[H]
		\centering
		\includegraphics[width=14cm]{contents/final_figure/Figure_2}
		\caption{Elevasi permukaan laut (warna dalam meter) dan sirkulasi permukaan laut (panah dalam m/s) pada (a) Februari dan (b) Agustus 2021, data dari HYCOM (https://www.hycom.org/).}
		\label{fig:arus}
	\end{figure}

	Arus pada Gambar 2 dipengaruhi oleh berbagai faktor, salah satunya adalah angin. Angin ditunjukkan pada Gambar 3 untuk pembahasan lebih lengkap. Stres atau tekanan angin diwakili oleh panah (dalam Pa), dan gradasi warna mewakili kecepatan angin (dalam m/s). Pada bulan Februari (Gambar 3(a)) angin bertiup dari arah timur laut ke arah barat daya, sedangkan pada bulan Agustus (Gambar 3(b)) angin bertiup dari arah barat daya menuju timur laut.
	
	\begin{figure}[H]
		\centering
		\includegraphics[width=14cm]{contents/final_figure/Figure_3}
		\caption{Kecepatan angin (warna dalam m/s) dan tekanan angin (panah dalam Pa) pada (a) Februari dan (b) Agustus 2021, data dari NCEP/NCAR (https://psl.noaa.gov/data/gridded/data.ncep.reanalysis.html).}
		\label{fig:angin}
	\end{figure}
	
	Secara umum kondisi angin pada bulan Februari mengakibatkan kecenderungan arus bergerak dari timur ke barat pada batas buka selatan. Mayoritas arus untuk bulan Februari lebih kuat di batas terbuka dan di pantai. Arusnya juga kencang karena letak pusaran air dekat dengan pantai.
	
	Sebaliknya, kondisi angin pada bulan Agustus mengakibatkan arus cenderung bergerak dari barat ke timur pada batas buka selatan. Sebagian besar arus untuk bulan Agustus lebih kuat di dekat pantai barat BoB karena pusaran arus yang kuat. Pada kedua bulan tersebut, kecepatan arus pada lokasi batas terbuka dan pusaran dapat mencapai 1 m/s.
	
\subsection[Profil Transek Vertikal]{Profil Transek Vertikal}
	
	Penampang temperature (\textit{cross-sectional temperature}) digunakan dalam penelitian ini untuk memberikan ide dan deskripsi mengenai lokasi gradien temperature tertinggi untuk memberikan hasil yang lebih komprehensif. Penampang temperature disajikan pada Gambar 4 untuk melihat kedalaman lapisan campuran (MLD) di bagian selatan BoB (lintang $9^\circ$N) dan utara BoB (lintang $19^\circ$N) selama 12 bulan. Estimasi kedalaman MLD dari Gambar 4 disajikan pada Tabel 1 sebagai bahan pelengkap pembahasan.
	
	Kriteria yang digunakan untuk menggambarkan MLD dalam peta pada Gambar 4 adalah nilai ambang suhu $0.1^\circ$C. Citra penampang diplot terlebih dahulu tanpa kontur dengan menggunakan kriteria ini untuk menghasilkan citra dengan warna berbeda yaitu biru, hijau, dan kuning. Gambar kontur kemudian ditambahkan untuk melihat nilai temperature. Indikator nilai ketebalan MLD berdasarkan temperature $25^\circ$C dengan warna jingga pada gambar. Gambar 4 menunjukkan penampang temperature lautan (dalam $^\circ$C) pada garis lintang $9^\circ$N dan garis lintang $19^\circ$N selama 12 bulan pada tahun 2021. Penampang dijelaskan menggunakan hubungan antara bujur atau longitude (sumbu x, dalam derajat) dan kedalaman (sumbu y, dalam meter). Nilai suhu bervariasi berdasarkan bujur dan kedalaman. Ini digambarkan oleh kontur dan bilah warna (\textit{colorbar}). Bilah warna ditetapkan dari rentang nilai $0 - 30^\circ$C, dan kedalamannya dibatasi dari $0-150$ meter agar gambar dapat dengan mudah dibaca.
	
	\begin{figure}[H]
		\centering
		\includegraphics[width=16cm]{contents/final_figure/Figure_4}
		\caption{Penampang temperature lautan pada (a) garis lintang $9^\circ$N dan (b) garis lintang $19^\circ$N pada tahun 2021 (12 bulan), diperoleh dari HYCOM (https://www.hycom.org/).}
		\label{fig:mld}
	\end{figure}
	
	Pada bagian selatan BoB, Gambar 4(a) menunjukkan adanya 4 warna yang sangat kontras, yaitu biru muda, hijau, jingga, dan kuning. Hal ini menunjukkan bahwa nilai suhu pada kedalaman 0 sampai 150 meter bervariasi antara $15^\circ$C sampai $30^\circ$C. Pada kedalaman $0-50$ m, nilai suhu berkisar antara $28^\circ$C hingga $30^\circ$C dan mendominasi di semua bulan. Ini ditandai dengan warna kuning, dan kontur pada gambar. Pada kedalaman $50-100$ m, nilai suhu berkisar antara $23^\circ$C hingga $28^\circ$C yang ditandai dengan warna hijau dan jingga. Pada kedalaman $100-150$ m, nilai suhu berkisar antara $16^\circ$C hingga $23^\circ$C yang ditandai dengan warna biru muda.
	
	Perbedaan 4 warna yang kontras (biru muda, hijau, jingga, dan kuning) juga terjadi di bagian utara BoB, Gambar 4(b). Pada kedalaman $0-50$ m, nilai suhu berkisar antara $27^\circ$C hingga $30^\circ$C yang ditandai dengan warna kuning dan kontur pada gambar. Pada kedalaman $50-100$m, nilai suhu berkisar antara $24^\circ$C hingga $27^\circ$C, kecuali beberapa bulan terakhir (September hingga Desember) yang dapat mencapai sekitar $22^\circ$C hingga $27^\circ$C. Ini ditandai dengan warna hijau dan jingga pada gambar. Terakhir, pada kedalaman $100 - 150$ m, nilai suhu berkisar antara $16^\circ$C hingga $24^\circ$C dari Januari hingga Agustus dan $17^\circ$C hingga $22^\circ$C dari September hingga Desember, yang ditandai dengan warna biru muda. Nilai MLD dalam Tabel \ref{table:MLD_thickness} diperkirakan dari Gambar 4 berdasarkan temperature $25^\circ$C dengan warna jingga sebagai indikasi MLD ketebalan.

	\begin{table}[H]
		\centering
		\caption{Estimasi ketebalan MLD (dalam meter) selama 12 bulan pada tahun 2021}
		\label{table:MLD_thickness}
		\resizebox{\columnwidth}{!}{%
			\begin{tabular}{|c|cccccccccccc|}
				\hline
				\multicolumn{1}{|c|}{\multirow{2}{*}{Domain}} & \multicolumn{12}{c|}{Months}                                                                                                                                                                                                                                                                                                                                      \\ \cline{2-13} 
				\multicolumn{1}{|c|}{}                        & \multicolumn{1}{l|}{1}        & \multicolumn{1}{l|}{2}        & \multicolumn{1}{l|}{3}        & \multicolumn{1}{l|}{4}        & \multicolumn{1}{l|}{5}       & \multicolumn{1}{l|}{6}       & \multicolumn{1}{l|}{7}       & \multicolumn{1}{l|}{8}       & \multicolumn{1}{l|}{9}       & \multicolumn{1}{l|}{10}      & \multicolumn{1}{l|}{11}       & 12      \\ \hline
				Latitude $9^\circ$N                                   & \multicolumn{1}{l|}{75-90m}  & \multicolumn{1}{l|}{70-100m} & \multicolumn{1}{l|}{60-95m}  & \multicolumn{1}{l|}{70-95m}  & \multicolumn{1}{l|}{70-85m} & \multicolumn{1}{l|}{70-85m} & \multicolumn{1}{l|}{65-90m} & \multicolumn{1}{l|}{65-85m} & \multicolumn{1}{l|}{65-80m} & \multicolumn{1}{l|}{65-80m} & \multicolumn{1}{l|}{70-100m} & 60-95m \\ \hline
				Latitude $19^\circ$N                                  & \multicolumn{1}{l|}{80-100m} & \multicolumn{1}{l|}{60-100m} & \multicolumn{1}{l|}{50-105m} & \multicolumn{1}{l|}{65-105m} & \multicolumn{1}{l|}{50-85m} & \multicolumn{1}{l|}{50-95m} & \multicolumn{1}{l|}{65-85m} & \multicolumn{1}{l|}{60-85m} & \multicolumn{1}{l|}{70-85m} & \multicolumn{1}{l|}{75-85m} & \multicolumn{1}{l|}{75-100m} & 75-85m \\ \hline
			\end{tabular}%
		}
		\raggedright
		\tiny
		
	\end{table}
		
		Berdasarkan indikator ketebalan MLD yang telah ditentukan sebelumnya, estimasi nilai ketebalan MLD dapat diperoleh pada Tabel \ref{table:MLD_thickness}. Nilai ketebalan MLD dibuat menggunakan interval ini untuk mengakomodasi nilai ketebalan yang bervariasi berdasarkan garis bujur yang berbeda. Tabel \ref{table:MLD_thickness} menunjukkan bahwa nilai MLD pada garis lintang $9^\circ$N dapat memiliki nilai ketebalan yang bervariasi mulai dari 60 m hingga 100 m. Terlihat bahwa variasi ketebalan MLD sepanjang garis bujur cukup besar pada bulan Maret dan Desember, yaitu dapat mencapai 35 m. Juga, bulan ini, MLD paling dangkal terjadi di 60 m. MLD terdalam ditunjukkan pada bulan Februari dan November, yaitu 100 m.
		
		Di sisi lain, garis lintang $19^\circ$N pada Tabel \ref{table:MLD_thickness} menunjukkan bahwa ketebalan MLD bervariasi dari 50 m hingga 105 m. Variasi ketebalan MLD sepanjang garis bujur dapat mencapai 55 m yang terjadi pada bulan Maret. Terlihat pula nilai MLD terdangkal terjadi pada bulan Maret, Mei, dan Juni yaitu 50 m, sedangkan MLD terdalam ditunjukkan pada bulan Maret dan April yaitu 105 m. Misalkan nilai variasi MLD di kedua garis lintang dirata-ratakan. Pada kasus tersebut, lintang $19^\circ$N menunjukkan variasi ketebalan MLD yang paling besar yaitu 28.3 m, dibandingkan dengan variasi ketebalan MLD pada lintang $9^\circ$N yang hanya 23.3 m. 
		
		\begin{table}[H]
			\centering
			\caption{Ketebalan MLD rata-rata (dalam meter) berdasarkan Monsun}
			\label{table:MLD_monsoon}
			\begin{tabular}{|c|cc|cc|}
				\hline
				\multirow{2}{*}{Domain} & \multicolumn{2}{c|}{Winter (Nov-Feb)} & \multicolumn{2}{c|}{Summer (Jun-Sep)} \\ \cline{2-5} 
				& \multicolumn{1}{c|}{MinX}   & MaxX  & \multicolumn{1}{c|}{MinX}   & MaxX  \\ \hline
				Latitude $9^\circ$N             & \multicolumn{1}{c|}{68.75}   & 96.25  & \multicolumn{1}{c|}{66.25}   & 85     \\ \hline
				Latitude $19^\circ$N            & \multicolumn{1}{c|}{72.5}    & 96.25  & \multicolumn{1}{c|}{61.25}   & 87.5   \\ \hline
			\end{tabular}
		\end{table}

		Ketebalan MLD rata-rata dalam Tabel \ref{table:MLD_monsoon} dihitung dari Tabel \ref{table:MLD_thickness} dengan rata-rata ketebalan minimum dan maksimum lapisan MLD berdasarkan musim. Tabel \ref{table:MLD_monsoon} menyajikan nilai rata-rata batas bawah (MinX) dan batas atas (MaxX), pada garis lintang $9^\circ$N dan $19^\circ$N pada bulan-bulan monsun musim dingin (November-Februari) dan bulan-bulan monsun musim panas (Juni-September).
		
		Rumus yang digunakan untuk mencari nilai Min dan Max adalah sebagai berikut
		\begin{equation*}
			\text{MinX}=\frac{\sum_{i=\{\{11,12,1,2\},\{6,7,8,9\}\}}\text{MinX}_i}{4}, \quad
			\text{MaxX}=\frac{\sum_{i=\{\{11,12,1,2\},\{6,7,8,9\}\}}\text{MaxX}_i}{4}
		\end{equation*}
	
		Dengan $i$ adalah jumlah bulan-bulan monsun musim dingin (Nov-Feb) atau musim panas (Jun-Sep). Tabel \ref{table:MLD_monsoon} menunjukkan bahwa ketebalan lapisan MLD lebih tebal pada musim dingin dibandingkan musim panas, baik untuk kedalaman minimum maupun maksimum. Ini berlaku untuk lintang $9^\circ$N dan lintang $19^\circ$N.
	
\subsection[Hubungan antara Parameter Meteorologi]{Hubungan antara Parameter Meteorologi}
	
		Terdapat beberapa gaya atmosfer dievaluasi untuk menerangkan perilaku MLD yang dianalisis dari model HYCOM, lima di antaranya adalah suhu udara 2m (AirT), kelembaban spesifik 2m (SHum), laju presipitasi konvektif (CPrecR), tekanan permukaan laut (SLP), dan tekanan angin (TauX dan TauY). Gambar \ref{fig:ncep} menunjukkan hasil visualisasi dari kelima gaya tersebut selama 20 tahun dari tahun 2002 hingga 2021 di bagian selatan BoB (lintang $9^\circ$N) dan bagian utara BoB (lintang $19^\circ$N). Gambar ini dihasilkan dari data \textit{reanalysis} NCEP, yang dapat diunduh di https://psl.noaa.gov/.
		
		Nilai ekstrim tahunan dari lima gaya atmosfer untuk dua domain penelitian (lintang $9^\circ$N dan lintang $19^\circ$N) ditunjukkan pada Table \ref{table:NCEP_9} dan \ref{table:NCEP_19} (Lampiran 1). Tabel ini berguna untuk menentukan puncak dan lembah dari setiap parameter meteorologi. Gambar \ref{fig:ncep} dimaksudkan untuk melihat perulangan (periodesitas) dari 5 parameter yang diteliti. Sedangkan Tabel \ref{table:NCEP_9} dan \ref{table:NCEP_19} diturunkan dari Gambar \ref{fig:ncep}. Tabel ini berfungsi untuk melihat data secara kuantitatif. Nilai ekstrim ini penting untuk melihat pada bulan apa saja nilai ekstrim tersebut terjadi, dan ini dijadikan acuan untuk melihat parameter meteorologi yang mempengaruhi terjadinya nilai ekstrim tersebut.
		
		Secara keseluruhan, tren dari ketiga variabel AirT, SHum, dan tekanan angin (TauX dan TauY) memiliki nilai minimum pada musim dingin atau Desember-Februari dan nilai maksimum pada akhir musim semi dan musim panas pada bulan Mei-Agustus. Di sisi lain, tren dua variabel yang tersisa yaitu CPrecR dan SLP memiliki nilai minimum pada akhir musim semi dan musim panas, atau Mei-Juli, dan nilai maksimum pada musim dingin, dari Desember-Februari.
		
		Tercatat bahwa, pada lintang $9^\circ$N, nilai rata-rata minimum dari lima variabel adalah $25.47, 0.015, -0.0003, 100.2, -0.15$, dan $-0.11$ untuk setiap unit. Sedangkan nilai rata-rata maksimum dari kelima variabel berturut-turut adalah $30.53, 0.022, 0, 101.61,$ $0.225$, dan $0.18$ untuk setiap unit. Apabila nilai ekstrim pada lintang $9^\circ$N dibandingkan dengan nilai ekstrim pada lintang $19^\circ$N, selisih selang minimum dan maksimum pada lintang $19^\circ$N lebih besar dibandingkan pada lintang $9^\circ$N. Akibatnya adalah antara dua domain yang diteliti, nilai ekstrim yang lebih besar terjadi pada garis lintang $19^\circ$N. Selanjutnya, pada garis lintang $19^\circ$N nilai rata-rata minimum untuk kelima variabel adalah $20.99, 0.009, -0.0004, 99.57,$ $-0.11, -0.11$ untuk setiap unit, sedangkan rata-rata nilai maksimum untuk kelima variabel adalah $31.2, 0.023, 0.102, 0.196 , 0.231$ untuk setiap unit.
		
		\begin{figure}[H]
			\centering
			\includegraphics[width=16cm]{contents/final_figure/Figure_5}
			\caption{Gaya atmosfer pada garis lintang $9^\circ$N (warna oranye) dan $19^\circ$N (warna biru) dari tahun 2002 hingga 2021, data dari NCEP/NCAR (https://psl.noaa.gov/data/gridded/data.ncep.reanalysis.html).}
			\label{fig:ncep}
		\end{figure}
		
\subsection[Model Iklim]{Model Iklim}
	
		Data selama 20 tahun (2002 - 2021) pada Gambar \ref{fig:ncep} diamati untuk menentukan model musiman di kedua domain (lintang $9^\circ$N dan lintang $19^\circ$N) dan bertujuan untuk melihat titik-titik musiman ekstrim setiap tahunnya. Hasil dari model musiman 20 tahun dipotong selama dua tahun terakhir (2020 - 2021) dan disajikan pada Gambar \ref{fig:SM}. Sebagai informasi tambahan, disajikan Tabel \ref{table:extreme} yang merangkum nilai titik ekstrim dari tahun 2002 - 2021 berdasarkan model musiman yang diperoleh.
		
		\begin{table}[H]
			\centering
			\caption{Konstanta dan koefisien prediktor y}
			\label{table:predictor}
			\begin{tabular}{|c|ccc|ccc|}
				\hline
				\multirow{2}{*}{Domain} & \multicolumn{3}{c|}{Latitude $9^\circ$N}                                                                               & \multicolumn{3}{c|}{Latitude $19^\circ$N}                                                                              \\ \cline{2-7} 
				& \multicolumn{1}{c|}{$\alpha$} & \multicolumn{1}{c|}{$\beta$} & $\gamma$ & \multicolumn{1}{c|}{$\alpha$} & \multicolumn{1}{c|}{$\beta$} & $\gamma$ \\ \hline
				AirT                    & \multicolumn{1}{c|}{28.125762}             & \multicolumn{1}{c|}{0.219930}             & -0.770494             & \multicolumn{1}{c|}{27.131780}             & \multicolumn{1}{c|}{-0.479135}            & -2.352771             \\ \hline
				SHum                    & \multicolumn{1}{c|}{1.941e-02}             & \multicolumn{1}{c|}{-2.133e-04}           & -8.554e-04            & \multicolumn{1}{c|}{1.811e-02}             & \multicolumn{1}{c|}{-1.194e-03}           & -4.624e-03            \\ \hline
				CPrecR                  & \multicolumn{1}{c|}{-6.752e-05}            & \multicolumn{1}{c|}{2.286e-05}            & 7.285e-06             & \multicolumn{1}{c|}{-5.242e-05}            & \multicolumn{1}{c|}{3.976e-05}            & 6.407e-05             \\ \hline
				SLP                     & \multicolumn{1}{c|}{1.009e+02}             & \multicolumn{1}{c|}{4.803e-02}            & 1.890e-01             & \multicolumn{1}{c|}{1.009e+02}             & \multicolumn{1}{c|}{1.277e-01}            & 6.030e-01             \\ \hline
				TauX                    & \multicolumn{1}{c|}{0.0197177}             & \multicolumn{1}{c|}{-0.0259651}           & -0.0737232            & \multicolumn{1}{c|}{0.0181436}             & \multicolumn{1}{c|}{0.0034581}            & -0.0291363            \\ \hline
				TauY                    & \multicolumn{1}{c|}{0.0189324}             & \multicolumn{1}{c|}{-0.0210637}           & -0.0609275            & \multicolumn{1}{c|}{0.0213336}             & \multicolumn{1}{c|}{0.0037017}            & -0.0511181            \\ \hline
			\end{tabular}
		\end{table}
		
		Tabel \ref{table:predictor} menunjukkan hasil analisis model musiman dari persamaan (1) untuk gaya atmosfer. Konstanta ($\alpha$) merupakan nilai konstanta yang tidak dipengaruhi oleh musim, sedangkan konstanta ($\beta$) dan ($\gamma$) merupakan variabel yang dipengaruhi oleh musim. Selanjutnya nilai konstanta pada Tabel \ref{table:predictor} dan persamaan (1) menggambarkan model prediksi selama 20 tahun dan hasilnya pada Gambar \ref{fig:SM}.
		
		Nilai rata-rata ($\alpha$) pada Tabel \ref{table:predictor} menunjukkan bahwa nilai AirT lebih besar pada lintang $9^\circ$N dibandingkan pada lintang $19^\circ$N. Hal yang sama terjadi pada variabel SHum dan TauX. Sebaliknya, variabel CPrecR dan TauY lebih besar pada lintang $19^\circ$N daripada lintang $9^\circ$N. Lebih lanjut, untuk variabel SLP nilainya sama di lokasi kedua.
		
		\begin{figure}[H]
			\centering
			\includegraphics[width=15cm]{contents/final_figure/Figure_9}
			\caption{Model musiman untuk pemaksaan atmosfer (a) pada garis lintang $9^\circ$N dan (b) pada garis lintang $19^\circ$N, dari tahun 2020 hingga 2021.}
			\label{fig:SM}
		\end{figure}
	
		Dari Tabel \ref{table:extreme} dan Gambar \ref{fig:SM}, lima gaya yang dikaji - temperature udara 2m (AirT), kelembaban spesifik 2m (SHum), laju presipitasi konvektif (CPrecR), tekanan permukaan laut (SLP), dan tekanan angin U (TauX) dan V ( TauY) - mempengaruhi domain yang diambil sebagai sampel penelitian. Dari kelima variabel tersebut rata-rata nilai ekstrim pada kedua domain terjadi pada bulan Februari dan Agustus setiap tahunnya pada model prediksi yang diperoleh. Tiga parameter AirT, SHum, dan tekanan angin (TauX dan TauY) menunjukkan tren yang sama: minimum di bulan Februari dan maksimum di bulan Agustus. Berbeda dengan tren variabel CPrecR, dan SLP, terlihat bahwa minimum terjadi pada bulan Agustus, dan maksimum terjadi pada bulan Februari. Dari kelima parameter meteorologi yang diperiksa, ekstrim minimum dan maksimum terjadi pada monsun musim dingin dan musim panas.
		
		\begin{table}[H]
			\centering
			\caption{Nilai ekstrem untuk model musiman}
			\label{table:extreme}
			\resizebox{\columnwidth}{!}{%
			\begin{tabular}{|c|c|c|c|c|c|c|l|l|}
				\hline
				Domain                       & Year(s)                       & Extreme & AirT & SHum & CPrecR & SLP & TauX & TauY \\ \hline
				\multirow{4}{*}{2002 - 2021} & \multirow{2}{*}{Latitude $9^\circ$N}  & Min     & Des  & Jan  & Agu    & Jul & Jan  & Jan  \\ \cline{3-9} 
				&                               & Max     & Jun  & Jul  & Feb    & Jan & Jul  & Jul  \\ \cline{2-9} 
				& \multirow{2}{*}{Latitude $19^\circ$N} & Min     & Jan  & Jan  & Jul    & Jul & Jan  & Jan  \\ \cline{3-9} 
				&                               & Max     & Jul  & Jul  & Jan    & Jan & Jul  & Jul  \\ \hline
			\end{tabular}%
		}
		\end{table}
		
\end{spacing}
	
	\vspace{-1pc}
\section[Aplikasi Model]{Aplikasi Model}
\begin{spacing}{1.5}
	\vspace{-1pc}
\subsection[Analisis Chl-a, SST, dan SSS di BoB]{Analisis Chl-a, SST, dan SSS di BoB}
	Gambar 2(a) menunjukkan distribusi Chl-a di BoB. Di sini kita mengambil logaritma dari nilai Chl-a karena nilainya sangat kecil sehingga variasinya menjadi kurang terlihat. Dari hasil logaritma dapat diketahui bahwa nilai Chl-a cukup homogen kecuali pada bagian pantai terutama pada bagian utara BoB. Sebagian besar area permukaan memiliki nilai -1 hingga -2 mgm-3. Di sisi lain, di pantai utara BoB, nilai Chl-a bisa mencapai 3 mgm-3.
	
	Untuk variabel SPL (lihat Gambar 2(b)), dapat diamati bahwa semakin jauh ke utara menuju pantai, semakin rendah nilai suhu yang ditunjukkan dengan penurunan dari 29oC menjadi 21oC. Hal yang sama juga terjadi pada kasus SSS (lihat Gambar 2(c)), salinitas di sebagian besar wilayah adalah 30 Psu, namun semakin dekat dengan pantai utara BoB, nilainya semakin menurun hingga mencapai <20 Psu.
	
	\begin{figure}[H]
		\centering
		\includegraphics[width=15cm]{contents/final_figure/Figure_9}
		\caption{Model musiman untuk pemaksaan atmosfer (a) pada garis lintang $9^\circ$N dan (b) pada garis lintang $19^\circ$N, dari tahun 2020 hingga 2021.}
		\label{fig:SM}
	\end{figure}
\subsubsection[Analisis Korelasi]{Analisis Korelasi}
	
	\lipsum[1]
	
\subsubsection[Uji Hipotesis dan Analisis Variansi]{Uji Hipotesis dan Analisis Variansi}
	
	\lipsum[1]
	
\subsection[Hubungan antara Eddies dan SSH di Samudera Hindia]{Hubungan antara Eddies dan SSH di Samudera Hindia}
	
	\lipsum[1]
	
\end{spacing}