\begin{spacing}{1.5}
	\pagestyle{empty}
	\begin{center}
		\vskip 1cm
		\justifying
		Teluk Benggala (\textit{Bay of Bengal} atau BoB) merupakan lautan berbentuk cekungan yang berbatasan dengan anak benua India, Asia Tenggara, dan utara samudera Hindia. Penelitian ini bertujuan untuk mengamati kedalaman lapisan campuran (\textit{Mixed Layer Depth}) berdasarkan parameter meteorologi yaitu 2m air temperature, 2m specific humidity, convective precipitation rate, sea level pressure, wind stress U, dan wind stress V di sebelah selatan BoB (latitude $9^\circ$), dan sebelah utara BoB (latitude $19^\circ$). Penelitian ini diharapkan mampu memberikan kontribusi ilmiah dan memperkaya pengetahuan tentang hubungan parameter meteorologi dengan kedalaman lapisan campuran. Hal ini karena kedalaman lapisan campuran berperan penting secara iklim fisik dalam hal menentukan interval kisaran temperatur di wilayah laut dan pesisir. Sebagai tambahan, panas yang tersimpan dalam lapisan campuran menyediakan sumber panas yang mendorong variabilitas global seperti El Ni$\tilde{n}$o. Kedalaman lapisan campuran juga berperan dalam menentukan tingkatan rata-rata cahaya yang dapat dilihat oleh organisme laut seperti fitoplankton. Selain itu, dari periodesitas model iklim yang diperoleh akan bermanfaat untuk tujuan fishing ground, mitigasi perubahan iklim dan bencana hidro-oseanografi, tata ruang dan konservasi laut, dan sumber energi terbarukan.
	\end{center}
\end{spacing}
\pagestyle{empty}