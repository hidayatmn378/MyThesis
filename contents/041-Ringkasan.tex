\begin{spacing}{1.5}
	\pagestyle{empty}
	\begin{center}
		\vskip 1cm
		\justifying
		Lautan memiliki permukaan dan volume dalam skala besar yang memungkinkan banyak fenomena terjadi di dalam dan sekitarnya. Penelitian-penelitian yang dilakukan beberapa dekade belakangan ini menunjukkan bahwa lingkungan laut sedang mengalami ancaman besar sampah plastik yang disebabkan oleh faktor ekonomi lingkungan (\textit{environmental economics}). Berdasarkan penelitian, produksi plastik dunia meningkat lebih dari 20 kali lipat antara tahun 1964 dan 2015 sebanyak 322 juta metrik ton per tahun, dan diprediksi akan berlipat ganda pada tahun 2035 dan menjadi 4 kali lipat pada tahun 2050. Penelitian ini bertujuan untuk menginvestigasi sebaran sampah mikroplastik yang berasal dari perairan Aceh dan melakukan kajian model numerik analisis laut lagrangian serta memperoleh hubungan antara kecepatan zonal dan meridional, dan gaya angin terhadap lintasan mikroplastik. Penelitian ini diharapkan mampu memberikan kontribusi ilmiah di bidang \textit{environmental science} dan mampu menjawab salah satu tantangan terkait sampah plastik dan cara penanggulangannya dengan mengetahui sebaran sampah plastik yang berasal dari wilayah sasaran penelitian. Penjabaran model numerik yang dilakukan akan menambah pengetahuan matematis serta dapat memperoleh gambaran tentang cara kerja model, dan potensi penelitian lanjutan.
	\end{center}
\end{spacing}
\pagestyle{empty}