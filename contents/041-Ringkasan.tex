\begin{spacing}{1.5}
	\pagestyle{empty}
	\begin{center}
		\vskip 1cm
		\justifying
		Samudra Hindia adalah samudra terbesar ketiga di dunia, meliputi sekitar 19.8\% dari total volume lautan dan merupakan lautan yang sangat berpengaruh bagi ekosistem di Bumi. Wilayah perairan Samudra Hindia mencakup Teluk Benggala (\textit{Bay of Bengal} (BoB)), Laut Andaman, Selat Malaka, dan perairan Aceh. Dengan cakupan wilayah yang begitu luas, Samudra Hindia merupakan penyumbang besar bagi sistem iklim dunia dan oleh karena itu sangat penting untuk dapat diprediksi. Pengembangan model kelautan bertujuan untuk menggambarkan iklim global secara akurat dan sesuai dengan hasil observasi atau pengamatan. Namun, variabilitas spasial dan temporal perlu dipahami untuk memperoleh hasil prediksi yang lebih baik. Kajian mengenai kontribusi parameter meteorologi: \textit{2m air temperature, 2m specific humidity, convective precipitation rate, sea level pressure, wind stress U}, dan \textit{wind stress V} terhadap variabilitas MLD menggunakan data output model resolusi tinggi untuk jangka panjang belum pernah dilakukan sebelumnya, oleh karena itu penelitian ini bertujuan untuk menginvestigasi MLD berdasarkan parameter meteorologi yang telah disebutkan sebelumnya. Penelitian ini diharapkan mampu memberikan kontribusi ilmiah dan memperkaya pengetahuan tentang kedalaman lapisan campuran. Hal ini karena kedalaman lapisan campuran berperan penting secara iklim fisik dalam hal menentukan interval kisaran temperatur di wilayah laut dan pesisir. Sebagai tambahan, panas yang tersimpan dalam lapisan campuran menyediakan sumber panas yang mendorong variabilitas global seperti El Ni$\tilde{n}$o. Kedalaman lapisan campuran juga berperan dalam menentukan tingkatan rata-rata cahaya yang dapat dilihat oleh organisme laut seperti fitoplankton. Selain itu, dari periodesitas model iklim yang diperoleh akan bermanfaat untuk tujuan fishing ground, mitigasi perubahan iklim dan bencana hidro-oseanografi, tata ruang dan konservasi laut, dan sumber energi terbarukan.
	\end{center}
\end{spacing}
\pagestyle{empty}