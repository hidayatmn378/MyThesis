\documentclass[12pt, a4paper, onecolumn, oneside, final]{report}
\usepackage[latin1]{inputenc}
%\usepackage[bahasa]{babel}
\usepackage{amsmath}
\usepackage{amsfonts}
\usepackage{amssymb}
\usepackage{tocloft}
\usepackage[left= 3.5cm,right=3cm,top=3cm,bottom=3cm]{geometry}
\usepackage{indentfirst}
\usepackage{amsmath,amssymb,amsfonts,amsthm}
\usepackage{array}
\usepackage{caption}
\usepackage{wrapfig}
\usepackage{graphicx}
\usepackage{varwidth}
\usepackage{float}
\usepackage{indentfirst}
\usepackage{textcomp}
\usepackage{lmodern}
\usepackage{enumerate}
\usepackage{tabularx}
\usepackage{microtype}
%\usepackage[framed]{matlab-prettifier}
\usepackage{inputenc}
\usepackage{tikz}
\usepackage{xcolor,colortbl}
\usepackage{multirow}
\usepackage[normalem]{ulem}
\useunder{\uline}{\ul}{}
\usepackage{lscape}
\usepackage{longtable}
\usepackage{times}
\usepackage{float}
\usepackage{hyperref}
\hypersetup{colorlinks=true,
	linkcolor=black,
	filecolor=black,      
	urlcolor=blue,}
%	pdfpagemode=FullScreen,}
\urlstyle{same}
\usepackage{setspace}
\usepackage{apacite}
\usepackage{enumitem}
\usepackage[labelsep=period]{caption}
%\usepackage{pbsi}
\usepackage[T1]{fontenc}
\usepackage{esint}
\usepackage{lipsum}
\usepackage{multirow}
\usepackage{ragged2e}
\usepackage[justification=centering]{caption}
%\include{contents/hype.indonesia}
\usepackage{amsmath,amssymb,amsfonts,amsthm}
\renewcommand{\chaptername}{BAB}
%\usepackage{setspace}

\setcounter{tocdepth}{1}

\definecolor{green}{rgb}{0.1,0.1,0.1}
\newcommand{\done}{\cellcolor{teal}done}  %{0.9}
\newcommand{\hcyan}[1]{{\color{teal} #1}}

\newtheorem{theorem}{Dalil}
\newtheorem{corollary}{Akibat}
\newtheorem{lemma}{Lemma}
\newcommand\at[2]{\left.#1\right|_{#2}}
\theoremstyle{definition}
\newtheorem{definition}{Definisi}{}
\newtheorem{example}{\textsf{Contoh}}

\newenvironment{bukti}[1][Bukti]{\noindent{\it\textit{#1. }}}
{\hspace{\stretch{1}}\rule{.5em}{.5em}}

\DeclareMathOperator{\mo}{mod\,}
\DeclareMathOperator{\ord}{ord}
\DeclareMathOperator{\fpb}{fpb}
\newcommand{\defi}{\overset{\mbox{\tiny{\sf def}}}{=}}
\newcommand{\znz}{\Bbb Z/n\Bbb Z}\newcommand{\zpz}{\Bbb Z/p\Bbb Z}
\numberwithin{equation}{chapter}

\setcounter{tocdepth}{3}
\setcounter{secnumdepth}{3}

\setlength\parindent{12.5mm}

\usepackage{blindtext}
\renewcommand\cftbeforetoctitleskip{-1cm}
 \renewcommand\cftbeforeloftitleskip{-1cm}
 \renewcommand\cftbeforelottitleskip{-1cm}
\renewcommand{\cftdotsep}{1}
\renewcommand{\cftchapleader}{\cftdotfill{\cftsecdotsep}}

\renewcommand{\contentsname}{DAFTAR ISI}
\renewcommand{\cfttoctitlefont}{\hfil\Large\bfseries\MakeUppercase}
\renewcommand{\cftchapfont}{\bfseries}
\renewcommand{\cftchappagefont}{\bfseries}
\renewcommand{\cftchappresnum}{BAB }
\renewcommand{\cftchapnumwidth}{3.7em}

\renewcommand{\cftlottitlefont}{\hfil\large\bfseries\MakeUppercase}
\renewcommand{\cfttabpresnum}{Tabel }
\renewcommand{\cfttabnumwidth}{6em}

\renewcommand{\cftloftitlefont}{\hfil\large\bfseries\MakeUppercase}
\renewcommand{\cftfigpresnum}{Gambar }
\renewcommand{\cftfignumwidth}{6em}
\renewcommand{\figurename}{Gambar}

\makeatletter
\def\ps@myPS{%
    \def\@oddfoot{\null\hfill\thepage}
    \def\@evenfoot{\thepage}%
    \def\@evenhead{\null\hfil\slshape\leftmark}%
    \def\@oddhead{{\slshape\rightmark}}}%
\makeatother

\makeatletter % default is "\newcommand\@chapapp{\chaptername}"
\renewcommand\@chapapp{\textls[40]{\MakeUppercase{\chaptername}}}
\makeatletter

\usepackage{titlesec}
% 1. Judul bab ditengah
\titleformat{\chapter}[display]
 {\normalfont\large\bfseries\centering}
 {\chaptertitlename\ \Roman{chapter}}{0pt}{\large}
% 2. Font section 12pt dan tambah titik section, so 1.1 menjadi 1.1.
\titlespacing{\chapter}{0pt}{50pt}{\baselineskip}
\titleformat{\section}[block] %tambah block untuk menampilkan format angka 
 {\normalfont\fontsize{12}{15}\bfseries}{\thesection.}{1em}{}
% 3. Font subsection 12pt dan tambah titik subsection
 \titleformat{\subsection}[block] %tambah block untuk menampilkan format angka 
 {\normalfont\fontsize{12}{15}\bfseries}{\thesubsection.}{1em}{} 
% 4. Hapus spasi setelah bab
\makeatletter
\def\ttl@mkchap@i#1#2#3#4#5#6#7{%
 \ttl@assign\@tempskipa#3\relax\beforetitleunit
 \vspace{\@tempskipa}%<<<<<< REMOVE THE * AFTER \vspace
 \global\@afterindenttrue
 \ifcase#5 \global\@afterindentfalse\fi
 \ttl@assign\@tempskipb#4\relax\aftertitleunit
 \ttl@topmode{\@tempskipb}{%
 \ttl@select{#6}{#1}{#2}{#7}}%
 \ttl@finmarks % Outside the box!
 \@ifundefined{ttlp@#6}{}{\ttlp@write{#6}}}
 
 \usetikzlibrary{shapes.geometric, arrows}
%Spasi sebelum dan sesudah judul sub bab
%\titlespacing*{\section}
%{0pt}{5.5ex plus 1ex minus .2ex}{4.3ex plus .2ex}
%\titlespacing*{\subsection}
%{0pt}{5.5ex plus 1ex minus .2ex}{4.3ex plus .2ex}

\newcommand{\listappendicesname}{DAFTAR LAMPIRAN}
\newlistof{appendices}{apc}{\listappendicesname}
\newcommand{\appendices}[1]{\addcontentsline{apc}{appendices}{#1}}
\renewcommand\cftbeforeapctitleskip{-1cm}
\renewcommand{\cftapctitlefont}{\hfil\large\bfseries\MakeUppercase}

\newcommand{\newappendix}[1]{\section*{#1}\appendices{#1}}

% Tambah kata ejaan yang salah di Latex 
\hyphenation{
    alphabhet A 
    alphabhet B 
    alphabhet C 
    alphabhet D 
    alphabhet E
    alphabhet F 
    alphabhet G 
    alphabhet H 
    alphabhet I
    alphabhet J
    alphabhet K 
    alphabhet L
    alphabhet M 
    alphabhet N 
    alphabhet O
    alphabhet P 
    alphabhet Q
    alphabhet R
    alphabhet S 
    alphabhet T 
    alphabhet U
    alphabhet V
    alphabhet W 
    alphabhet X
    alphabhet Y
    alphabhet Z
    special}
  \makeatletter
\renewcommand*\@pnumwidth{3em}
\makeatother

%=====================================================================
\begin{document}

\tikzstyle{rect} = [draw, rectangle, fill=white!20, text width=25em, text centered, minimum height=2em]%text width=20em/13em
\tikzstyle{elli} = [draw, ellipse, fill=white!20, minimum height=2em]
\tikzstyle{circ} = [draw, circle, fill=white!20, minimum height=2em, inner sep=10pt]
\tikzstyle{diam} = [draw, diamond, fill=white!20, text width=6em, text badly centered, inner sep=0pt]
\tikzstyle{line} = [draw, -latex']
%=====================================================================
\pagenumbering{roman}
%=====================================================================
% Halaman Judul
%\addtocontents{toc}{~\hfill{\it Halaman}\par}
\addtocontents{toc}{\begingroup\protect\setlength{\protect\cftsecindent}{-\leftmargin}}
%\addcontentsline{toc}{section}{\protect\numberline{}Judul}
\addtocontents{toc}{\endgroup}
\begin{spacing}{1}
	\begin{center}
		{\Large\textbf{Pengaruh Parameter Meteorologi dalam Variabilitas Lapisan Laut di Teluk Benggala}}\\[1.0cm]
	\end{center}
	\vspace*{0.8cm} 
	
	\begin{center}
		
		% harus dalam 16pt Times New Roman
		\large{\textbf{Proposal TESIS}}
		\\\vspace*{1.8cm}    
		\normalsize{Diajukan untuk melengkapi tugas-tugas dan \\
			Memenuhi syarat-syarat guna pelaksanaa penelitian Tesis}\\[1.5cm]
		%\vspace*{0.3 cm}    
		%% harus dalam 16pt Times New Roman
		\vspace*{1cm}  
		{\large Oleh:}\\
		\vspace*{1cm}       
		% penulis dan NIM
		\large{\textbf{\underline{MUH. NUR HIDAYAT}}}
		\\\large{\textbf{2108201010005}} 
	\end{center}\vspace*{1cm}   
	
	\begin{figure}[h]
		\centering
		\includegraphics[width=5cm]{contents/USK} %logo universitas
	\end{figure}
	\vspace*{1.5cm}   
	
	\begin{center}
		% informasi mengenai fakultas dan program studi
		\textbf{PROGRAM STUDI MAGISTER MATEMATIKA \\
			PROGRAM PASCASARJANA UNIVERSITAS SYIAH KUALA\\
			DARUSSALAM, BANDA ACEH\\
			2022}
	\end{center}
	\thispagestyle{empty}
\end{spacing}
%=====================================================================
% Halaman Pengesahan
\pagebreak
\chapter*{PENGESAHAN}
\addtocontents{toc}{\begingroup\protect\setlength{\protect\cftsecindent}{-\leftmargin}}
%\addcontentsline{toc}{section}{\protect\numberline{}Pengesahan}	
\addtocontents{toc}{\endgroup}
\setcounter{page}{2}
\vspace{2pc}

\begin{center}
	\normalsize
	\noindent
	Pengaruh Parameter Meteorologi terhadap Kedalaman \\
	Lapisan Campuran (\textit{Mixed Layer Depth}) di Perairan Aceh \\
	
	\vspace{2cm}
	Oleh \\
	Muh. Nur Hidayat\\
	NPM. 2108201010005
\end{center}

\begin{center}
	\normalsize
	\noindent
	Oleh
	\begin{tabular}{l l l}
		Judul Tesis \verb"  " &: Pengaruh Parameter Meteorologi terhadap Kedalaman \\
		& Lapisan Campuran (\textit{Mixed Layer Depth}) di Perairan Aceh  \\
		Nama Mahasiswa &: Muh. Nur Hidayat \\
		NPM &: 2108201010005 \\
		Program Studi	&: Magister Matematika \\ 
	\end{tabular} \\
\end{center}
\begin{center}
	\vspace{4cm}
	Menyetujui\\
	Komisi Pembimbing
	
	\vspace{1cm}
	
	\begin{tabular}{l l }
		Pembimbing Utama,\verb"                 " & Pembimbing Pendamping, \verb"            "\\[2.25cm]
		\underline{Prof. Dr. Ir. Syamsul Rizal} & \underline{Prof. Dr. Marwan Ramli, S.Si.,M.Si.}\\
		NIP. 196101221987031003 & NIP. 197111251999031003
	\end{tabular}
\end{center}

\begin{center}
	\vspace{0.5cm}
	Mengetahui:\\%[0.5cm]
	
	\vspace{1cm}
	
	\begin{tabular}{l }
		\verb" "Ketua Prodi Magister Matematika FMIPA\\
		\verb" "Universitas Syiah Kuala,\\[2.25cm]
		\verb" "\underline{Dr. Dra. Intan Syahrini, M.Si}\\
		\verb" "NIP. 196409081991022001
	\end{tabular}
\end{center}
\vspace{0.3cm}
\begin{center}
	
\end{center}
\thispagestyle{empty}

%\input{LPS}
%=====================================================================
% Halaman Bebas Plagiasi
%\pagebreak
%\chapter*{PERNYATAAN BEBAS PLAGIASI}
%\addtocontents{toc}{\begingroup\protect\setlength{\protect\cftsecindent}{-\leftmargin}}
%\addcontentsline{toc}{section}{\protect\numberline{}Pernyataan Bebas Plagiasi}	
%\addtocontents{toc}{\endgroup}
%\input{contents/02-Plagiasi}
%=====================================================================
% Halaman Abstrak
%\pagebreak
%\chapter*{ABSTRAK}
%\addtocontents{toc}{\begingroup\protect\setlength{\protect\cftsecindent}{-\leftmargin}}
%%\addcontentsline{toc}{section}{\protect\numberline{}Abstrak}	
%\addtocontents{toc}{\endgroup}
%\input{contents/03-Abstrak}
%=====================================================================
% Halaman Kata Pengantar
\pagebreak
\chapter*{KATA PENGANTAR}	
\addtocontents{toc}{\begingroup\protect\setlength{\protect\cftsecindent}{-\leftmargin}}
\addcontentsline{toc}{section}{\protect\numberline{}\textbf{KATA PENGANTAR}}
\addtocontents{toc}{\endgroup}
\begin{spacing}{1.5}
	\pagestyle{empty}
	
	\vskip 1cm
	\par Puji syukur kehadirat Allah SWT yang telah melimpahkan nikmat karunia-Nya sehingga proposal penelitian yang berjudul \textbf{Pengaruh Parameter Meteorologi terhadap Kedalaman Lapisan Campuran (\textit{Mixed Layer Depth}) dan Aplikasinya di Perairan Samudera Hindia} dapat terselesaikan dengan baik. Penelitian ini dilakukan untuk memenuhi salah satu syarat dalam memperoleh gelar Magister pada Program Studi Matematika, Universitas Syiah Kuala.
	\par Penyusunan penelitian tesis ini tidak dapat selesai tanpa bantuan dari tim pembimbing. Oleh karena itu, ucapan terima  kasih disampaikan kepada pihak-pihak tersebut.
	\par Penelitian tesis ini tidak luput dari segala kekurangan, baik dalam hal penulisan maupun pembahasan dari topik penelitian. Oleh sebab itu, diperlukan saran demi penyusunan penelitian yang lebih baik. Semoga penelitian dapat memberi manfaat bagi pembaca untuk melaksanakan penelitian selanjutnya.
	\vskip 1cm  
	\begin{flushright}
		Banda Aceh, 01 November 2022
		\vskip 2cm
		Penulis	
	\end{flushright}
\end{spacing}
\pagestyle{empty}
%=====================================================================
% Halaman Ringkasan
\pagebreak
\chapter*{RINGKASAN}
\addtocontents{toc}{\begingroup\protect\setlength{\protect\cftsecindent}{-\leftmargin}}
\addcontentsline{toc}{section}{\protect\numberline{}\textbf{RINGKASAN}}	
\addtocontents{toc}{\endgroup}
\begin{spacing}{1.5}
	\pagestyle{empty}
	\begin{center}
		\vskip 1cm
		\justifying
		Teluk Benggala (\textit{Bay of Bengal} atau BoB) merupakan lautan berbentuk cekungan yang berbatasan dengan anak benua India, Asia Tenggara, dan utara samudera Hindia. Penelitian ini bertujuan untuk mengamati kedalaman lapisan campuran (Mixed Layer Depth) berdasarkan parameter meteorologi yaitu 2m air temperature, 2m specific humidity, convective precipitation rate, sea level pressure, wind stress U, dan wind stress V di sebelah selatan BoB (latitude $9^\circ$), dan sebelah utara BoB (latitude $19^\circ$). Penelitian ini diharapkan mampu memberikan kontribusi ilmiah dan memperkaya pengetahuan tentang hubungan parameter meteorologi dengan kedalaman lapisan campuran. Hal ini karena kedalaman lapisan campuran berperan penting secara iklim fisik dalam hal menentukan interval kisaran temperatur di wilayah laut dan pesisir. Sebagai tambahan, panas yang tersimpan dalam lapisan campuran menyediakan sumber panas yang mendorong variabilitas global seperti El Ni$\tilde{n}$o. Kedalaman lapisan campuran juga berperan dalam menentukan tingkatan rata-rata cahaya yang dapat dilihat oleh organisme laut seperti fitoplankton. Selain itu, dari periodesitas model iklim yang diperoleh akan bermanfaat untuk tujuan fishing ground, mitigasi perubahan iklim dan bencana hidro-oseanografi, tata ruang dan konservasi laut, dan sumber energi terbarukan.
	\end{center}
\end{spacing}
\pagestyle{empty}
%=====================================================================
% Halaman Daftar Isi
\pagebreak
\addtocontents{toc}{\begingroup\protect\setlength{\protect\cftsecindent}{-\leftmargin}}
\addcontentsline{toc}{section}{\protect\numberline{}\textbf{DAFTAR ISI}}	
\addtocontents{toc}{\endgroup}{
\hypersetup{linkcolor=black}
\tableofcontents}
%=====================================================================
% Halaman Daftar Tabel
\pagebreak
\renewcommand{\listtablename}{DAFTAR TABEL}
\addtocontents{toc}{\begingroup\protect\setlength{\protect\cftsecindent}{-\leftmargin}}
\addcontentsline{toc}{section}{\protect\numberline{}\textbf{DAFTAR TABEL}}	
\addtocontents{toc}{\endgroup}
%\addtocontents{lot}{~\hfill{\it Halaman}\par}
{\hypersetup{linkcolor=black}
\listoftables}
%=====================================================================
% Halaman Daftar Gambar
\pagebreak
\renewcommand{\listfigurename}{DAFTAR GAMBAR}
%\addtocontents{lof}{~\hfill{\it Halaman}\par}
\addtocontents{toc}{\begingroup\protect\setlength{\protect\cftsecindent}{-\leftmargin}}
\addcontentsline{toc}{section}{\protect\numberline{}\textbf{DAFTAR GAMBAR}}	
\addtocontents{toc}{\endgroup}
{\hypersetup{linkcolor=black}
\listoffigures}
%=====================================================================
% Halaman Daftar Simbol
%\pagebreak
%\chapter*{DAFTAR SIMBOL}
%\addtocontents{toc}{\begingroup\protect\setlength{\protect\cftsecindent}{-\leftmargin}}
%\addcontentsline{toc}{section}{\protect\numberline{}\textbf{DAFTAR SIMBOL}}	
%\addtocontents{toc}{\endgroup}
%\vspace{1.5pc}
\vspace{1.5pc}
\begin{center}
\begin{tabular}{lp{0.75\textwidth}}

\end{tabular}
\end{center}
%=====================================================================
% Halaman Daftar Lampiran
%\pagebreak 
%{\hypersetup{linkcolor=black}
%\listofappendices}
%%\addtocontents{apc}{~\hfill{\it Halaman}\par}
%\addtocontents{toc}{\begingroup\protect\setlength{\protect\cftsecindent}{-\leftmargin}}
%\addcontentsline{toc}{section}{\protect\numberline{}\textbf{DAFTAR LAMPIRAN}}	
%\addtocontents{toc}{\endgroup}
%=====================================================================
% Halaman Persembahan
%\pagebreak
%\chapter*{PERSEMBAHAN}
%\addtocontents{toc}{\begingroup\protect\setlength{\protect\cftsecindent}{-\leftmargin}}
%\addcontentsline{toc}{section}{\protect\numberline{}PERSEMBAHAN}	
%\addtocontents{toc}{\endgroup}
%\input{contents/06-Persembahan}
%=====================================================================
\newpage
\makeatother
\newpagestyle{chapterpage}{\setfoot{}{}{\thepage}}
\assignpagestyle\chapter{chapterpage}
\setcounter{page}{1}
\pagenumbering{arabic}
\pagestyle{myPS}
\def\thechapter{\Roman{chapter}} 
\def\thesection{\arabic{chapter}.\arabic{section}}
\def\thesubsection{\arabic{chapter}.\arabic{section}.\arabic{subsection}}
\def\theequation{\arabic{chapter}.\arabic{equation}}
\def\thefigure{\arabic{chapter}.\arabic{figure}}
\def\thetable{\arabic{chapter}.\arabic{table}}
%=====================================================================% BAB I
\chapter{PENDAHULUAN}
%\vspace{1.5pc}
\vspace{1.5pc}
\section[Latar Belakang]{Latar Belakang}
\begin{spacing}{1.5}
	Samudera Hindia adalah samudera terbesar ketiga di dunia, meliputi sekitar 19.8\% dari total volume lautan \shortcite{Eakins2010} dan merupakan lautan yang sangat berpengaruh bagi ekosistem di Bumi. Cakupan wilayah dari Samudera Hindia termasuk di dalamnya Teluk Benggala (\textit{Bay of Bengal} (BoB)), Laut Andaman, Selat Malaka, dan Perairan Aceh. Dengan cakupan wilayah yang begitu luas, Samudera Hindia merupakan penyumbang besar bagi sistem iklim dunia dan oleh karena itu sangat penting untuk dapat diprediksi. Pengembangan model kelautan berusaha untuk menggambarkan iklim global dengan cukup baik disertai dengan pengamatan. Namun, variabilitas spasial dan temporal perlu dipahami untuk prediksi yang lebih baik. Pemanasan matahari dan kekuatan angin bervariasi dalam ruang dan waktu yang akan tercermin dalam variabilitas lapisan campuran laut dan suhu permukaan. Oleh karena itu, fokus utama dari tesis ini adalah peran gaya atmosfer lokal pada variabilitas lapisan campuran dan akibatnya pada suhu permukaan laut.
	
	Beberapa studi observasional dan pemodelan telah dilakukan untuk mempelajari pengaruh interaksi atmosfer-laut terhadap variabilitas suhu permukaan laut (SST), salinitas permukaan laut (SSS), klorofil-a (chl-a), kedalaman lapisan campuran (MLD) dan sirkulasi pada wilayah perairan Samudera Hindia, diantaranya adalah, \shortciteNP{Kantha2019} yang meneliti tentang pencampuran turbulen di lapisan atas BoB utara dipengaruhi oleh lapisan dangkal yang menutupi perairan asin teluk, yang dihasilkan dari arus besar air tawar dari sungai-sungai besar yang mengalir dari anak benua Asia dan dari curah hujan di atas teluk selama musim panas. Karena BoB juga berbatasan dengan laut Arab, perbedaan sering terjadi pada musim dingin, yaitu upwelling dan konveksi musim dingin, yang meningkatkan biomassa fitoplankton di Laut Arab, tetapi sangat lemah atau bahkan tidak ada di BoB. Demikian pula, masukan nutrisi melalui aliran sungai ke BoB tidak cukup untuk meningkatkan stok fitoplankton di luar perairan \shortcite{Jyothibabu2021}. BoB memiki keunikan akibat instrusi air tawar dari curah hujan yang tinggi selama musim panas sebagai hasil penetrasi insolasi matahari di kolom air \cite{Kantha2019}. \shortciteNP{Srivastava2018} mensimulasikan model tanpa gaya angin dekat permukaan, hasilnya adalah SST (\textit{Sea Surface Temperature}) wilayah tersebut sangat meningkat di semua musim, sedangkan, tanpa adanya gaya radiasi gelombang pendek yang masuk, mereka mendapatkan hasil yang benar-benar berlawanan. Ditemukan bahwa pengaruh pemaksaan fluks air tawar pada SST wilayah tersebut sangat kecil. Ditemukan juga bahwa SSS (\textit{Sea Surface Salinity}) laut Arab dan BoB menurun tanpa adanya gaya angin dekat permukaan dan radiasi gelombang pendek yang masuk, sedangkan di BoB utara meningkat tanpa adanya gaya fluks air tawar \shortcite{Srivastava2018}.
	
	Adveksi lateral yang kuat dari air salinitas rendah mengarah pada pengembangan stratifikasi laut atas yang kuat (stratifikasi salinitas), yang dapat berdampak signifikan pada evolusi SST dan SSS dengan memodifikasi pencampuran di dekat permukaan. Fluks udara-laut tidak cukup untuk mensimulasikan evolusi SST dengan benar di BoB utara, dan bahwa penghitungan adveksi air tawar diperlukan untuk mengurangi kesalahan dalam SST \shortcite{Buckley2020}. Pendinginan SST yang nyata (sekitar $2.0 - 2.5^\circ$C) dan peningkatan salinitas permukaan laut ($\sim$ 1 psu) di sisi kanan jalur topan. SST yang tinggi, TCHP (\textit{tropical cyclone heat potential}) dan kedalaman lapisan isotermal yang dalam adalah kekuatan pemicu samudera utama untuk mengintensifkan siklon Titli \shortcite{Akhter2022}. 
	
	Pengaruh radiasi panas terhadap lapisan permukaan batas BoB tergantung pada variabel biologis (Chl-a atau Klorofil-a) dan fisik (panas). Pemanasan biologis $10\; \text{Wm}^{-2}$ akan menghasilkan pemanasan tambahan $0,008^\circ \text{C jam}^{-1}$ di laut bagian atas yang menunjukkan dampak signifikan dari peningkatan konsentrasi chl-a \shortcite{Parida2022}. Konsentrasi maksimum klorofil-a di permukaan dan di bawah permukaan atau \textit{subsurface chlorophyll-a maximum} (SCM) lebih tinggi selama musim panas dan awal musim gugur dibandingkan musim lainnya, terutama di sepanjang wilayah pesisir dan bagian barat BoB. Selama musim panas dan awal musim gugur, masukan nutrisi sungai, intrusi air bergizi dari Laut Arab, dan upwelling pesisir adalah tiga pendorong dominan yang mengendalikan konsentrasi klorofil-a di permukaan dan SCM. Pengangkatan termoklin yang diinduksi oleh tegangan angin positif meningkatkan pasokan nutrisi dan dengan demikian secara signifikan meningkatkan konsentrasi klorofil-a di SCM di sepanjang sisi barat teluk selama paruh kedua tahun ini. Selama musim semi, kedalaman eufotik yang dalam memainkan peran penting dalam mengendalikan konsentrasi dan kedalaman SCM \shortcite{Chowdhury2021}.
	
	Dalam kaitannya dengan MLD, diketahui bahwa angin memiliki dampak langsung terhadap MLD dimana kekuatan angin mempengaruhi secara simultan kondisi MLD. Disisi lain, presipitasi menunjukkan dampak tidak langsung pada MLD. Curah hujan membutuhkan waktu untuk mengumpulkan efek untuk mengubah keadaan MLD. Waktu yang diperlukan untuk presipitasi adalah dua bulan sebelum terjadi perubahan MLD \shortcite{Ikhwan2022}. Pendinginan SST terutama disebabkan oleh pencampuran hangat ($32^\circ$C), tutupan segar yang terbentuk selama bulan-bulan sebelumnya dari angin sepoi-sepoi dan langit cerah, yang menyumbang sekitar setengah dari pendinginan. Fluks panas udara-laut memainkan peran sekunder, terhitung sekitar seperempat dari pendinginan. Kedalaman pencampuran didiagnosis dengan dua ukuran: kedalaman lapisan campuran tradisional dan "kedalaman pencampuran" yang didefinisikan sebagai kedalaman terdalam yang tidak stabil terhadap ketidakstabilan geser. Kedalaman pencampuran kira-kira dua kali ($\sim$ 65 m) dari kedalaman lapisan campuran ($\sim$ 35 m), yang menggambarkan pentingnya "lapisan transisi" di antara mereka. Lapisan campuran diratifikasi kembali menjadi 2 lapisan dalam sehari setelah badai berakhir dengan gelombang frekuensi mendekati inersia yang ditimbulkan oleh badai Roanu meningkatkan laju pencampuran diapiknal pada kedalaman lapisan transisi \shortcite{Kumar2019}. 
	
	MLD secara signifikan terdampar di selatan garis lintang pantai timur India (EICC) yang terpisah, area yang didominasi oleh aktivitas pusaran antisiklon. Lapisan campuran yang lebih dangkal dan stratifikasi yang ditingkatkan dengan efek \textit{relative wind} (RW) dikaitkan dengan dominasi isopiknal oleh kecepatan Ekman ke atas yang tidak normal, yang dengan sendirinya dihasilkan oleh interaksi arus permukaan antisiklonik dan angin monsun barat daya yang berlaku. Bagian barat daya BoB ini merupakan titik panas untuk pertukaran momentum antara sirkulasi permukaan dan angin monsun, sehingga merupakan area potensial untuk pengukuran lapangan terfokus untuk energetika sirkulasi laut dan interaksi udara-laut \shortcite{Seo2019}. MLD yang sebenarnya tidak hanya bergantung pada kecepatan angin, tapi salinitas juga berperan di teluk utara. Namun, ada perubahan yang dapat diabaikan dalam SST bahkan ketika MLD berubah secara signifikan karena termoklin dalam memisahkan perubahan MLD dan SST. Sebaliknya, termoklin yang lebih dangkal di teluk barat membatasi potensi MLD, yang menyebabkan perubahan SST yang lebih besar. Gelombang Rossby upwelling (atau downwelling) pada dasarnya mengkondisikan laut bagian atas dengan mengurangi (atau meningkatkan) potensi MLD. Variasi SST melemah hanya ketika termoklin semakin dalam selama peristiwa downwelling, yang terjadi kemudian di teluk barat karena gelombang Rossby merambat ke barat \shortcite{Jain2021}. 
	
	Dampak angin kencang dirasakan pada kedalaman yang lebih besar untuk suhu daripada salinitas di seluruh domain; namun, dampaknya diwujudkan dengan distribusi vertikal yang berbeda di bagian utara daripada di bagian selatan Teluk. Seperti yang diharapkan, pencampuran yang ditingkatkan yang disebabkan oleh angin yang lebih kuat menurunkan (atau meningkatkan) suhu laut bagian atas (salinitas) sebesar $0.2^\circ$C (0.3 psu), dan melemahkan stratifikasi dekat-permukaan. Selain itu, angin yang lebih kuat meningkatkan aktivitas pusaran air, memperkuat arus batas barat musim semi dan meningkatkan upwelling pantai selama musim semi dan musim panas di sepanjang pantai timur India \shortcite{Jana2018}. Berdasarkan inversi termal, rata-rata profil BoB barat laut memiliki lapisan campuran yang lebih dalam (MLD 10.30 m) dan lapisan isotermal (ILD 8.40 m) dibandingkan profil di BoB timur laut. Lapisan penghalang di BoB barat laut juga lebih tebal (2.79 m) daripada di BoB timur laut (1.05 m). Salah satu alasan yang mungkin untuk perbedaan ini adalah masuknya air tawar besar-besaran di BoB barat laut, karena air tawar mengurangi salinitas (27 PSU di BoB barat laut versus 35 PSU di BoB timur laut) dan menghasilkan MLD dan ILD yang lebih dangkal \shortcite{Masud-Ul-Alam2022}.  Stratifikasi dan lapisan depan lapisan campuran berkembang dalam skala waktu yang relatif singkat, kemungkinan sebagai respons terhadap kekuatan atmosfer yang kuat baik yang terkait dengan siklon tropis, kondisi monsun timur laut yang berkelanjutan, atau kombinasi keduanya \shortcite{Shroyer2020}. 
		
	Masuknya air tawar yang besar berkaitan erat dengan MLD yang dangkal, pembentukan lapisan penghalang yang tebal, dan sirkulasi yang kuat dan pembalikan suhu \shortcite{Dandapat2020}. 	Korelasi parsial menunjukkan bahwa fluks panas bersih (Qnet) adalah kontributor utama pendalaman MLD pada BoB utara, sedangkan tekanan angin mengontrol pendalaman atas BOB selatan. Variabilitas musiman menunjukkan pendalaman MLD selama monsun musim panas dan musim dingin dan pendangkalan selama pra dan pasca monsun di atas BoB \shortcite{Sadhukhan2021}. Perubahan yang diamati pada MLD dengan jelas membatasi rezim utara-selatan yang berbeda dengan $15^\circ$LU sebagai garis lintang pembatas. Utara dari garis lintang ini MLD tetap dangkal ($\sim$20 m) hampir sepanjang tahun tanpa menunjukkan musim yang berarti. Kurangnya musim menunjukkan bahwa air salinitas rendah, yang selalu ada di teluk utara, mengontrol stabilitas dan MLD. Penyegaran musim dingin yang diamati didorong oleh curah hujan musim dingin dan debit sungai terkait, yang didorong ke lepas pantai di bawah sirkulasi yang berlaku. Stratifikasi yang dihasilkan begitu kuat sehingga bahkan pendinginan $4^\circ$C pada suhu permukaan laut (SST) selama musim dingin tidak dapat memulai pencampuran konvektif. Sebaliknya, wilayah selatan menunjukkan variabilitas semi-tahunan yang kuat dengan MLD yang dalam selama musim panas dan musim dingin dan MLD yang dangkal selama musim semi dan musim gugur. MLD dangkal di musim semi dan musim gugur dihasilkan dari pemanasan primer dan sekunder yang terkait dengan peningkatan radiasi matahari yang masuk dan angin yang lebih ringan selama periode ini. Lapisan campuran yang dalam selama musim panas dihasilkan dari dua proses: peningkatan kekuatan angin dan intrusi air salinitas tinggi yang berasal dari Laut Arab \shortcite{Narvekar2006}.
	
	Penelitian tentang MLD di wilayah Laut Andaman dan perairan Aceh sendiri, masih jarang dilakukan. \shortciteNP{Ikhwan2022} meneliti tentang MLD di Laut Andaman dengan menggunakan data salinitas dari model CMEMS. Sinyal musiman digambarkan dengan data angin, presipitasi, temperatur, dan salinitas selama 26 tahun untuk mengidentifikasi jumlah musim MLD dalam setahun. Dari hasil penelitian diperoleh bahwa perbedaan kedalaman lapisan campuran di Laut Andaman dipengaruhi oleh angin dan presipitasi. Disisi lain, \shortciteNP{Yunita2021} mengkaji tentang MLD berdasarkan temperatur dan angin permukaan laut di perairan Aceh utara pada tahun 2017. Dengan membandingkan hasil data output model CMEMS dengan data Aqua MODIS, hasil penelitian menunjukkan bahwa kedua model relatif sama dengan variasi hingga $2^\circ$C. Sebagai tambahan, Verifikasi data SST CMEMS bulan Februari, April, Agustus dan Oktober menunjukkan hasil yang cukup baik dengan nilai korelasi r = 0,8523. Analisis yang dilakukan menunjukkan bahwa MLD terdalam terjadi pada bulan Februari, Agustus, dan Oktober. Lebih lanjut, MLD di perairan utara Aceh adalah 68-91 meter, dan perairan Sabang dan Krueng Raya 68-79 meter. 
	
	Dari beberapa penelitian yang telah disebutkan di atas, kajian mengenai kontribusi parameter meteorologi: \textit{2m air temperature, 2m specific humidity, convective precipitation rate, sea level pressure, wind stress U}, dan \textit{wind stress V} terhadap variabilitas MLD menggunakan data output model resolusi tinggi untuk jangka panjang belum pernah dilakukan sebelumnya khususnya untuk wilayah perairan Aceh, oleh karena itu penelitian ini bertujuan untuk menginvestigasi MLD berdasarkan parameter meteorologi yang telah disebutkan sebelumnya. Analisis dengan model iklim ditekankan sebagai verifikasi untuk observasi MLD yang dilakukan pada sampel stasiun wilayah peneltian. Pada akhirnya, dari hasil analisis yang dilakukan akan diperoleh hubungan antara parameter meteorologi dan MLD.
	
	\section[Rumusan Masalah]{Rumusan Masalah}
	Pada latar belakang, telah diuraikan penelitian-penelitian terkait MLD dan mengapa MLD penting untuk menggambarkan iklim global. Telah dijelaskan pula secara ringkas mengenai hal-hal apa saja yang akan dilakukan dalam penelitian ini. Fokus dari penelitian tesis ini adalah menjawab masalah utama, yaitu
	
	Bagaimana pengaruh parameter meteorologi terhadap kedalaman lapisan campuran (\textit{Mixed Layer Depth}) di Perairan Aceh?
	
	Subpertanyaan berikut akan berkontribusi pada perumusan jawaban atas masalah utama.
	\begin{itemize}
		\item Bagaimana analisis kedalaman lapisan campuran (MLD) di wilayah perairan Aceh dalam 12 bulan pada tahun 2021? 
		\item Bagaimana analisis model iklim untuk parameter-parameter meteorologi \textit{2m air temperature, 2m specific humidity, convective precipitation rate, sea level pressure, wind stress U}, dan \textit{wind stress V} selama 22 tahun, tahun 2000 - 2021?
		\item Bagaimana hubungan parameter meteorologi terhadap analisis kedalaman lapisan campuran (MLD) di wilayah perairan Aceh?
	\end{itemize}
	\section[Tujuan Penelitian]{Tujuan Penelitian}
	
	Tujuan dari penelitian tesis ini adalah mencari tahu pengaruh parameter meteorologi terhadap kedalaman lapisan campuran (\textit{Mixed Layer Depth}) di Perairan Aceh dengan cara menjawab beberapa masalah terkait,
	
	\begin{itemize}
		\item Analisis kedalaman lapisan campuran (MLD) di wilayah perairan Aceh dalam 12 bulan pada tahun 2021.
		\item Analisis model iklim untuk parameter-parameter meteorologi \textit{2m air temperature, 2m specific humidity, convective precipitation rate, sea level pressure, wind stress U}, dan \textit{wind stress V} selama 22 tahun, tahun 2000 - 2021.
		\item Hubungan parameter meteorologi terhadap analisis kedalaman lapisan campuran (MLD) di wilayah perairan Aceh.
	\end{itemize}
	
	\section[Urgensi dan Kebaruan Penelitian]{Urgensi dan Kebaruan Penelitian}

	Sejauh pengamatan kami, studi secara detail terkait 6 parameter meteorologi dan dampaknya terhadap lapisan vertikal di wilayah perairan Aceh belum pernah dilakukan sebelumnya. Oleh karena itu, dirasa penting untuk melakukan penelitian ini guna mengetahui pengaruh paramater meteorologi terhadap kedalaman lapisan campuran (MLD).

	\section[Manfaat Penelitian]{Manfaat Penelitian}
	
	Penelitian ini diharapkan mampu memberikan kontribusi ilmiah dan memperkaya pengetahuan tentang kedalaman lapisan campuran atau MLD. Hal ini karena MLD berperan penting secara iklim fisik dalam hal menentukan interval kisaran temperatur di wilayah laut dan pesisir. Sebagai tambahan, panas yang tersimpan dalam lapisan campuran menyediakan sumber panas yang mendorong variabilitas global seperti El Ni$\tilde{n}$o. MLD juga berperan dalam menentukan tingkatan rata-rata cahaya yang dapat dilihat oleh organisme laut seperti fitoplankton. Selain itu, dari periodesitas model iklim yang diperoleh akan bermanfaat untuk tujuan fishing ground, mitigasi perubahan iklim dan bencana hidro-oseanografi, tata ruang dan konservasi
	laut, dan sumber energi terbarukan. 

	\section[Sistematika Penulisan]{Sistematika Penulisan}

	Tesis ini tersusun atas 5 bab. Bab pertama menjelaskan pendahuluan tentang latar belakang mengapa penelitian ini dilakukan, background masalah yang mendasari, tujuan penelitian, manfaat penelitian, serta kebaruan dari penelitian. Bab kedua berisikan tinjauan pustaka menyangkut ulasan singkat materi penelitian. Bab ketiga membahas tentang metode penelitian yang dilakukan, data yang yang digunakan, serta diagram alir (\textit{flowchart}) dari penelitian. Bab keempat membahas hasil dan pembahasan penelitian. Terakhir, bab kelima membahas tentang kesimpulan dari penelitian.
	
\end{spacing}
%=====================================================================
% BAB II
\chapter{TINJAUAN PUSTAKA}
%\vspace{1.5pc}
\vspace{1.5pc}
%\section[State of the Art]{State of the Art}
\begin{spacing}{1.5}
	
	Bab ini menjelaskan lebih detail mengenai pustaka relevan dan tinjauan teori dalam penelitian ini. Hal ini bertujuan untuk mereview, mengupdate, mengkritik dan mensintesis literatur, melakukan meta-analisis literatur, melakukan konsepsi ulang dari topik yang direview, dan menjawab pertanyaan spesifik penelitian dari topik yang telah direview dalam literatur \shortcite{Torraco2016}. Struktur pembahasan studi relevan dan tinjauan teori selanjutnya dibagi dalam beberapa hal: pertama, akan dibahas mengenai persamaan gerak fluida dan Navier-Stokes dalam pemodelan laut, termasuk didalamnya grid C Arakawa, dan diskritisasi numerik atas persamaan Navier-Stokes serta kriteria kestabilan dari model. Terakhir, akan dibahas mengenai model iklim yang digunakan.
	
	
\end{spacing}
\vspace{-0.1pc}
\section[Persamaan Gerak Fluida]{Persamaan Gerak Fluida}
\begin{spacing}{1.5}
	
	Persamaan matematika yang mengatur aliran viskoelastik fluida berasal dari persamaan-persamaan hukum konservasi fisika yaitu konservasi massa, momentum dan persamaan konstitutif reologi \shortcite{Alves2021}. Penjabaran dari hukum-hukum tersebut menentukan bagaimana suatu persamaan model hidrodinamika dibuat. Salah satu persamaan fluida yang paling terkenal adalah persamaan Navier-Stokes yang terdiri dari persamaan momentum, persamaan kontinuitas, dan persamaan konservasi densitas \shortcite{Haditiar2020}. Persamaan Navier-Stokes digunakan untuk menggambarkan fluida yang mengalir dan dianggap memiliki pergerakan yang kontinu. Diketahui bahwa hasil pengamatan dari sebuah partikel fluida yang mengalir memiliki sifat-sifat fluida secara umum yaitu kecepatan, temperatur, tekanan dan densitas \shortcite{Rafiq2019,Das2018,Khan2019}. Sebuah partikel fluida diilustrasikan pada Gambar \ref{fig:cube}a, dan \ref{fig:cube}b. Komponen fluida seperti tekanan $p$, kecepatan $u$, dan densitas $\rho$ terletak pada pusat partikel yang bergantung terhadap waktu $(t)$ dan ruang $(x,y,z)$. Sehingga, komponen-komponen tersebut dapat ditulis dalam fungsi $p(x,y,z,t), u(x,y,z,t)$  dan $\rho(x,y,z,t)$. 
	
	Asumsikan bahwa partikel fluida yang diobservasi sangat kecil sehingga sifat fluida pada permukaan kubus dapat diekspresikan secara akurat dengan menggunakan dua suku pertama dari ekspansi deret Taylor, 
	\begin{equation*}
		\sum_{n=0}^{\infty}\frac{f^{n}(a)}{n!}(x-a)^n = f(a)+\frac{f'(a)}{1!}(x-a)+\dots
	\end{equation*} 
	Sebagai contoh, tekanan pada muka $W$ dan $E$, keduanya memiliki jarak $\frac{1}{2}\delta x$ dari posisi partikel di tengah sehingga diperoleh bentuk ekspresi,
	\begin{equation*}
		p-\frac{\partial p}{\partial x}\frac{1}{2}\delta x \quad \text{dan} \quad
		p+\frac{\partial p}{\partial x}\frac{1}{2}\delta x.
	\end{equation*}
	Hal yang sama dapat dilakukan untuk variabel yang lainnya.
	\begin{figure}[H]
		\centering
		\includegraphics[width=16cm]{contents/cube}
		\caption{(a) Ilustrasi partikel sebagai sifat fisis fluida. (b) Aliran massa jenis masuk dan keluar \protect\shortcite{versteeg2007introduction}}
		\label{fig:cube}
		\medspace
		\small
		Massa jenis dari partikel $\rho(x,y,z,t)$ pada gambar bagian (a) dapat diterjemahkan sebagai aliran yang masuk dan keluar. Pada gambar bagian (b), arah aliran massa jenis pada partikel pusat merupakan jumlahan dari aliran massa jenis masuk dan keluar. Dengan cara yang sama, dapat juga dilakukan untuk tekanan dan kecepatan. 
	\end{figure}
	
\end{spacing}
\vspace{-1pc}
\section[Persamaan Navier-Stokes 3 Dimensi]{Persamaan Navier-Stokes 3 Dimensi}
\begin{spacing}{1.5}
	\par Model sirkulasi laut atau \textit{Ocean General Circulation Models} (OGCM) menggunakan persamaan Navier-Stokes untuk memodelkan fenomena fisis yang terjadi di lautan. Persamaan gerak Navier-Stokes nonhidrostatik dalam model 3-D terdiri dari persamaan momentum, persamaan kontinuitas, dan persamaan konservasi densitas \shortcite{Haditiar2020}. Pada model Navier-Stokes dengan pendekatan nonhidrostatik, tekanan air laut (P) dipecah menjadi dua bagian utama, yaitu: tekanan hidrostatik (p) dan tekanan nonhidrostatik (q)
	\begin{equation}
		P = p+q.
	\end{equation}
	Tekanan p dihitung secara diagnostik dari densitas  dan percepatan gravitasi g seperti pada persamaan berikut 
	\begin{equation}
		\frac{\partial p}{\partial z} = -(\rho - \rho_0)g.
	\end{equation}
	Sedangkan tekanan q dihitung secara prognostik dalam persamaan momentum (implisit). Hal ini karena tekanan q bergantung terhadap sirkulasi arus.
	\par Persamaan momentum lengkap untuk model nonhidrostatik adalah sebagai berikut
	\begin{equation}
		\begin{aligned}
			\frac{\partial u}{\partial t} + \text{adv}(u)-fv &= \frac{-1}{\rho_0}\frac{\partial(p+q)}{\partial x}+\text{diff}(u) \\
			\frac{\partial v}{\partial t} + \text{adv}(v)+fu &= \frac{-1}{\rho_0}\frac{\partial(p+q)}{\partial y}+\text{diff}(v) \\
			\frac{\partial w}{\partial t} +\text{adv}(w) &= \frac{-1}{\rho_0}\frac{\partial(q)}{\partial y}+\text{diff}(w).
		\end{aligned}	
	\end{equation}
	\par Dengan $\text{adv}(\psi)=u\frac{\partial \psi}{\partial x}+v\frac{\partial \psi}{\partial y}+w\frac{\partial \psi}{\partial z}$ adalah persamaan adveksi dan $\text{diff}(\psi)=\frac{\partial}{\partial x}(A_{H} \frac{\partial \psi}{\partial x})+\frac{\partial}{\partial y}(A_{H} \frac{\partial \psi}{\partial y})+\frac{\partial}{\partial z}(A_{Z} \frac{\partial \psi}{\partial z})$ adalah persamaan difusi dengan $A_H$ dan $A_Z$ koefisien gesekan eddy horizontal dan vertikal. Kecepatan arus dalam sistem koordinat Cartesian 3-D didefinisikan dengan u,v, dan w. Waktu didefinisikan dengan t, parameter Coriolis dengan f, dan densitas air laut referensi dengan $\rho_0$.
	
	Konservasi volume diekspresikan oleh persamaan kontinuitas untuk fluida yang tak termampatkan,
	\begin{equation}
		\frac{\partial u}{\partial t} + \frac{\partial v}{\partial t} + \frac{\partial w}{\partial t} = 0.
	\end{equation}
	Berdasarkan persamaan kontinuitas (2.4), tekanan dinamis pada lapisan permukaan dapat dihitung dengan persamaan berikut
	\begin{equation}
		\frac{\partial q_s}{\partial t} = \rho_0 g_i \times \left( \frac{(\partial \left(H \langle u \rangle \right)} {\partial x} + \frac{(\partial \left(H \langle v \rangle \right)} {\partial y}\right)
	\end{equation}
	dengan $q_s = \rho_0 g \eta$. Disini $\rho_0$ adalah densitas air laut referensi, dan $\eta$ adalah elevasi permukaan laut, $H$ adalah total kedalaman laut, dan $<.>$ adalah operator rata-rata vertikal.
	\par Densitas air laut bergantung pada temperatur, salinitas, dan tekanan. Selanjutnya asumsikan bahwa air laut hanya bergantung linear terhadap temperatur dan salinitas, serta difusifitas \textit{eddy} untuk temperatur dan salinitas sama. Persamaan konservasi densitas diberikan oleh,
	\begin{equation}
		\frac{\partial \rho}{\partial t} + \text{adv}(\rho) = \text{diff}(\rho).
	\end{equation}
	
	Dalam aplikasinya, persamaan Navier-Stokes tidak hanya digunakan untuk memodelkan laut, tapi juga merambah ke bidang pemodelan cuaca \shortcite{Rohli2021}, aliran air dalam pipa \shortcite{Ouchiha2012} dan aliran udara di sekitar sayap pesawat \shortcite{Tulus2019}. Dalam bentuk persamaan lengkap dan simplifikasi, persamaan ini juga dapat digunakan untuk mendesain kereta api \shortcite{Croquer2020}, pesawat terbang \shortcite{Chau2021}, dan mobil \shortcite{Ambarita2018}. Terdapat juga studi tentang aliran darah \shortcite{Gill2021}, desain stasiun pembangkit listrik \shortcite{Yang2019}, dan analisis polusi udara \shortcite{Issakhov2022}. 

\subsection[Diskritisasi Numerik]{Diskritisasi Numerik}
\subsubsection[Arakawa C grid]{Arakawa C grid}
	Diskritisasi grid di bidang horizontal dapat dibedakan menjadi grid persegi (\textit{rectiliniear}) Gambar \ref{fig:grid}a dan grid lengkung (\textit{curvlinear}) Gambar \ref{fig:grid}b, di bidang vertikal berupa grid level z (\textit{z-coordinates}) Gambar \ref{fig:grid}c dan grid level s (\textit{$\sigma$-coordinate}) Gambar \ref{fig:grid}d \shortcite{Delandmeter2019}.
	
	\begin{figure}[H]
		\centering
		\includegraphics[width=7cm]{contents/grid.jpg}
		\caption{Diskritisasi grid dalam Parcels. Di bidang horizontal: (a) grid persegi, (b) grid lengkung, di bidang vertikal: (c) grid level z, (d) grid level s \protect\shortcite{Delandmeter2019}}.
		\label{fig:grid}
	\end{figure}
	
	Dalam aplikasinya, beberapa software pemodelan laut mengimplementasikan grid bertingkat (\textit{staggered grid}) yang diperkenalkan oleh \shortciteNP{ARAKAWA1977}, yaitu grid A, B dan C. Lebih lanjut, antara grid A, dan grid C terdapat perbedaan fundamental yaitu letak penyimpanan simpul variabel (lihat Gambar \ref{fig:arakawa}), sedangkan grid B dapat dianggap sebagai peralihan dari grid A ke grid C dan perbedaan tipe model grid ini menjadi penting dikarenakan peningkatan kapasitas komputasi yang stabil di banyak pusat pemodelan iklim telah mengantarkan periode transisi untuk model laut global  \shortcite{Barham2018,Delandmeter2019}. 
	
	\begin{figure}[H]
		\centering
		\includegraphics[width=13cm]{contents/arakawa.jpg}
		\caption{Grid Arakawa: (a) Grid A dan (b) Grid C \protect\shortcite{Delandmeter2019}}
		\label{fig:arakawa}
		\medspace
		\small
		Grid A adalah satu-satunya \textit{unstaggered grid} dalam grid Arakawa dimana variabel-variabelnya (\textit{zonal velocity (u), meridional velocity (v), tracers (T)}) hanya terdapat pada titik sudut grid, berbeda dengan grid C yang berada di sisi dan tengah grid. $i$ dan $j$ adalah indeks yang merepresentasikan variabel kolom dan baris dimana variabel disimpan.
	\end{figure}
\subsubsection[Solusi Nonhidrostatik dari Persamaan Momentum]{Solusi Nonhidrostatik dari Persamaan Momentum}
	Dari persamaan 2.4 dan dengan aturan perkalian turunan, suku adveksi untuk parameter sembarang $B$ dapat dituliskan sebagai,
	\begin{equation*}
		u\frac{\partial B}{\partial x} + v\frac{\partial B}{\partial y} + w\frac{\partial B}{\partial z} = \frac{\partial (uB)}{\partial x} + \frac{\partial (vB)}{\partial y} + \frac{\partial (wB)}{\partial z} - B\left(\frac{\partial u}{\partial x} + \frac{\partial v}{\partial y} + \frac{\partial w}{\partial z}\right).
	\end{equation*}
	
	Solusi dari tekanan hidrostatik $q$ dipecah menjadi komponen eksplisit dan implisit sehingga,
	\begin{equation*}
		q \Rightarrow q^n + \Delta q^{n+1}
	\end{equation*}
	Untuk persamaan momentum 3-D, persamaan poisson untuk $\Delta q$ yang bersesuaian adalah
	\begin{equation}\label{eq:poisson}
		\begin{aligned}
			a_e \Delta q_{i,j,k+1}^{n+1} + a_w \Delta q_{i,j,k-1}^{n+1} + a_n \Delta q_{i,j+1,k}^{n+1} + a_s \Delta q_{i,j-1,k}^{n+1} \; + \\
			a_b \Delta q_{i+1,j,k}^{n+1} + 
			a_t \Delta q_{i-1,j,k}^{n+1} -
			a_o \Delta q_{i,j,k}^{n+1} = q_{i,j,k}^{*}
		\end{aligned}
	\end{equation}
	dengan nilai-nilai koefisien,
	\begin{equation}
		\begin{aligned}
			a_e &= a_w = \frac{\Delta z}{\Delta x} \\
			a_n &= a_s = \frac{\Delta z \Delta x}{(\Delta y)^2} \\
			a_e &= a_w = \frac{\Delta x}{\Delta z} \\
			a_o &= a_e + a_w + a_n + a_s + a_b + a_t.
		\end{aligned}
	\end{equation}
	Ruas kanan dari \ref{eq:poisson} merepresentasikan divergensi nilai tebakan pertama dari kecepatan, 
	\begin{equation*}
		\begin{aligned}
		q_{i,j,k}^{*} = \frac{\rho_o}{\Delta t}\left[(u_{i,j,k}^{*}-u_{i,j,k-1}^{*})\Delta z +  (v_{i,j,k}^{*}-v_{i,j,k-1}^{*})\frac{\Delta x \Delta z}{\Delta y}+(w_{i,j,k}^{*}-w_{i,j,k-1}^{*})\Delta x\right].
		\end{aligned}
	\end{equation*}
	Selanjutnya, nilai tebakan pertama dari komponen kecepatan dalam persamaan terakhir dihitung dengan cara,
	\begin{equation}
		\begin{aligned}
			u_{i,j,k}^{*} &= \cos(\alpha)u_{i,j,k}^{n}+\sin(\alpha)v_{u}^{n} - \Delta t \; \text{adv}(u) + \Delta t F_{u}^{n}\\
			v_{i,j,k}^{*} &= \cos(\alpha)v_{i,j,k}^{n}+\sin(\alpha)u_{v}^{n} - \Delta t \; \text{adv}(v) + \Delta t F_{v}^{n}\\
			w_{i,j,k}^{*} &= w_{i,j,k}^{n} - \Delta t \; \text{adv}(w) + \Delta t F_{w}^{n}
		\end{aligned}
	\end{equation}
	dengan $\alpha = \Delta tf, \; u_v \;\text{dan}\;v_u$ adalah nilai $u$ dan $v$ yang diinterpolasi pada titik grid $v$ dan $u$. Nilai $F_{u}^{n}, F_{v}^{n},\;\text{dan}\;F_{w}^{n}$ diberikan oleh,
	\begin{equation}
		\begin{aligned}
			F_{u}^{n} &=-\frac{1}{\rho_o \Delta x}(p_{i,j,k+1}^{n}-p_{i,j,k}^{n}+q_{i,j,k+1}^{n}-q_{i,j,k}^{n})+\text{diff}(u)\\
			F_{v}^{n} &=-\frac{1}{\rho_o \Delta y}(p_{i,j+1,k}^{n}-p_{i,j,k}^{n}+q_{i,j+1,k}^{n}-q_{i,j,k}^{n})+\text{diff}(v)\\
			F_{w}^{n} &=-\frac{1}{\rho_o \Delta x}(q_{i-1,j,k}^{n}-q_{i,j,k}^{n})+\text{diff}(w)
		\end{aligned}
	\end{equation}
	dengan diff($u$), diff($v$), dan diff($w$) menunjukkan momentum difusi. Setelah iterasi \textit{Successive Over-Relaxation} (SOR) telah terkumpul untuk akurasi tekanan yang ditentukan, nilai-nilai komponen kecepatan pada tingkat waktu berikutnya $(n+1)$ diberikan oleh,
	\begin{equation}
		\begin{aligned}
			u_{i,j,k}^{n+1} &= u_{i,j,k}^{*}-\frac{\Delta t}{\rho_o \Delta x}(\Delta q_{i,j,k+1}^{r} - \Delta q_{i,j,k}^{r})\\
			v_{i,j,k}^{n+1} &= v_{i,j,k}^{*}-\frac{\Delta t}{\rho_o \Delta y}(\Delta q_{i,j+1,k}^{r} - \Delta q_{i,j,k}^{r})\\
			w_{i,j,k}^{n+1} &= w_{i,j,k}^{*}-\frac{\Delta t}{\rho_o \Delta z}(\Delta q_{i-1,j,k+1}^{r} - \Delta q_{i,j,k}^{r})\\
		\end{aligned}
	\end{equation}
\subsection[Kriteria Kestabilan]{Kriteria Kestabilan}
	Kriteria stabilitas CFL (Courant-Friedrichs-Lewy) terkait dengan adveksi dari properti yang diberikan, dirumuskan sebagai
	\begin{equation}
		\Delta t \leq min\left(\frac{\Delta x}{u},\frac{\Delta y}{v},\frac{\Delta z}{w}\right).
	\end{equation}
	Terkait dengan perambatan gelombang gravitasi permukaan, kriteria kestabilan diberikan oleh
	\begin{equation}
		\Delta t \leq \frac{min(\Delta x,\Delta y)}{\sqrt{2gh_{max}}}.
	\end{equation} 
	dengan $h_{max}$ adalah kedalaman air maksimum dari domain model.
\end{spacing}
\vspace{-0.1pc}
\section[Model Iklim]{Model Iklim}
\begin{spacing}{1.5}
	Aplikasi deret waktu (\textit{time series}) banyak melibatkan data yang menunjukkan siklus musiman. Contoh yang paling umum digunakan adalah data cuaca. Dalam penelitian \shortciteNP{Haridhi2016}, model nonlinear regresi (Pers. \ref{eq:nrl}) digunakan untuk mengkarakterisasi hubungan antara SST (\textit{sea surface temperature}) dan ND (\textit{net deployment}) - penyebaran jaring nelayan pukat cincin tradisional. Untuk menvalidasi temuan ini, mereka menggunakan persamaan siklus musiman \citeA[p. 793]{crawley2012r} dan mencari korelasi antara data SST dan data meteorologi. Dilain hal, \shortciteNP{Ikhwan2022} dalam penelitiannya mengkaji tentang kedalaman lapisan campuran (MLD) di laut Andaman menggunakan data salinitas (SSS) dari model 3-D CMEMS (\textit{Copernicus Marine Environment Monitoring Service}). Model iklim digunakan untuk mengidentifikasi dan memvalidasi jumlah musim MLD dalam setahun. Persamaan nonregresi linear \shortcite{Haridhi2016} diformulasikan dalam bentuk ,
	\begin{equation}\label{eq:nrl}
		y = b_1 + b_2(\sin(b_3x+b_4))
	\end{equation}
	dengan $b_1$ adalah konstanta pergeseran vertikal, $b_2$ adalah amplitudo gelombang sinus, $b_3$ adalah frekuensi, x adalah variabel waktu, dan $b_4$ adalah fase.
	Persamaan untuk siklus musiman \shortcite[p. 793]{crawley2012r} diberikan oleh,
	\begin{equation}
		y = \alpha + \beta \sin(2\pi t)+\gamma \cos(2\pi t) + \epsilon
	\end{equation}
	dengan adalah $\alpha$ konstanta pergesaran vertikal, $\beta$ adalah amplitude dari gelombang sinus, $\gamma$ adalah amplitude dari gelombang kosinus, $t$ adalah waktu, dan $\epsilon$ adalah elemen residual yang mungkin mewakili komponen white-noise tidak beraturan dalam proses yang mendasari data.
\end{spacing}
\vspace{-0.1pc}
\section[Kedalaman Lapisan Campuran]{Kedalaman Lapisan Campuran}
\begin{spacing}{1.5}
	Laut dan atmosfer berinteraksi dengan permukaan lautan. Gaya permukaan dari atmosfer dan matahari menentukan pola keseluruhan dari suhu permukaan laut (SST). SST yang tinggi pada daerah tropis disebabkan oleh pemanasan (sedikit lebih tinggi dari $29^\circ$C di daerah tropis yang paling hangat) dan SST yang rendah pada daerah latitude atau lintang tinggi disebabkan oleh pendinginan (sekitar $-1.8^\circ$C di daerah pembentuk es, dengan variasi musiman terutama
	terlihat jelas di lintang menengah hingga tinggi). Struktur dari potensial temperatur vertikal dapat dibagi menjadi 3 zona mayor: (1) lapisan campuran (\textit{mixed layer}), (2) lapisan termoklin, dan (3) lapisan abisal. Struktur dari lapisan ini khas untuk daerah lintang rendah dan menengah dengan SST tinggi. Lebih lanjut, 2 zona pertama berada di lapisan paling atas lautan, sedangkan zona temperature yang ketiga berada di lapisan menengah, dalam, dan bawah lautan. Di lintang tinggi di mana SST rendah, struktur ini berbeda, dan dapat memiliki lapisan campuran, suhu minimum vertikal dan maksimum di bawah permukaan laut.
	
	\begin{figure}[H]
		\centering
		\includegraphics[width=13cm]{contents/mld_theory}
		\caption{Profil suhu potensial ($^\circ$C)/kedalaman (m) tipikal untuk laut terbuka di (a) Pasifik Utara bagian barat tropis ($5^\circ$LU), (b) Pasifik Utara subtropis barat dan timur ($24^\circ$LU), dan ( c) Pasifik Utara subkutub barat ($47^\circ$LU) \protect\shortcite[p. 71]{talley2011descriptive}}
		\label{fig:mld_theory}
	\end{figure}
	Lapisan campuran (\textit{mixed layer} (ML)) adalah lapisan permukaan dengan sifat yang relatif tercampur dengan baik, khususnya di penghujung malam (siklus diurnal) dan di musim dingin (siklus musiman). Pada musim panas di lintang rendah, ML bisa jadi sangat tipis atau bahkan tidak ada sama sekali. Pada musim dingin di lintang menengah hingga tinggi, dan di daerah konveksi dalam yang terisolasi, ML dapat memiliki ketebalan hingga 2000 m. Sebagai tambahan informasi, Lapisan ini bercampur dengan angin dan kehilangan daya apung karena pendinginan atau penguapan (evaporasi) di permukaan laut. ML tidak tercampur oleh pemanasan dan presipitasi di permukaan laut dan oleh sirkulasi di dalam lapisan campuran yang memindahkan air campuran yang berdekatan dengan sifat yang berbeda satu sama lain.
	
	Lapisan termoklin adalah zona vertikal dengan penurunan temperatur yang cepat dengan kedalaman kira-kira 1000 m. Dalam lapisan abisal, antara termoklin dan bawah laut, nilai potensial temperatur menurun secara perlahan. Di lintang tinggi, suhu minimum dekat permukaan (lapisan dikothermal) sering ditemukan, sisa dari lapisan campuran musim dingin yang "tertutup" dengan air yang lebih hangat di musim lain (Gambar \ref{fig:mld_theory}c); suhu maksimum yang mendasari (lapisan mesothermal) dihasilkan dari adveksi air dari lokasi yang agak lebih hangat. Struktur suhu ini stabil karena ada stratifikasi salinitas yang kuat, dengan air tawar di lapisan permukaan. Suhu tipikal di garis lintang subtropis adalah $20^\circ$C di permukaan, $8^\circ$C pada 500 m, $5^\circ$C pada 1000 m, dan $1-2^\circ$C pada 4000 m. Semua nilai ini dan bentuk sebenarnya dari profil suhu adalah fungsi dari garis lintang, seperti yang ditunjukkan oleh tiga profil yang berbeda pada Gambar \ref{fig:mld_theory}.
	
	Lebih lanjut, di semua wilayah pemanasan musim semi dan musim panas menghasilkan lapisan hangat tipis yang menutupi lapisan campuran musim dingin. Di daerah subtropis barat serta daerah lain, sering ada dua termoklin dengan lapisan yang kurang berlapis (lebih isotermal) (termostat) di antara keduanya, semuanya di atas 1000 m (Gambar \ref{fig:mld_theory}b). Di beberapa daerah lapisan campuran lain ditemukan di bagian paling bawah ("lapisan batas bawah") dan dapat mencapai ketebalan 100 m.
\end{spacing}
\vspace{-0.1pc}
%\section[Parameter Meteorologi]{Parameter Meteorologi}
%\begin{spacing}{1.5}
%
%\end{spacing}

%=====================================================================
% BAB III
\chapter{METODOLOGI PENELITIAN}
\vspace{1.5pc}
\section[Domain Penelitian]{Domain Penelitian}
\begin{spacing}{1.5}
	Domain penelitian meliputi wilayah BoB, perairan Andaman, dan samudra Hindia dengan koordinat $5.5^\circ-24.6^\circ$ LU dan $78.2^\circ-96.7^\circ$ BT (lihat Gambar \ref{fig:domain}). Data batimetri untuk domain penelitian diperoleh dari SRTM15+ \href{https://topex.ucsd.edu/pub/archive/srtm15/V1/}{(https://topex.ucsd.edu/)} - kisi elevasi global yang diperbarui pada interval pengambilan sampel spasial 15 arc-second (ukuran piksel $\sim 500 \times 500$ m di ekuator) \shortcite{Tozer2019}. Penelitian dilakukan dengan mengkaji variabilitas lapisan vertikal berdasarkan data meteorology di dua latitude terpisah yakni, di sebelah selatan domain pada latitude $9^\circ$N dan di sebelah utara domain pada latitude $19^\circ$N. 
	\begin{figure}[H]
		\centering
		\includegraphics[width=15cm]{contents/bathymetri}
		\caption{Data batimetri domain model BoB, diturunkan dari SRTM15+}
		\label{fig:domain}
	\end{figure}
\end{spacing}
\section[Data Penelitian]{Data Penelitian}
\begin{spacing}{1.5}
\vspace{-1pc}
\subsection[Data Oseanografi]{Data Oseanografi}
	Data oseanografi yang digunakan adalah data elevasi dan arus permukaan, serta data temperature dari HYCOM (\textit{HYbrid Coordinate Ocean Model}) yang merupakan salah satu model sirkulasi laut (OGCM) yang menggunakan model numerik tiga dimensi Navier-Stokes dengan input data batimetri dari GEBCO (\textit{General Bathymetric Chart of the Oceans}), data asimilasi hidrografi laut dari NCODA (\textit{Navy Coupled Ocean Data Assimilation}) dan komponen meteorologi dari NCEP (\textit{National Centers for Environmental Prediction}) ataupun NAVGEM (\textit{The NAVy Global Environmental Model}) berupa angin, kecepatan, fluks panas, tekanan permukaan laut, presipitasi, temperature, dan kelembapan \shortcite{JosephMetzger2013}. Koordinat vertikal dalam HYCOM adalah isopiknal di lautan terbuka yang terstratifikasi dan memiliki transisi yang mulus dan dinamis serta bergantung terhadap waktu pada medan daerah pesisir yang dangkal dan pada tingkat tekanan tetap di lapisan campuran permukaan atau lautan yang tidak terstratifikasi \shortcite{chassignet2017,Park2013}. 
	\par Untuk data temperature HYCOM, data yang digunakan adalah data analisis global dengan resolusi spasial 5 menit untuk longitude dan 2.5 menit untuk latitude selama 12 bulan (Januari - Desember) tahun 2021 dan dengan ketebalan bervariasi pada bidang vertikal, yaitu 40-lapisan $(k \in [1,40])$:
	\begin{equation*}
		\begin{aligned}
			z_k = \{0.0, 2.0, 4.0, 6.0, 8.0, 10.0, 12.0, 15.0, 20.0, 25.0, 30.0, 35.0, 40.0, 45.0, 50.0, \\
			60.0, 70.0,	80.0, 90.0, 100.0, 125.0, 150.0, 200.0, 250.0, 300.0, 350.0, 400.0, 500.0, 600.0,\\
			700.0, 800.0, 900.0, 1000.0, 1250.0, 1500.0, 2000.0, 2500.0, 3000.0, 4000.0, 5000.0\} (m). \\
		\end{aligned}
	\end{equation*}
\subsection[Data Meteorologi]{Data Meteorologi}
	Data meteorologi yang digunakan adalah data reanalysis NCEP/NCAR per 6 jam \href{https://psl.noaa.gov/data/gridded/data.ncep.reanalysis.html}{(https://psl.noaa.gov/data/gridded/data.ncep.reanalysis.html)} selama 20 tahun dari tahun 2002 sampai 2021 untuk 6 parameter yaitu: 2m air temperature, 2m specific humidity, convective precipitation rate, sea level pressure, wind stress U, dan wind stress V.
\end{spacing}
\vspace{-0.5pc}
\section[Prosedur Penelitian]{Prosedur Penelitian}
\begin{spacing}{1.5}
%	\begin{figure}[H]
%		\centering
%		\includegraphics[width=15cm]{contents/flowchart.png}
%		\caption{Diagram alir }
%		\label{fig:flowchart}
%	\end{figure}
	Prosedur penelitian mengikuti diagram alir pada Gambar \ref{fig:flowchart}. Pertama-tama akan dilakukan konfigurasi modul yang ada dalam Python untuk menentukan input langkah waktu simulasi yang digunakan, penentuan lokasi dan  jumlah partikel yang keluar, serta konfigurasi kernel dalam hal ini perlakuan partikel, perilaku, dan lama waktu penelitian. Selanjutnya data-data OGCM yang diperlukan didownload dari situs CMEMS, HYCOM, dan NCEP. Data-data yang telah didownload kemudian diinterpolasi agar sesuai dengan software Parcel yang digunakan dan disimpan dalam data \textit{field}. Setelah kernel didefinisikan dan dikonfigurasi, data \textit{field} kemudian diproses menggunakan algoritma Parcel dan dilakukan perulangan untuk waktu yang terintegrasi, dijalankan bersama dengan modul tambahan secara paralel serta memperbaharui kondisi partikel-partikel, dan menghasilkan output berupa NetCDF atau NumPy array. Langkah terakhir adalah proses analisis dan visualisasi hasil.
\end{spacing}
%=====================================================================
%% Halaman Output
%\pagebreak
%\chapter*{OUTPUT PROPOSAL TESIS}
%\addtocontents{toc}{\begingroup\protect\setlength{\protect\cftsecindent}{-\leftmargin}}
%\addcontentsline{toc}{section}{\protect\numberline{}\textbf{OUTPUT PROPOSAL TESIS}}	
%\addtocontents{toc}{\endgroup}
%\vspace{1.5pc}
\begin{spacing}{1.5}
	Sebagai informasi bahwa proposal ini merupakan awal dari Penelitian Proposal PMDSU (Kesepatan dengan Prof. Dr. Syamsul Rizal dan Prof Dr. Marwan Ramli, S.Si., M.Si.) yang harus saya jalankan, oleh karena itu segala output dari penelitian ini diharapkan mampu memenuhi persyaratan publikasi PMDSU. 
	
	Output dari proposal penelitian ini direncanakan dapat terpublikasi di beberapa konferensi dan jurnal berikut:
	\begin{enumerate}
		\item Output konferensi AIC (Annual International Conference 2022), terindeks scopus.
		\begin{itemize}
			\item Waktu: Deadline submit abstrak 29 Agustus 2022, Konferensi 12-13 Oktober 2022.
			\item Judul (tentatif): \textit{Surface Circulation in Aceh Waters, Malacca Strait, and Part of South China Sea.}
			\item Domain penelitian:
			\begin{figure}[H]
				\centering
				\includegraphics[width=8cm]{contents/srtm15plus}
				\caption{Domain penelitian meliputi: Perairan Aceh, Selat Malaka, dan Bagian Laut Cina Selatan}
			\end{figure}
		\end{itemize}
		\item Output konferensi ICFAES (International Conference on Fisheries, Aquatic, and Environmental Sciences), terindeks scopus.
		\begin{itemize}
			\item Waktu: Deadline submit abstrak 30 Agustus 2022, Konferensi 26 Oktober 2022.
			\item Judul (tentatif): \textit{Monthly analysis of Chlorophyll-a, Sea Surface Temperature, and Sea Surface Salinity in the Bay of Bengal (years 2015-2021).}
			\item Domain penelitian:
			\begin{figure}[H]
				\centering
				\includegraphics[width=8cm]{contents/srtm15plus_1}
				\caption{Domain penelitian Teluk Benggala}
			\end{figure}
		\end{itemize}
		\item Output jurnal JGR (Journal of Geophysical Research)
		\begin{itemize}
			\item Status: \textit{Submitted} (2022-07-4).
			\item Judul (tentatif): \textit{Effect of Meteorological Parameters on Ocean Layer Variability in the Bay of Bengal.}
			\item Domain penelitian:
			\begin{figure}[H]
				\centering
				\includegraphics[width=8cm]{contents/srtm15plus_1}
				\caption{Domain penelitian Teluk Benggala}
			\end{figure}
		\end{itemize}
	\end{enumerate}
\end{spacing}
%=====================================================================
% BAB IV
\chapter{HASIL DAN PEMBAHASAN}
%\vspace{1.5pc}
\vspace{1.5pc}
\section[Aplikasi pada \textit{Bay of Bengal} (BoB)]{Aplikasi pada \textit{Bay of Bengal}  (BoB)}
\begin{spacing}{1.5}
\vspace{-1pc}
\subsection[Sirkulasi Arus dan Angin]{Sirkulasi Arus dan Angin}
	
	Gambar \ref{fig:arus} menunjukkan plot arus rata-rata (u dan v) di BoB di atas permukaan laut untuk bulan Januari dan Juli 2021. Warna pada gambar mewakili elevasi permukaan air laut (dalam meter) dan vektor arah sebagai sirkulasi permukaan laut (dalam m/s). Pada bulan Januari (Gambar \ref{fig:arus}(a)), arus permukaan lebih kuat di sepanjang batas terbuka (\textit{open boundaries}) dan di beberapa area di mana pusaran (\textit{eddies}) terjadi. Tiga pusaran tampak sangat menonjol pada koordinat ($82^\circ$E, $12^\circ$N), ($87^\circ$E, $19^\circ$N), dan ($95^\circ$E, $8^\circ$N) yang ditandai dengan arah arus yang berbeda dan nilai elevasi permukaan yang kontras. \textit{Eddy} pada koordinat ($82^\circ$E, $12^\circ$N) memiliki arah berlawanan jarum jam dan elevasi permukaan yang rendah, $0.3 - 0.4$ m. Sebaliknya, \textit{eddy} pada koordinat ($87^\circ$E, $19^\circ$N) memiliki arah searah jarum jam dan elevasi permukaan yang tinggi, $0.6 - 0.7$ m. Arah arus searah jarum jam juga terjadi dengan \textit{eddy} pada koordinat ($95^\circ$E, $8^\circ$N) dan dengan elevasi permukaan $0.55 - 0.6$ m
	
	Pada bulan Juli (Gambar \ref{fig:arus}(b)), arus permukaan yang diamati juga menunjukkan arus yang kuat di sepanjang batas terbuka dan di beberapa daerah di mana pusaran terjadi. Terlihat bahwa 2 pusaran yang sangat menonjol pada bulan Juli berada pada koordinat ($83^\circ$E, $14^\circ$N) dan ($83^\circ$E, $16^\circ$N) yang juga ditandai dengan arah arus yang berbeda dan nilai elevasi permukaan yang kontras. Eddy pada koordinat ($83^\circ$E, $14^\circ$N) memiliki arah searah jarum jam dan elevasi permukaan yang tinggi, $0.6 - 0.7$ m. Di sisi lain, eddy pada koordinat ($83^\circ$E, $16^\circ$N) memiliki arah berlawanan jarum jam dan elevasi permukaan yang rendah, $0.3$ m.
	
	\begin{figure}[H]
		\centering
		\includegraphics[width=14.5cm]{contents/final_figure/Figure_2}
		\caption{Elevasi permukaan laut (warna dalam meter) dan sirkulasi permukaan laut (panah dalam m/s) pada (a) Januari dan (b) Juli 2021, data dari HYCOM.}
		\label{fig:arus}
	\end{figure}

	Arus pada Gambar \ref{fig:arus} dipengaruhi oleh berbagai faktor, salah satunya adalah angin. Angin ditunjukkan pada Gambar \ref{fig:angin} untuk pembahasan lebih lengkap. Stres atau tekanan angin diwakili oleh panah (dalam Pa), dan gradasi warna mewakili kecepatan angin (dalam m/s). Pada bulan Januari (Gambar \ref{fig:angin}(a)) angin bertiup dari arah timur laut ke arah barat daya, sedangkan pada bulan Juli (Gambar \ref{fig:angin}(b)) angin bertiup dari arah barat daya menuju timur laut.
	
	\begin{figure}[H]
		\centering
		\includegraphics[width=14cm]{contents/final_figure/Figure_3}
		\caption{Kecepatan angin (warna dalam m/s) dan tekanan angin (panah dalam Pa) pada (a) Januari dan (b) Juli 2021, data dari NCEP/NCAR.}
		\label{fig:angin}
	\end{figure}
	
	Secara umum kondisi angin pada bulan Januari mengakibatkan kecenderungan arus bergerak dari timur ke barat pada batas buka selatan. Mayoritas arus untuk bulan Januari lebih kuat di batas terbuka dan di pantai. Arus yang juga kencang terpantau karena letak pusaran air dekat dengan pantai.
	
	Sebaliknya, kondisi angin pada bulan Juli mengakibatkan arus cenderung bergerak dari barat ke timur pada batas buka selatan. Sebagian besar arus untuk bulan Juli lebih kuat di dekat pantai barat BoB karena pusaran arus yang kuat. Pada kedua bulan tersebut, kecepatan arus pada lokasi batas terbuka dan pusaran dapat mencapai 1 m/s.
	
\subsection[Profil Transek Vertikal]{Profil Transek Vertikal}
	
	Penampang temperature (\textit{cross-sectional temperature}) digunakan dalam penelitian ini untuk memberikan ide dan deskripsi mengenai lokasi gradien temperature tertinggi untuk memberikan hasil yang lebih komprehensif. Penampang temperature disajikan pada Gambar \ref{fig:mld} untuk melihat kedalaman lapisan campuran (MLD) di bagian selatan BoB (lintang $9^\circ$N) dan utara BoB (lintang $19^\circ$N) selama 12 bulan. Estimasi kedalaman MLD dari Gambar \ref{fig:mld} disajikan pada Tabel \ref{table:MLD_thickness} sebagai bahan pelengkap pembahasan.
	
	Kriteria yang digunakan untuk menggambarkan MLD dalam peta pada Gambar \ref{fig:mld} adalah nilai ambang suhu (\textit{threshold}) $0.1^\circ$C. Citra penampang diplot terlebih dahulu tanpa kontur dengan menggunakan kriteria ini untuk menghasilkan citra dengan warna berbeda yaitu merah, putih, dan biru. Gambar kontur kemudian ditambahkan untuk melihat nilai temperature. Indikator nilai ketebalan MLD berdasarkan temperature $25^\circ$C dengan warna merah muda pada gambar. Gambar \ref{fig:mld} menunjukkan penampang temperature lautan (dalam $^\circ$C) pada garis lintang $9^\circ$N dan garis lintang $19^\circ$N selama 12 bulan pada tahun 2021. Penampang dijelaskan menggunakan hubungan antara bujur atau longitude (sumbu x, dalam derajat) dan kedalaman (sumbu y, dalam meter). Nilai suhu bervariasi berdasarkan bujur dan kedalaman. Ini digambarkan oleh kontur dan bilah warna (\textit{colorbar}). Bilah warna ditetapkan dari rentang nilai $0 - 30^\circ$C, dan kedalamannya dibatasi dari $0-150$ meter agar gambar dapat dengan mudah dibaca.
	
	\begin{figure}[H]
		\centering
		\includegraphics[width=16cm]{contents/final_figure/Figure_4}
		\caption{Penampang temperature lautan pada (a) garis lintang $9^\circ$N dan (b) garis lintang $19^\circ$N pada tahun 2021 (12 bulan), data dari HYCOM.}
		\label{fig:mld}
	\end{figure}
	
	Pada bagian selatan BoB, Gambar \ref{fig:mld}(a) menunjukkan adanya 3 warna yang sangat kontras, yaitu merah, merah muda, dan putih. Hal ini menunjukkan bahwa nilai suhu pada kedalaman 0 sampai 150 meter bervariasi antara $15^\circ$C sampai $30^\circ$C. Pada kedalaman $0-50$ m, nilai suhu berkisar antara $28^\circ$C hingga $30^\circ$C dan mendominasi di semua bulan. Ini ditandai dengan warna merah, dan kontur pada gambar. Pada kedalaman $50-100$ m, nilai suhu berkisar antara $23^\circ$C hingga $28^\circ$C yang ditandai dengan warna merah muda. Pada kedalaman $100-150$ m, nilai suhu berkisar antara $16^\circ$C hingga $23^\circ$C yang ditandai dengan warna putih.
	
	Perbedaan 3 warna yang kontras (merah, merah muda, dan putih) juga terjadi di bagian utara BoB, Gambar \ref{fig:mld}(b). Pada kedalaman $0-50$ m, nilai suhu berkisar antara $27^\circ$C hingga $30^\circ$C yang ditandai dengan warna merah dan kontur pada gambar. Pada kedalaman $50-100$m, nilai suhu berkisar antara $24^\circ$C hingga $27^\circ$C, kecuali beberapa bulan terakhir (September hingga Desember) yang dapat mencapai sekitar $22^\circ$C hingga $27^\circ$C. Ini ditandai dengan merah muda pada gambar. Terakhir, pada kedalaman $100 - 150$ m, nilai suhu berkisar antara $16^\circ$C hingga $24^\circ$C dari Januari hingga Agustus dan $17^\circ$C hingga $22^\circ$C dari September hingga Desember, yang ditandai dengan warna putih. Nilai MLD dalam Tabel \ref{table:MLD_thickness} diperkirakan dari Gambar 4 berdasarkan temperature $25^\circ$C dengan warna merah muda sebagai indikasi MLD ketebalan.

	\begin{table}[H]
		\centering
		\caption{Estimasi ketebalan MLD (dalam meter) selama 12 bulan pada tahun 2021}
		\label{table:MLD_thickness}
		\resizebox{\columnwidth}{!}{%
			\begin{tabular}{|c|cccccccccccc|}
				\hline
				\multicolumn{1}{|c|}{\multirow{2}{*}{Domain}} & \multicolumn{12}{c|}{Months}                                                                                                                                                                                                                                                                                                                                      \\ \cline{2-13} 
				\multicolumn{1}{|c|}{}                        & \multicolumn{1}{l|}{1}        & \multicolumn{1}{l|}{2}        & \multicolumn{1}{l|}{3}        & \multicolumn{1}{l|}{4}        & \multicolumn{1}{l|}{5}       & \multicolumn{1}{l|}{6}       & \multicolumn{1}{l|}{7}       & \multicolumn{1}{l|}{8}       & \multicolumn{1}{l|}{9}       & \multicolumn{1}{l|}{10}      & \multicolumn{1}{l|}{11}       & 12      \\ \hline
				Latitude $9^\circ$N                                   & \multicolumn{1}{l|}{75-90m}  & \multicolumn{1}{l|}{70-100m} & \multicolumn{1}{l|}{60-95m}  & \multicolumn{1}{l|}{70-95m}  & \multicolumn{1}{l|}{70-85m} & \multicolumn{1}{l|}{70-85m} & \multicolumn{1}{l|}{65-90m} & \multicolumn{1}{l|}{65-85m} & \multicolumn{1}{l|}{65-80m} & \multicolumn{1}{l|}{65-80m} & \multicolumn{1}{l|}{70-100m} & 60-95m \\ \hline
				Latitude $19^\circ$N                                  & \multicolumn{1}{l|}{80-100m} & \multicolumn{1}{l|}{60-100m} & \multicolumn{1}{l|}{50-105m} & \multicolumn{1}{l|}{65-105m} & \multicolumn{1}{l|}{50-85m} & \multicolumn{1}{l|}{50-95m} & \multicolumn{1}{l|}{65-85m} & \multicolumn{1}{l|}{60-85m} & \multicolumn{1}{l|}{70-85m} & \multicolumn{1}{l|}{75-85m} & \multicolumn{1}{l|}{75-100m} & 75-85m \\ \hline
			\end{tabular}%
		}
		\raggedright
		\tiny
		
	\end{table}
		
		Berdasarkan indikator ketebalan MLD yang telah ditentukan sebelumnya, estimasi nilai ketebalan MLD dapat diperoleh pada Tabel \ref{table:MLD_thickness}. Nilai ketebalan MLD dibuat menggunakan interval ini untuk mengakomodasi nilai ketebalan yang bervariasi berdasarkan garis bujur yang berbeda. Tabel \ref{table:MLD_thickness} menunjukkan bahwa nilai MLD pada garis lintang $9^\circ$N dapat memiliki nilai ketebalan yang bervariasi mulai dari 60 m hingga 100 m. Terlihat bahwa variasi ketebalan MLD sepanjang garis bujur cukup besar pada bulan Maret dan Desember, yaitu dapat mencapai 35 m. Juga, bulan ini, MLD paling dangkal terjadi di 60 m. MLD terdalam ditunjukkan pada bulan Februari dan November, yaitu 100 m.
		
		Di sisi lain, garis lintang $19^\circ$N pada Tabel \ref{table:MLD_thickness} menunjukkan bahwa ketebalan MLD bervariasi dari 50 m hingga 105 m. Variasi ketebalan MLD sepanjang garis bujur dapat mencapai 55 m yang terjadi pada bulan Maret. Terlihat pula nilai MLD terdangkal terjadi pada bulan Maret, Mei, dan Juni yaitu 50 m, sedangkan MLD terdalam ditunjukkan pada bulan Maret dan April yaitu 105 m. Misalkan nilai variasi MLD di kedua garis lintang dirata-ratakan. Pada kasus tersebut, lintang $19^\circ$N menunjukkan variasi ketebalan MLD yang paling besar yaitu 28.3 m, dibandingkan dengan variasi ketebalan MLD pada lintang $9^\circ$N yang hanya 23.3 m. 
		
		\begin{table}[H]
			\centering
			\caption{Ketebalan MLD rata-rata (dalam meter) berdasarkan Monsun}
			\label{table:MLD_monsoon}
			\begin{tabular}{|c|cc|cc|}
				\hline
				\multirow{2}{*}{Domain} & \multicolumn{2}{c|}{Winter (Nov-Feb)} & \multicolumn{2}{c|}{Summer (Jun-Sep)} \\ \cline{2-5} 
				& \multicolumn{1}{c|}{MinX}   & MaxX  & \multicolumn{1}{c|}{MinX}   & MaxX  \\ \hline
				Latitude $9^\circ$N             & \multicolumn{1}{c|}{68.75}   & 96.25  & \multicolumn{1}{c|}{66.25}   & 85     \\ \hline
				Latitude $19^\circ$N            & \multicolumn{1}{c|}{72.5}    & 96.25  & \multicolumn{1}{c|}{61.25}   & 87.5   \\ \hline
			\end{tabular}
		\end{table}

		Ketebalan MLD rata-rata dalam Tabel \ref{table:MLD_monsoon} dihitung dari Tabel \ref{table:MLD_thickness} dengan rata-rata ketebalan minimum dan maksimum lapisan MLD berdasarkan musim. Tabel \ref{table:MLD_monsoon} menyajikan nilai rata-rata batas bawah (MinX) dan batas atas (MaxX), pada garis lintang $9^\circ$N dan $19^\circ$N pada bulan-bulan monsun musim dingin (November-Februari) dan bulan-bulan monsun musim panas (Juni-September).
		
		Rumus yang digunakan untuk mencari nilai Min dan Max adalah sebagai berikut
		\begin{equation*}
			\text{MinX}=\frac{\sum_{i=\{\{11,12,1,2\},\{6,7,8,9\}\}}\text{MinX}_i}{4}, \quad
			\text{MaxX}=\frac{\sum_{i=\{\{11,12,1,2\},\{6,7,8,9\}\}}\text{MaxX}_i}{4}
		\end{equation*}
	
		Dengan $i$ adalah jumlah bulan-bulan monsun musim dingin (Nov-Feb) atau musim panas (Jun-Sep). Tabel \ref{table:MLD_monsoon} menunjukkan bahwa ketebalan lapisan MLD lebih tebal pada musim dingin dibandingkan musim panas, baik untuk kedalaman minimum maupun maksimum. Ini berlaku untuk lintang $9^\circ$N dan lintang $19^\circ$N.
	
\subsection[Hubungan antara Parameter Meteorologi]{Hubungan antara Parameter Meteorologi}
	
		Terdapat beberapa gaya atmosfer dievaluasi untuk menerangkan perilaku MLD yang dianalisis dari model HYCOM, lima di antaranya adalah temperature udara 2m (AirT), kelembaban spesifik 2m (SHum), laju presipitasi konvektif (CPrecR), tekanan permukaan laut (SLP), dan tekanan angin (TauX dan TauY). Gambar \ref{fig:ncep_1} dan \ref{fig:ncep_2} menunjukkan hasil visualisasi dari kelima gaya tersebut selama 20 tahun dari tahun 2002 hingga 2021 di bagian selatan BoB (lintang $9^\circ$N) dan bagian utara BoB (lintang $19^\circ$N). Gambar ini dihasilkan dari data \textit{reanalysis} NCEP, yang dapat diunduh di https://psl.noaa.gov/.
		
		Nilai ekstrim tahunan dari lima gaya atmosfer untuk dua domain penelitian (lintang $9^\circ$N dan lintang $19^\circ$N) ditunjukkan pada Table \ref{table:NCEP_9} dan \ref{table:NCEP_19} (Lampiran 1). Tabel ini berguna untuk menentukan puncak dan lembah dari setiap parameter meteorologi. Gambar \ref{fig:ncep_1} dan \ref{fig:ncep_2} dimaksudkan untuk melihat perulangan (periodesitas) dari 5 parameter yang diteliti. Sedangkan Tabel \ref{table:NCEP_9} dan \ref{table:NCEP_19} diturunkan dari Gambar \ref{fig:ncep_1} dan \ref{fig:ncep_2}. Tabel ini berfungsi untuk melihat data secara kuantitatif. Nilai ekstrim ini penting untuk melihat pada bulan apa saja nilai ekstrim tersebut terjadi, dan ini dijadikan acuan untuk melihat parameter meteorologi yang mempengaruhi terjadinya nilai ekstrim tersebut.
		
		Secara keseluruhan, tren dari ketiga variabel AirT, SHum, dan tekanan angin (TauX dan TauY) memiliki nilai minimum pada musim dingin atau Desember-Februari dan nilai maksimum pada akhir musim semi dan musim panas pada bulan Mei-Agustus. Di sisi lain, tren dua variabel yang tersisa yaitu CPrecR dan SLP memiliki nilai minimum pada akhir musim semi dan musim panas, atau Mei-Juli, dan nilai maksimum pada musim dingin, dari Desember-Februari.
		
		Tercatat bahwa, pada lintang $9^\circ$N, nilai rata-rata minimum dari lima variabel adalah $25.47, 0.015, -0.0003, 100.2, -0.15$, dan $-0.11$ untuk setiap unit. Sedangkan nilai rata-rata maksimum dari kelima variabel berturut-turut adalah $30.53, 0.022, 0, 101.61,$ $0.225$, dan $0.18$ untuk setiap unit. Apabila nilai ekstrim pada lintang $9^\circ$N dibandingkan dengan nilai ekstrim pada lintang $19^\circ$N, selisih selang minimum dan maksimum pada lintang $19^\circ$N lebih besar dibandingkan pada lintang $9^\circ$N. Akibatnya adalah antara dua domain yang diteliti, nilai ekstrim yang lebih besar terjadi pada garis lintang $19^\circ$N. Selanjutnya, pada garis lintang $19^\circ$N nilai rata-rata minimum untuk kelima variabel adalah $20.99, 0.009, -0.0004, 99.57,$ $-0.11, -0.11$ untuk setiap unit, sedangkan rata-rata nilai maksimum untuk kelima variabel adalah $31.2, 0.023, 0.102, 0.196 , 0.231$ untuk setiap unit.
		
		\begin{figure}[H]
			\centering
			\includegraphics[width=16cm]{contents/final_figure/Figure_5a}
			\caption{Gaya atmosfer pada garis lintang $9^\circ$N dari tahun 2002 hingga 2021, data dari NCEP/NCAR.}
			\label{fig:ncep_1}
		\end{figure}
	
		\begin{figure}[H]
			\centering
			\includegraphics[width=16cm]{contents/final_figure/Figure_5b}
			\caption{Gaya atmosfer pada garis lintang $19^\circ$N dari tahun 2002 hingga 2021, data dari NCEP/NCAR.}
			\label{fig:ncep_2}
		\end{figure}
		
\subsection[Model Iklim]{Model Iklim}
	
		Data selama 20 tahun (2002 - 2021) pada Gambar \ref{fig:ncep_1} dan \ref{fig:ncep_2} diamati untuk menentukan model musiman di kedua domain (lintang $9^\circ$N dan lintang $19^\circ$N) dan bertujuan untuk melihat titik-titik musiman ekstrim setiap tahunnya. Hasil dari model musiman 20 tahun dipotong selama dua tahun terakhir (2020 - 2021) dan disajikan pada Gambar \ref{fig:SM}. Sebagai informasi tambahan, disajikan Tabel \ref{table:extreme} yang merangkum nilai titik ekstrim dari tahun 2002 - 2021 berdasarkan model musiman yang diperoleh.
		
		\begin{table}[H]
			\centering
			\caption{Konstanta dan koefisien prediktor y}
			\label{table:predictor}
			\resizebox{\columnwidth}{!}{%
			\begin{tabular}{|c|ccc|ccc|}
				\hline
				\multirow{2}{*}{Domain} & \multicolumn{3}{c|}{Latitude $9^\circ$N}                                                                               & \multicolumn{3}{c|}{Latitude $19^\circ$N}                                                                              \\ \cline{2-7} 
				& \multicolumn{1}{c|}{$\alpha$} & \multicolumn{1}{c|}{$\beta$} & $\gamma$ & \multicolumn{1}{c|}{$\alpha$} & \multicolumn{1}{c|}{$\beta$} & $\gamma$ \\ \hline
				AirT                    & \multicolumn{1}{c|}{28.125762}             & \multicolumn{1}{c|}{0.219930}             & -0.770494             & \multicolumn{1}{c|}{27.131780}             & \multicolumn{1}{c|}{-0.479135}            & -2.352771             \\ \hline
				SHum                    & \multicolumn{1}{c|}{1.941e-02}             & \multicolumn{1}{c|}{-2.133e-04}           & -8.554e-04            & \multicolumn{1}{c|}{1.811e-02}             & \multicolumn{1}{c|}{-1.194e-03}           & -4.624e-03            \\ \hline
				CPrecR                  & \multicolumn{1}{c|}{-6.752e-05}            & \multicolumn{1}{c|}{2.286e-05}            & 7.285e-06             & \multicolumn{1}{c|}{-5.242e-05}            & \multicolumn{1}{c|}{3.976e-05}            & 6.407e-05             \\ \hline
				SLP                     & \multicolumn{1}{c|}{1.009e+02}             & \multicolumn{1}{c|}{4.803e-02}            & 1.890e-01             & \multicolumn{1}{c|}{1.009e+02}             & \multicolumn{1}{c|}{1.277e-01}            & 6.030e-01             \\ \hline
				TauX                    & \multicolumn{1}{c|}{0.0197177}             & \multicolumn{1}{c|}{-0.0259651}           & -0.0737232            & \multicolumn{1}{c|}{0.0181436}             & \multicolumn{1}{c|}{0.0034581}            & -0.0291363            \\ \hline
				TauY                    & \multicolumn{1}{c|}{0.0189324}             & \multicolumn{1}{c|}{-0.0210637}           & -0.0609275            & \multicolumn{1}{c|}{0.0213336}             & \multicolumn{1}{c|}{0.0037017}            & -0.0511181            \\ \hline
			\end{tabular}%
		}
		\end{table}
		
		Tabel \ref{table:predictor} menunjukkan hasil analisis model musiman dari persamaan (1) untuk gaya atmosfer. Konstanta ($\alpha$) merupakan nilai konstanta yang tidak dipengaruhi oleh musim, sedangkan konstanta ($\beta$) dan ($\gamma$) merupakan variabel yang dipengaruhi oleh musim. Selanjutnya nilai konstanta pada Tabel \ref{table:predictor} dan persamaan (1) menggambarkan model prediksi selama 20 tahun dan hasilnya pada Gambar \ref{fig:SM}.
		
		Nilai rata-rata ($\alpha$) pada Tabel \ref{table:predictor} menunjukkan bahwa nilai AirT lebih besar pada lintang $9^\circ$N dibandingkan pada lintang $19^\circ$N. Hal yang sama terjadi pada variabel SHum dan TauX. Sebaliknya, variabel CPrecR dan TauY lebih besar pada lintang $19^\circ$N daripada lintang $9^\circ$N. Lebih lanjut, untuk variabel SLP nilainya sama di lokasi kedua.
		
		\begin{figure}[H]
			\centering
			\includegraphics[width=15cm]{contents/final_figure/Figure_9}
			\caption{Model musiman untuk pemaksaan atmosfer (a) pada garis lintang $9^\circ$N dan (b) pada garis lintang $19^\circ$N, dari tahun 2020 hingga 2021.}
			\label{fig:SM}
		\end{figure}
	
		Dari Tabel \ref{table:extreme} dan Gambar \ref{fig:SM}, lima gaya yang dikaji - temperature udara 2m (AirT), kelembaban spesifik 2m (SHum), laju presipitasi konvektif (CPrecR), tekanan permukaan laut (SLP), dan tekanan angin U (TauX) dan V ( TauY) - mempengaruhi domain yang diambil sebagai sampel penelitian. Dari kelima variabel tersebut rata-rata nilai ekstrim pada kedua domain terjadi pada bulan Februari dan Agustus setiap tahunnya pada model prediksi yang diperoleh. Tiga parameter AirT, SHum, dan tekanan angin (TauX dan TauY) menunjukkan tren yang sama: minimum di bulan Februari dan maksimum di bulan Agustus. Berbeda dengan tren variabel CPrecR, dan SLP, terlihat bahwa minimum terjadi pada bulan Agustus, dan maksimum terjadi pada bulan Februari. Dari kelima parameter meteorologi yang diperiksa, ekstrim minimum dan maksimum terjadi pada monsun musim dingin dan musim panas.
		
		\begin{table}[H]
			\centering
			\caption{Nilai ekstrim untuk model musiman}
			\label{table:extreme}
			\resizebox{\columnwidth}{!}{%
			\begin{tabular}{|c|c|c|c|c|c|c|l|l|}
				\hline
				Domain                       & Year(s)                       & Extreme & AirT & SHum & CPrecR & SLP & TauX & TauY \\ \hline
				\multirow{4}{*}{2002 - 2021} & \multirow{2}{*}{Latitude $9^\circ$N}  & Min     & Des  & Jan  & Agu    & Jul & Jan  & Jan  \\ \cline{3-9} 
				&                               & Max     & Jun  & Jul  & Feb    & Jan & Jul  & Jul  \\ \cline{2-9} 
				& \multirow{2}{*}{Latitude $19^\circ$N} & Min     & Jan  & Jan  & Jul    & Jul & Jan  & Jan  \\ \cline{3-9} 
				&                               & Max     & Jul  & Jul  & Jan    & Jan & Jul  & Jul  \\ \hline
			\end{tabular}%
		}
		\end{table}
		
\end{spacing}
	
	\vspace{-1pc}
\section[Aplikasi Model]{Aplikasi Model}
\begin{spacing}{1.5}
	\vspace{-1pc}
\subsection[Analisis Chl-a, SST, dan SSS di BoB]{Analisis Chl-a, SST, dan SSS di BoB}
	Gambar \ref{fig:paper1_1}(a) menunjukkan distribusi Chl-a di BoB. Nilai logaritma dari Chl-a dihitung karena nilainya sangat kecil sehingga variasinya menjadi kurang terlihat. Dari hasil logaritma dapat diketahui bahwa nilai Chl-a cukup homogen kecuali pada bagian pantai terutama pada bagian utara BoB. Sebagian besar area permukaan memiliki nilai -1 hingga -2 $mgm^{-3}$. Di sisi lain, di pantai utara BoB, nilai Chl-a bisa mencapai 3 $mgm^{-3}$.
	
	Untuk variabel SST (lihat Gambar \ref{table:paper1_1}(b)), dapat diamati bahwa semakin ke utara menuju pantai, nilai temperature semakin rendah yang ditandai dengan penurunan nilai dari $29^\circ$C menjadi $21^\circ$C. Hal yang sama juga terjadi pada kasus SSS (lihat Gambar \ref{fig:paper1_1}(c)), salinitas di sebagian besar wilayah bernilai 30 Psu, akan tetapi semakin dekat ke pantai utara BoB, nilainya semakin menurun hingga mencapai <20 Psu.
	
	\begin{figure}[H]
		\centering
		\includegraphics[width=15cm]{contents/final_figure_paper1/gambar_1}
		\caption{(a) Distribusi klorofil-a (Chl-a) (satuan dalam $mgm^{-3}$, nilai logaritmik diambil untuk melihat variasi nilai lebih jelas), (b) Temperature permukaan laut (SST) (satuan dalam $^\circ$C), (c) Salinitas permukaan laut (SSS) (satuan dalam Psu) pada Januari 2021.}
		\label{fig:paper1_1}
	\end{figure}
\subsubsection[Analisis Korelasi]{Analisis Korelasi}
	
	Analisis terhadap 3672 populasi untuk masing-masing variabel (Chl-a, SST, dan SSS) (lihat Tabel \ref{table:paper1_1}) menghasilkan analisis korelasi (lihat Gambar \ref{fig:paper1_2}, tabel uji hipotesis (lihat Tabel \ref{table:paper1_2}), dan tabel analisis varians (lihat Tabel \ref{table:paper1_3}).
	
	Tabel \ref{table:paper1_1} menunjukkan bahwa jumlah terbesar dari 3672 data yang digunakan adalah variabel SSS, diikuti oleh SST dan kemudian Chl-a. Hal yang sama terjadi untuk jumlah rata-rata. Nilai varians terkecil yang ditunjukkan oleh data Chl-a menunjukkan data yang seragam untuk Chl-a. Sedangkan variansi terbesar yang ditunjukkan oleh data SSS menunjukkan data yang cukup bervariasi.
	
	\begin{table}[H]
		\centering
		\caption{Rangkuman data yang digunakan dalam penelitian ini}
		\label{table:paper1_1}
		\begin{tabular}{|c|c|c|c|c|}
			\hline
			Grup  & Banyak & Jumlah   & Rata-rata & Variansi \\ \hline
			Chl-a & 3672   & 893.7106 & 0.243385  & 0.126751 \\ \hline
			SST   & 3672   & 101963.2 & 27.76776  & 1.247772 \\ \hline
			SSS   & 3672   & 117570.7 & 32.01817  & 5.915565 \\ \hline
		\end{tabular}
	\end{table}

	Hasil analisis korelasi pada Gambar \ref{fig:paper1_2}(a-c) menunjukkan bahwa korelasi negatif terjadi pada pasangan Chl-a - SST dan Chl-a - SSS, yang ditunjukkan dengan kemiringan garis regresi negatif (garis biru). Sebaliknya, pasangan SST-SSS menunjukkan korelasi positif yang ditunjukkan dengan kemiringan garis regresi yang positif. Hubungan korelasi negatif menunjukkan bahwa perbandingan nilai antara Chl-a dan SST atau Chl-a dengan SSS berbanding terbalik sehingga jika nilai Chl-a tinggi pada suatu titik koordinat, SST akan memiliki nilai yang rendah pada titik koordinat tersebut. Sebaliknya korelasi positif menunjukkan bahwa perbandingan nilai antara SST dan SSS berbanding lurus sehingga jika nilai SST tinggi pada suatu titik koordinat, maka akan diimbangi dengan nilai SSS yang tinggi pada titik koordinat tersebut.Selanjutnya, evaluasi variasi antara nilai pasangan variabel ditampilkan menggunakan RMSE. Hasilnya adalah pasangan SST - SSS memiliki variasi nilai yang lebih kecil, diikuti oleh Chl-a - SST dan kemudian Chl-a - SSS.
	
	\begin{figure}[H]
		\centering
		\includegraphics[width=15cm]{contents/final_figure_paper1/gambar_2}
		\caption{Analisis korelasi untuk (a) Klorofil-a - suhu permukaan laut, (b) Klorofil-a - salinitas permukaan laut, dan (c) Suhu permukaan laut - salinitas permukaan laut}
		\label{fig:paper1_2}
	\end{figure}
	
	Gambar \ref{fig:paper1_3} menunjukkan koefisien korelasi dari pasangan variabel yang dibandingkan. Terlihat bahwa pasangan SST - SSS memiliki korelasi paling kuat, diikuti oleh Chl-a - SSS dan kemudian Chl-a - SST. Menurut \shortciteNP{Schober2018}, Chl-a - SST memiliki korelasi sedang dan negatif, Chl-a - SSS memiliki korelasi sedang dan negatif, dan SST - SSS memiliki korelasi yang kuat dan positif. Jika dihitung koefisien determinan, akan diperoleh
	\begin{equation*}
		\begin{aligned}
			r^2_\text{Chl-a - SST}&=0.2511 \;(25.11\%) \\
			r^2_\text{Chl-a - SSS}&=0.4598 \;(45.98\%) \\
			r^2_\text{SST - SSS}&=0.5480 \;(54.8\%) 
		\end{aligned}	
	\end{equation*} 
	Hal ini menunjukkan bahwa kontribusi atau pengaruh Chl-a terhadap SST sebesar 25.11\%, pengaruh Chl-a terhadap SST sebesar 45.98\%, dan pengaruh SST terhadap SSS sebesar 54.8\%. 
	
	\begin{figure}[H]
		\centering
		\includegraphics[width=15cm]{contents/final_figure_paper1/gambar_3}
		\caption{Koefisien korelasi (tanpa unit)}
		\label{fig:paper1_3}
	\end{figure}

\subsubsection[Uji Hipotesis dan Analisis Variansi]{Uji Hipotesis dan Analisis Variansi}
	
	Uji hipotesis dilakukan untuk mengetahui apakah koefisien korelasi ($r$) berbeda secara signifikan atau tidak. Berdasarkan hasil pengujian hipotesis pada Tabel \ref{table:paper1_2}, karena 
	\begin{equation*}
		\begin{aligned}
			\{|t_{\text{Chl-a - SST}}|,|t_{\text{Chl-a - SSS}}|,|t_{\text{SST - SSS}}|\}\geq t_{\text{tabel}} \quad \text{dan} \quad \\
			\{|r_{\text{Chl-a - SST}}|,|r_{\text{Chl-a - SSS}}|,|r_{\text{SST - SSS}}|\}\geq r_{\text{tabel}} 	
		\end{aligned}	
	\end{equation*} 
	
	maka hipotesis ($H_o$) bahwa tidak ada hubungan linier antar pasangan variabel, ditolak sehingga terdapat korelasi yang nyata dan negatif untuk pasangan-pasangan Chl-a - SST dan Chl- a - SSS. Di sisi lain, korelasinya nyata dan positif untuk pasangan SST - SSS.
	
	\begin{table}[H]
		\centering
		\caption{Uji hipotesis}
		\label{table:paper1_2}
		\resizebox{\columnwidth}{!}{%
		\begin{tabular}{|c|c|c|c|c|}
			\hline
			Pasangan    & t-nilai & t-tabel ($\alpha=5\%$)                  & r-nilai (koefisien korelasi) & r-tabel ($\alpha=5\%$)               \\ \hline
			Chl-a - SST & -40.535 & \multirow{3}{*}{(0.025; 3670) = 1.9606} & -0.501                       & \multirow{3}{*}{(5\%; 3670) = 0.032} \\ \cline{1-2} \cline{4-4}
			Chl-a - SSS & -76.043  &  & -0.678 &  \\ \cline{1-2} \cline{4-4}
			SST - SSS   & 103.0864 &  & 0.74   &  \\ \hline
		\end{tabular}%
	}
	\end{table}
	Berdasarkan Tabel \ref{table:paper1_3}, karena 
	\begin{equation*}
		\begin{aligned}
			\{|F_{\text{Chl-a - SST}}|,|F_{\text{Chl-a - SSS}}|,|F_{\text{SST - SSS}}|\}\geq F_{\text{tabel}} 	
		\end{aligned}	
	\end{equation*} 
	dan $P-\text{nilai}\leq (\alpha=0.05)$ untuk masing-masing pasangan, dapat disimpulkan bahwa: (1) variabel Chl-a berpengaruh signifikan terhadap variasi nilai variabel SST (2) variabel Chl-a berpengaruh signifikan terhadap variasi nilai variabel SSS, dan (3) variabel SST berpengaruh signifikan terhadap variasi nilai variabel SSS.
	
	\begin{table}[H]
		\centering
		\caption{Analisis variansi (ANOVA)}
		\label{table:paper1_3}
		\resizebox{\columnwidth}{!}{%
		\begin{tabular}{|c|c|c|c|c|c|c|c|}
			\hline
			Pasangan & Sumber Variasi & Jumlah Kuadrat & Derajat Kebebasan & Rata-rata Jumlah Kuadrat & F & P-nilai & F-tabel ($\alpha=5\%$) \\ \hline
			\multirow{3}{*}{Chl-a - SST} & Between Groups & 1390937.466 & 1    & 1390937  & 2023883  & 0 & 3.842726 \\ \cline{2-8} 
			& Within Groups  & 5045.875117 & 7342 & 0.687262 &          &   &          \\ \cline{2-8} 
			& Total          & 1395983.341 & 7343 &          &          &   &          \\ \hline
			\multirow{3}{*}{Chl-a - SSS} & Between Groups & 1853693     & 1    & 1853693  & 613570.5 & 0 & 3.842726 \\ \cline{2-8} 
			& Within Groups  & 22181.34    & 7342 & 3.021158 &          &   &          \\ \cline{2-8} 
			& Total          & 1875875     & 7343 &          &          &   &          \\ \hline
			\multirow{3}{*}{SST - SSS}   & Between Groups & 33169.15    & 1    & 33169.15 & 9260.809 & 0 & 3.842726 \\ \cline{2-8} 
			& Within Groups  & 26296.61    & 7342 & 3.581669 &          &   &          \\ \cline{2-8} 
			& Total          & 59465.76    & 7343 &          &          &   &          \\ \hline
		\end{tabular}%
	}
	\end{table}
	
\subsection[Hubungan antara \textit{Eddies} dan SSH di Samudera Hindia]{Hubungan antara \textit{Eddies} dan SSH di Samudera Hindia}
	
	Angin yang berhembus pada domain penelitian yang ditunjukkan dalam gambar \ref{fig:paper2_1}a tampak tidak begitu teratur dan terlihat bahwa memiliki kondisi yang berbeda di sisi timur laut dan barat daya yang dipisahkan oleh pulau Sumatera. Tampak bahwa di bagian utara domain yang meliputi wilayah bagian laut Cina Selatan, selat Malaka, dan perairan Aceh. Secara umum, di wilayah ini, angin berhembus dari arah timur menuju ke barat. 
	
	Di sisi lain, angin berhembus dari Barat menuju Timur di bagian tengah dan selatan domain yang merupakan wilayah laut lepas samudera Hindia. Dari hasil pengamatan, diketahui bahwa angin paling kencang berada pada bagian laut Cina selatan dan laut lepas samudera hindia dengan nilai kecepatan angin berkisar antara 2 m/s - 6 m/s dan tekanan angin berkisar antara 0.1 Pa - 1.3 Pa. Angin yang tidak begitu kencang mendominasi di wilayah Selat Malaka dan Perairan Barat Aceh dengan kecepatan angin berkisar antara 0.5 m/s - 2.5 m/s dan tekanan angin berkisar antara 0.01 Pa - 0.1 Pa. Penggambaran angin ini sekaligus memvalidasi hasil temuan di Samudera Hindia \shortcite{Schott2001}.
	
	\begin{figure}[H]
		\centering
		\includegraphics[width=15cm]{contents/final_figure_paper2/gambar_1}
		\caption{(a) Vektor tekanan angin (panah hitam, dalam Pa) dan kecepatan angin (warna, dalam m/s) pada bulan April 2020, (b) Vektor kecepatan arus permukaan laut (UV) (panah hitam, dalam m/s) dan Ketinggian permukaan laut di atas geoid (SSH) (warna, dalam m/s) pada bulan April 2020. Lingkaran merah menunjukkan lokasi pusaran yang terjadi dan (c) Temperature potensial air laut permukaan (SST) (warna, dalam $^\circ$C) pada bulan April 2020.}
		\label{fig:paper2_1}
	\end{figure}
	
	Dalam Gambar \ref{fig:paper2_1}b, tampak bahwa perbedaan nilai SSH sangat kontras terjadi. Nilai SSH lebih tinggi di bagian laut Cina Selatan dibandingkan wilayah domain yang lain. Dari hasil SSH yang digambarkan tampak bahwa, SSH yang rendah muncul disekitar lokasi pusaran. Pusaran (\textit{Eddies}) terjadi dibeberapa titik lokasi dalam penelitian khususnya yang bersesuaian dengan bagian belahan bumi Utara (\textit{Northern Hemisphere}), yaitu Selat Malaka pada koordinat ($98^\circ$E, $6^\circ$N), dan Perairan Aceh pada koordinat ($93^\circ$E, $5^\circ$N). Kedua titik lokasi pusaran yang terjadi memiliki arah arus yang berlawanan arah jarum jam. Sirkulasi laut yang dipengaruhi oleh angin mengakibatkan terjadinya pusaran yang berlawanan arah jarum jam. Aliran pada lapisan permukaan Ekman di kedua titik lokasi pusaran berasosiasi dengan perilaku angin dan memiliki karakteristik divergent yang mendorong gerakan vertikal dari bawah ke atas (peristiwa \textit{upwelling}) \shortcite{10.2307/24805614}. Fenomena ini membawa air dengan konsentrasi tinggi nutrisi seperti nitrat dan fosfat ke permukaan laut. Perairan yang kaya nutrisi ini menjadi pendorong bagi pertumbuhan plankton dan ganggang mikroskopis di perairan tersebut. Karena air dari kedalaman yang dibawa ke permukaan seringkali mengandung kandungan nutrisi yang tinggi, upwelling dapat membantu pertumbuhan rumput laut dan plankton. Selanjutnya, rumput laut dan plankton menjadi penyedia sumber makanan bagi ikan-ikan, mamalia laut, dan burung-burung di daerah tersebut.
	
	Kedua pusaran pada wilayah Selat Malaka dan Perairan Aceh yang diidentifikasi oleh SSH memiliki kesesuaian dengan fenomena SST yang rendah pada lokasi terjadinya pusaran. Hal ini ditunjukkan pada Gambar \ref{fig:paper2_1}c, dimana pada wilayah selat Malaka nilai SST berkisar antara $24^\circ$C - $28^\circ$C dan bernilai semakin rendah seiring menuju ke pusat pusaran. Untuk wilayah perairan Aceh, nilai SST berkisar antara $27^\circ$C - $28^\circ$C dan nilai terendah berada di pusat pusaran.
\end{spacing}
%=====================================================================
% BAB V
\chapter{PENUTUP}
\vspace{1.5pc}
\begin{spacing}{1.5}
\section[Kesimpulan]{KESIMPULAN}
\subsection[MLD]{MLD}

	Deskripsi siklus musiman MLD berbasis-temperature telah dikonstruksi dengan menggunakan data temperature dari HYCOM. Definisi MLD yang digunakan dalam penelitian mengikuti metode \textit{threshold} $0.1^\circ$C, dengan referensi dari permukaan laut. Kemudian dari parameter-parameter meteorology NCEP/NCAR, dibentuk model iklim untuk mengidentifikasi parameter yang mempengaruhi ketebalan MLD. Model iklim ini sangat berguna untuk mengestimasi puncak dan lembah dari parameter-parameter meteorologi \shortcite{Toharudin2021,Yates2020,Ikhwan2022,Haridhi2016}. 
	
	Nilai estimasi MLD di kedua lintang pada Table \ref{table:MLD_thickness} menunjukkan bahwa variasi MLD pada lintang $19^\circ$N lebih besar dibandingkan lintang $9^\circ$N. Terlebih, ketebalan MLD antara 2 lintang tidak menujukkan posisi dalam bulan yang sama. MLD yang dangkal pada lintang $9^\circ$N terletak pada bulan Maret and Desember, sementara itu MLD yang dangkal pada lintang $19^\circ$N terletak pada bulan Maret, Mei, and Juni. Di sisi lain MLD paling dangkal pada lintang $9^\circ$N terletak pada bulan Februari and November, sedangkan MLD untuk lintang $19^\circ$N terletak pada bulan Maret and April.
	
	Hasil penelitian ini menunjukkan bahwa peran musim sangat tinggi. Hal ini terlihat dari rata-rata nilai MLD batas bawah (MinX) dan batas atas (MaxX) yang ditunjukkan pada Tabel \ref{table:MLD_monsoon}, dimana MLD lebih tebal pada bulan-bulan monsun musim dingin (November-Februari) dibandingkan pada bulan-bulan monsun musim panas (Juni-September). Ini berlaku untuk kedua lintang: lintang $9^\circ$N dan lintang $19^\circ$N.
	
	Hasil model musiman menunjukkan bahwa 3 parameter yaitu temperature udara, kelembaban spesifik, dan tekanan angin terjadi ekstrim minimum pada bulan Desember dan Januari dan ekstrim maksimum pada bulan Juni dan Juli. Terkait dengan MLD, nilai minimum ekstrim dari 3 parameter ini memberikan kontribusi positif terhadap ketebalan MLD. Sementara nilai maksimum ekstrim membuat MLD menjadi lebih tipis di bulan-bulan musim panas. Ini berlaku untuk kedua garis lintang.
	
	Untuk 2 parameter lainnya, laju presipitasi konvektif dan tekanan permukaan laut yang ekstrim minimum terjadi pada bulan Juli dan Agustus, dan hal ini berkontribusi pada tipisnya lapisan MLD. Sementara itu, pada bulan Januari dan Februari, parameter tersebut terjadi pada tingkat maksimum yang ekstrem, dan ini berkontribusi pada MLD yang lebih tebal. Ini juga terjadi di kedua garis lintang.
	
	Model musiman untuk lima parameter meteorologi menunjukkan bahwa semua parameter yang mencapai ekstrim maksimum dan minimum hanya terjadi pada bulan Desember-Februari (musim dingin) dan Juni-Agustus (musim panas). Oleh karena itu, dapat disimpulkan bahwa semua parameter meteorologi mengikuti musim monsun.
	
\subsection[Analisis Chl-a, SST, dan SSS]{Analisis Chl-a, SST, dan SSS}
	Perbedaan yang mencolok antara klorofil-a, temperature permukaan laut, dan salinitas permukaan laut di dekat pantai khususnya di bagian utara domain dalam gambar \ref{fig:paper1_2}(a-c) menjadi landasan dalam menganalisis hubungan antara Chl-a - SST, Chl-a - SSS dan SST - SSS dengan menggunakan analisis-analisis statistik. Hasilnya adalah terdapat korelasi antara pasangan-pasangan variable tersebut. Korelasi sedang dan negative terjadi untuk pasangan Chl-a - SST, dan Chl-a - SSS dengan koefisien determinansi berturut-turut sebesar 25.11\% dan 45.98\% yang menunjukkan pengaruh Chl-a terhadap SST ataupun pengaruh Chl-a terhadap SSS. Sebaliknya, korelasi kuat dan positif ditunjukkan oleh pasangan SST - SSS dengan koefisien determinansi sebesar yang menunjukkan pengaruh SST terhadap SSS. Uji hipotesis yang dilakukan menunjukkan bahwa penarikan kesimpulan atas hubungan-hubungan korelasi ini dapat diterima. Lebih lanjut, dari analisis variansi yang diperoleh, ditemukan bahwa Chl-a berpengaruh nyata terhadap variasi SST dan SSS, SST berpengaruh nyata terhadap variasi SSS.
	
\subsection[\textit{Eddies} dan SSH]{\textit{Eddies} dan SSH}

	Dari penelitian yang dilakukan diperoleh hasil yaitu terjadi dua arus yang berlawanan arah jarum jam (\textit{counter-clockwise current}) di Perairan Aceh dan di bagian utara Selat Malaka pada Bulan April 2020. Karena wilayah tersebut terletak di \textit{Northern Hemisphere}, maka terjadi Ekman transport ke arah luar dari lingkaran \textit{eddies} sehingga terjadi kekosongan massa air pada lingkaran eddies tersebut. Karena kekosongan massa air, maka harus diisi dari bawah. Karakteristik air dari bawah yang dingin, menghasilkan SST yang dingin pula. Ini terkonfirmasi dari SST yang rendah dibandingkan daerah sekitarnya karena telah terjadi peristiwa \textit{upwelling}, pada dua \textit{eddies} yang dibahas sebelumnya.
	
\section[Saran]{SARAN}

	Kami merekomendasikan bahwa penyelidikan parameter meteorologi lainnya seperti fluks panas, radiasi gelombang panjang dan pendek, fluks air tawar, tutupan awan dan parameter lainnya harus dilanjutkan.

\end{spacing}
%=====================================================================
% Halaman Daftar Pustaka
\pagebreak
%\addcontentsline{toc}{chapter}{\textbf{DAFTAR PUSTAKA}}
\renewcommand\bibname{DAFTAR PUSTAKA}	
\bibliographystyle{packages/apacite}
\bibliography{contents/ReferenceMendeley.bib}
%\{DAFTAR}
%=====================================================================
% Halaman Lampiran
\newpage
\newappendix{Lampiran 1. Tabulasi Data NCEP/NCAR}
\addcontentsline{toc}{chapter}{LAMPIRAN}
% Please add the following required packages to your document preamble:
% \usepackage{multirow}
% \usepackage[normalem]{ulem}
% \useunder{\uline}{\ul}{}
\begin{table}[H]
	\centering
	\caption{Tekanan atmosfer dengan nilai ekstrim pada garis lintang $9^\circ$N}
	\label{table:NCEP_9}
	\resizebox{\columnwidth}{!}{%
		\begin{tabular}{|c|llll|llll|llll|llll|llll|llll|}
			\hline
			\multirow{3}{*}{Years} &
			\multicolumn{4}{c|}{AirT ($^\circ$C)} &
			\multicolumn{4}{c|}{SHum (kg/kg)} &
			\multicolumn{4}{c|}{CPrecR ($kg/m^2s$)} &
			\multicolumn{4}{c|}{SLP (kPa)} &
			\multicolumn{4}{c|}{TauX (Pa)} &
			\multicolumn{4}{c|}{TauY (Pa)} \\ \cline{2-25} 
			&
			\multicolumn{2}{c|}{Min} &
			\multicolumn{2}{c|}{Max} &
			\multicolumn{2}{c|}{Min} &
			\multicolumn{2}{c|}{Max} &
			\multicolumn{2}{c|}{Min} &
			\multicolumn{2}{c|}{Max} &
			\multicolumn{2}{c|}{Min} &
			\multicolumn{2}{c|}{Max} &
			\multicolumn{2}{c|}{Min} &
			\multicolumn{2}{c|}{Max} &
			\multicolumn{2}{c|}{Min} &
			\multicolumn{2}{c|}{Max} \\ \cline{2-25} 
			&
			\multicolumn{1}{c|}{Value} &
			\multicolumn{1}{c|}{Time} &
			\multicolumn{1}{c|}{Value} &
			\multicolumn{1}{c|}{Time} &
			\multicolumn{1}{c|}{Value} &
			\multicolumn{1}{c|}{Time} &
			\multicolumn{1}{c|}{Value} &
			\multicolumn{1}{c|}{Time} &
			\multicolumn{1}{c|}{Value} &
			\multicolumn{1}{c|}{Time} &
			\multicolumn{1}{c|}{Value} &
			\multicolumn{1}{c|}{Time} &
			\multicolumn{1}{c|}{Value} &
			\multicolumn{1}{c|}{Time} &
			\multicolumn{1}{c|}{Value} &
			\multicolumn{1}{c|}{Time} &
			\multicolumn{1}{c|}{Value} &
			\multicolumn{1}{c|}{Time} &
			\multicolumn{1}{c|}{Value} &
			\multicolumn{1}{c|}{Time} &
			\multicolumn{1}{c|}{Value} &
			\multicolumn{1}{c|}{Time} &
			\multicolumn{1}{c|}{Value} &
			\multicolumn{1}{c|}{Time} \\ \hline
			2002 &
			\multicolumn{1}{l|}{25.69} &
			\multicolumn{1}{l|}{Feb} &
			\multicolumn{1}{l|}{31} &
			May &
			\multicolumn{1}{l|}{0.0157} &
			\multicolumn{1}{l|}{Jan} &
			\multicolumn{1}{l|}{0.0223} &
			May &
			\multicolumn{1}{l|}{-0.0003} &
			\multicolumn{1}{l|}{Nov} &
			\multicolumn{1}{l|}{0} &
			Jan &
			\multicolumn{1}{l|}{100.27} &
			\multicolumn{1}{l|}{Jun} &
			\multicolumn{1}{l|}{101.77} &
			Feb &
			\multicolumn{1}{l|}{-0.2} &
			\multicolumn{1}{l|}{Jan} &
			\multicolumn{1}{l|}{0.199} &
			Sep &
			\multicolumn{1}{l|}{-0.11} &
			\multicolumn{1}{l|}{Jan} &
			\multicolumn{1}{l|}{0.168} &
			May \\ \hline
			2003 &
			\multicolumn{1}{l|}{25.9} &
			\multicolumn{1}{l|}{Jan} &
			\multicolumn{1}{l|}{30.45} &
			May &
			\multicolumn{1}{l|}{0.0153} &
			\multicolumn{1}{l|}{Jan} &
			\multicolumn{1}{l|}{0.0217} &
			May &
			\multicolumn{1}{l|}{-0.0004} &
			\multicolumn{1}{l|}{May} &
			\multicolumn{1}{l|}{0} &
			Jan &
			\multicolumn{1}{l|}{100.21} &
			\multicolumn{1}{l|}{May} &
			\multicolumn{1}{l|}{101.58} &
			Jan &
			\multicolumn{1}{l|}{-0.18} &
			\multicolumn{1}{l|}{Dec} &
			\multicolumn{1}{l|}{0.214} &
			Aug &
			\multicolumn{1}{l|}{-0.1} &
			\multicolumn{1}{l|}{Dec} &
			\multicolumn{1}{l|}{0.194} &
			May \\ \hline
			2004 &
			\multicolumn{1}{l|}{25.33} &
			\multicolumn{1}{l|}{Feb} &
			\multicolumn{1}{l|}{30.32} &
			Apr &
			\multicolumn{1}{l|}{0.015} &
			\multicolumn{1}{l|}{Feb} &
			\multicolumn{1}{l|}{0.0215} &
			May &
			\multicolumn{1}{l|}{-0.0003} &
			\multicolumn{1}{l|}{May} &
			\multicolumn{1}{l|}{0} &
			Jan &
			\multicolumn{1}{l|}{100.34} &
			\multicolumn{1}{l|}{Aug} &
			\multicolumn{1}{l|}{101.58} &
			Feb &
			\multicolumn{1}{l|}{-0.2} &
			\multicolumn{1}{l|}{Dec} &
			\multicolumn{1}{l|}{0.22} &
			May &
			\multicolumn{1}{l|}{-0.09} &
			\multicolumn{1}{l|}{Dec} &
			\multicolumn{1}{l|}{0.155} &
			May \\ \hline
			2005 &
			\multicolumn{1}{l|}{25.8} &
			\multicolumn{1}{l|}{Dec} &
			\multicolumn{1}{l|}{30.69} &
			May &
			\multicolumn{1}{l|}{0.0158} &
			\multicolumn{1}{l|}{Mar} &
			\multicolumn{1}{l|}{0.0217} &
			May &
			\multicolumn{1}{l|}{-0.0003} &
			\multicolumn{1}{l|}{Sep} &
			\multicolumn{1}{l|}{0} &
			Jan &
			\multicolumn{1}{l|}{100.19} &
			\multicolumn{1}{l|}{Sep} &
			\multicolumn{1}{l|}{101.72} &
			Mar &
			\multicolumn{1}{l|}{-0.16} &
			\multicolumn{1}{l|}{Jan} &
			\multicolumn{1}{l|}{0.251} &
			Jul &
			\multicolumn{1}{l|}{-0.1} &
			\multicolumn{1}{l|}{Jan} &
			\multicolumn{1}{l|}{0.204} &
			Jul \\ \hline
			2006 &
			\multicolumn{1}{l|}{25.3} &
			\multicolumn{1}{l|}{Jan} &
			\multicolumn{1}{l|}{30.69} &
			Apr &
			\multicolumn{1}{l|}{0.014} &
			\multicolumn{1}{l|}{Jan} &
			\multicolumn{1}{l|}{0.0222} &
			Apr &
			\multicolumn{1}{l|}{-0.0003} &
			\multicolumn{1}{l|}{May} &
			\multicolumn{1}{l|}{0} &
			Jan &
			\multicolumn{1}{l|}{100.22} &
			\multicolumn{1}{l|}{Jun} &
			\multicolumn{1}{l|}{101.58} &
			Jan &
			\multicolumn{1}{l|}{-0.18} &
			\multicolumn{1}{l|}{Dec} &
			\multicolumn{1}{l|}{0.226} &
			Jul &
			\multicolumn{1}{l|}{-0.11} &
			\multicolumn{1}{l|}{Dec} &
			\multicolumn{1}{l|}{0.171} &
			Jul \\ \hline
			2007 &
			\multicolumn{1}{l|}{25.49} &
			\multicolumn{1}{l|}{Jan} &
			\multicolumn{1}{l|}{30.47} &
			May &
			\multicolumn{1}{l|}{0.0141} &
			\multicolumn{1}{l|}{Mar} &
			\multicolumn{1}{l|}{0.022} &
			May &
			\multicolumn{1}{l|}{-0.0003} &
			\multicolumn{1}{l|}{Jun} &
			\multicolumn{1}{l|}{0} &
			Jan &
			\multicolumn{1}{l|}{100.06} &
			\multicolumn{1}{l|}{Jun} &
			\multicolumn{1}{l|}{101.62} &
			Feb &
			\multicolumn{1}{l|}{-0.18} &
			\multicolumn{1}{l|}{Jan} &
			\multicolumn{1}{l|}{0.177} &
			Jul &
			\multicolumn{1}{l|}{-0.14} &
			\multicolumn{1}{l|}{Jan} &
			\multicolumn{1}{l|}{0.16} &
			Jul \\ \hline
			2008 &
			\multicolumn{1}{l|}{25.84} &
			\multicolumn{1}{l|}{Jan} &
			\multicolumn{1}{l|}{30.09} &
			May &
			\multicolumn{1}{l|}{0.0159} &
			\multicolumn{1}{l|}{Jan} &
			\multicolumn{1}{l|}{0.0217} &
			May &
			\multicolumn{1}{l|}{-0.0003} &
			\multicolumn{1}{l|}{Apr} &
			\multicolumn{1}{l|}{0} &
			Jan &
			\multicolumn{1}{l|}{100.29} &
			\multicolumn{1}{l|}{Aug} &
			\multicolumn{1}{l|}{101.52} &
			Feb &
			\multicolumn{1}{l|}{-0.11} &
			\multicolumn{1}{l|}{Nov} &
			\multicolumn{1}{l|}{0.215} &
			Jul &
			\multicolumn{1}{l|}{-0.1} &
			\multicolumn{1}{l|}{Dec} &
			\multicolumn{1}{l|}{0.204} &
			Jul \\ \hline
			2009 &
			\multicolumn{1}{l|}{24.93} &
			\multicolumn{1}{l|}{Jan} &
			\multicolumn{1}{l|}{30.35} &
			May &
			\multicolumn{1}{l|}{0.0157} &
			\multicolumn{1}{l|}{Jan} &
			\multicolumn{1}{l|}{0.0218} &
			May &
			\multicolumn{1}{l|}{-0.0003} &
			\multicolumn{1}{l|}{May} &
			\multicolumn{1}{l|}{0} &
			Jan &
			\multicolumn{1}{l|}{100.14} &
			\multicolumn{1}{l|}{May} &
			\multicolumn{1}{l|}{101.68} &
			Jan &
			\multicolumn{1}{l|}{-0.13} &
			\multicolumn{1}{l|}{Jan} &
			\multicolumn{1}{l|}{0.242} &
			Jul &
			\multicolumn{1}{l|}{-0.11} &
			\multicolumn{1}{l|}{Feb} &
			\multicolumn{1}{l|}{0.18} &
			May \\ \hline
			2010 &
			\multicolumn{1}{l|}{25.58} &
			\multicolumn{1}{l|}{Dec} &
			\multicolumn{1}{l|}{30.9} &
			May &
			\multicolumn{1}{l|}{0.0152} &
			\multicolumn{1}{l|}{Feb} &
			\multicolumn{1}{l|}{0.0226} &
			May &
			\multicolumn{1}{l|}{-0.0004} &
			\multicolumn{1}{l|}{May} &
			\multicolumn{1}{l|}{0} &
			Jan &
			\multicolumn{1}{l|}{100.22} &
			\multicolumn{1}{l|}{Aug} &
			\multicolumn{1}{l|}{101.69} &
			Jan &
			\multicolumn{1}{l|}{-0.14} &
			\multicolumn{1}{l|}{Dec} &
			\multicolumn{1}{l|}{0.186} &
			Jul &
			\multicolumn{1}{l|}{-0.13} &
			\multicolumn{1}{l|}{Jan} &
			\multicolumn{1}{l|}{0.15} &
			May \\ \hline
			2011 &
			\multicolumn{1}{l|}{25.3} &
			\multicolumn{1}{l|}{Jan} &
			\multicolumn{1}{l|}{29.82} &
			Jun &
			\multicolumn{1}{l|}{0.0164} &
			\multicolumn{1}{l|}{Feb} &
			\multicolumn{1}{l|}{0.0212} &
			May &
			\multicolumn{1}{l|}{-0.0003} &
			\multicolumn{1}{l|}{Dec} &
			\multicolumn{1}{l|}{0} &
			Feb &
			\multicolumn{1}{l|}{100.28} &
			\multicolumn{1}{l|}{Jul} &
			\multicolumn{1}{l|}{101.39} &
			Feb &
			\multicolumn{1}{l|}{-0.22} &
			\multicolumn{1}{l|}{Nov} &
			\multicolumn{1}{l|}{0.222} &
			Jun &
			\multicolumn{1}{l|}{-0.13} &
			\multicolumn{1}{l|}{Jan} &
			\multicolumn{1}{l|}{0.159} &
			Jul \\ \hline
			2012 &
			\multicolumn{1}{l|}{25.34} &
			\multicolumn{1}{l|}{Jan} &
			\multicolumn{1}{l|}{30.22} &
			Apr &
			\multicolumn{1}{l|}{0.0158} &
			\multicolumn{1}{l|}{Mar} &
			\multicolumn{1}{l|}{0.0216} &
			May &
			\multicolumn{1}{l|}{-0.0003} &
			\multicolumn{1}{l|}{Nov} &
			\multicolumn{1}{l|}{0} &
			Feb &
			\multicolumn{1}{l|}{100.23} &
			\multicolumn{1}{l|}{Jul} &
			\multicolumn{1}{l|}{101.52} &
			Jan &
			\multicolumn{1}{l|}{-0.13} &
			\multicolumn{1}{l|}{Dec} &
			\multicolumn{1}{l|}{0.229} &
			Jun &
			\multicolumn{1}{l|}{-0.12} &
			\multicolumn{1}{l|}{Jan} &
			\multicolumn{1}{l|}{0.173} &
			May \\ \hline
			2013 &
			\multicolumn{1}{l|}{25.4} &
			\multicolumn{1}{l|}{Dec} &
			\multicolumn{1}{l|}{30.65} &
			May &
			\multicolumn{1}{l|}{0.0155} &
			\multicolumn{1}{l|}{Mar} &
			\multicolumn{1}{l|}{0.0225} &
			May &
			\multicolumn{1}{l|}{-0.0004} &
			\multicolumn{1}{l|}{Nov} &
			\multicolumn{1}{l|}{0} &
			Jan &
			\multicolumn{1}{l|}{100.16} &
			\multicolumn{1}{l|}{Jun} &
			\multicolumn{1}{l|}{101.52} &
			Jan &
			\multicolumn{1}{l|}{-0.15} &
			\multicolumn{1}{l|}{Dec} &
			\multicolumn{1}{l|}{0.215} &
			Jun &
			\multicolumn{1}{l|}{-0.11} &
			\multicolumn{1}{l|}{Dec} &
			\multicolumn{1}{l|}{0.169} &
			Jul \\ \hline
			2014 &
			\multicolumn{1}{l|}{24.89} &
			\multicolumn{1}{l|}{Feb} &
			\multicolumn{1}{l|}{30.48} &
			May &
			\multicolumn{1}{l|}{0.015} &
			\multicolumn{1}{l|}{Mar} &
			\multicolumn{1}{l|}{0.022} &
			Jun &
			\multicolumn{1}{l|}{-0.0003} &
			\multicolumn{1}{l|}{Jul} &
			\multicolumn{1}{l|}{0} &
			Feb &
			\multicolumn{1}{l|}{100.11} &
			\multicolumn{1}{l|}{Jun} &
			\multicolumn{1}{l|}{101.62} &
			Jan &
			\multicolumn{1}{l|}{-0.16} &
			\multicolumn{1}{l|}{Dec} &
			\multicolumn{1}{l|}{0.235} &
			Jul &
			\multicolumn{1}{l|}{-0.13} &
			\multicolumn{1}{l|}{Jan} &
			\multicolumn{1}{l|}{0.171} &
			Jul \\ \hline
			2015 &
			\multicolumn{1}{l|}{25.72} &
			\multicolumn{1}{l|}{Jan} &
			\multicolumn{1}{l|}{30.6} &
			May &
			\multicolumn{1}{l|}{0.0142} &
			\multicolumn{1}{l|}{Feb} &
			\multicolumn{1}{l|}{0.0219} &
			May &
			\multicolumn{1}{l|}{-0.0004} &
			\multicolumn{1}{l|}{Nov} &
			\multicolumn{1}{l|}{0} &
			Jan &
			\multicolumn{1}{l|}{100.29} &
			\multicolumn{1}{l|}{Jul} &
			\multicolumn{1}{l|}{101.74} &
			Jan &
			\multicolumn{1}{l|}{-0.13} &
			\multicolumn{1}{l|}{Feb} &
			\multicolumn{1}{l|}{0.262} &
			Jul &
			\multicolumn{1}{l|}{-0.11} &
			\multicolumn{1}{l|}{Dec} &
			\multicolumn{1}{l|}{0.19} &
			Jul \\ \hline
			2016 &
			\multicolumn{1}{l|}{25.79} &
			\multicolumn{1}{l|}{Dec} &
			\multicolumn{1}{l|}{31.1} &
			May &
			\multicolumn{1}{l|}{0.0162} &
			\multicolumn{1}{l|}{Dec} &
			\multicolumn{1}{l|}{0.0225} &
			May &
			\multicolumn{1}{l|}{-0.0004} &
			\multicolumn{1}{l|}{Dec} &
			\multicolumn{1}{l|}{0} &
			Feb &
			\multicolumn{1}{l|}{100.3} &
			\multicolumn{1}{l|}{May} &
			\multicolumn{1}{l|}{101.65} &
			Feb &
			\multicolumn{1}{l|}{-0.19} &
			\multicolumn{1}{l|}{Jan} &
			\multicolumn{1}{l|}{0.232} &
			Aug &
			\multicolumn{1}{l|}{-0.15} &
			\multicolumn{1}{l|}{Jan} &
			\multicolumn{1}{l|}{0.206} &
			May \\ \hline
			2017 &
			\multicolumn{1}{l|}{24.15} &
			\multicolumn{1}{l|}{Feb} &
			\multicolumn{1}{l|}{30.64} &
			May &
			\multicolumn{1}{l|}{0.0141} &
			\multicolumn{1}{l|}{Feb} &
			\multicolumn{1}{l|}{0.0222} &
			May &
			\multicolumn{1}{l|}{-0.0004} &
			\multicolumn{1}{l|}{Dec} &
			\multicolumn{1}{l|}{0} &
			Feb &
			\multicolumn{1}{l|}{100.25} &
			\multicolumn{1}{l|}{May} &
			\multicolumn{1}{l|}{101.66} &
			Feb &
			\multicolumn{1}{l|}{-0.13} &
			\multicolumn{1}{l|}{Dec} &
			\multicolumn{1}{l|}{0.187} &
			Jun &
			\multicolumn{1}{l|}{-0.12} &
			\multicolumn{1}{l|}{Jan} &
			\multicolumn{1}{l|}{0.157} &
			Jul \\ \hline
			2018 &
			\multicolumn{1}{l|}{25.67} &
			\multicolumn{1}{l|}{Feb} &
			\multicolumn{1}{l|}{30.2} &
			May &
			\multicolumn{1}{l|}{0.015} &
			\multicolumn{1}{l|}{Feb} &
			\multicolumn{1}{l|}{0.0219} &
			May &
			\multicolumn{1}{l|}{-0.0003} &
			\multicolumn{1}{l|}{Jun} &
			\multicolumn{1}{l|}{0} &
			Feb &
			\multicolumn{1}{l|}{100.3} &
			\multicolumn{1}{l|}{Aug} &
			\multicolumn{1}{l|}{101.66} &
			Feb &
			\multicolumn{1}{l|}{-0.12} &
			\multicolumn{1}{l|}{Mar} &
			\multicolumn{1}{l|}{0.238} &
			Aug &
			\multicolumn{1}{l|}{-0.09} &
			\multicolumn{1}{l|}{Feb} &
			\multicolumn{1}{l|}{0.206} &
			Jun \\ \hline
			2019 &
			\multicolumn{1}{l|}{25.93} &
			\multicolumn{1}{l|}{Jan} &
			\multicolumn{1}{l|}{30.94} &
			May &
			\multicolumn{1}{l|}{0.017} &
			\multicolumn{1}{l|}{Mar} &
			\multicolumn{1}{l|}{0.0224} &
			Apr &
			\multicolumn{1}{l|}{-0.0003} &
			\multicolumn{1}{l|}{Apr} &
			\multicolumn{1}{l|}{0} &
			Feb &
			\multicolumn{1}{l|}{100.31} &
			\multicolumn{1}{l|}{Jul} &
			\multicolumn{1}{l|}{101.66} &
			Jan &
			\multicolumn{1}{l|}{-0.12} &
			\multicolumn{1}{l|}{Nov} &
			\multicolumn{1}{l|}{0.213} &
			Aug &
			\multicolumn{1}{l|}{-0.11} &
			\multicolumn{1}{l|}{Jan} &
			\multicolumn{1}{l|}{0.173} &
			Sep \\ \hline
			2020 &
			\multicolumn{1}{l|}{25.6} &
			\multicolumn{1}{l|}{Jan} &
			\multicolumn{1}{l|}{30.4} &
			May &
			\multicolumn{1}{l|}{0.0155} &
			\multicolumn{1}{l|}{Jan} &
			\multicolumn{1}{l|}{0.0228} &
			May &
			\multicolumn{1}{l|}{-0.0003} &
			\multicolumn{1}{l|}{May} &
			\multicolumn{1}{l|}{0} &
			Jan &
			\multicolumn{1}{l|}{100.12} &
			\multicolumn{1}{l|}{May} &
			\multicolumn{1}{l|}{101.63} &
			Jan &
			\multicolumn{1}{l|}{-0.14} &
			\multicolumn{1}{l|}{Feb} &
			\multicolumn{1}{l|}{0.278} &
			Jul &
			\multicolumn{1}{l|}{-0.09} &
			\multicolumn{1}{l|}{Jan} &
			\multicolumn{1}{l|}{0.223} &
			May \\ \hline
			2021 &
			\multicolumn{1}{l|}{25.7} &
			\multicolumn{1}{l|}{Dec} &
			\multicolumn{1}{l|}{30.54} &
			May &
			\multicolumn{1}{l|}{0.0152} &
			\multicolumn{1}{l|}{Feb} &
			\multicolumn{1}{l|}{0.0216} &
			May &
			\multicolumn{1}{l|}{-0.0003} &
			\multicolumn{1}{l|}{May} &
			\multicolumn{1}{l|}{0} &
			Feb &
			\multicolumn{1}{l|}{100.31} &
			\multicolumn{1}{l|}{Oct} &
			\multicolumn{1}{l|}{101.49} &
			Dec &
			\multicolumn{1}{l|}{-0.12} &
			\multicolumn{1}{l|}{Feb} &
			\multicolumn{1}{l|}{0.253} &
			May &
			\multicolumn{1}{l|}{-0.13} &
			\multicolumn{1}{l|}{Feb} &
			\multicolumn{1}{l|}{0.192} &
			Jun \\ \hline
			Average &
			\multicolumn{1}{l|}{25.47} &
			\multicolumn{1}{l|}{Jan} &
			\multicolumn{1}{l|}{30.53} &
			May &
			\multicolumn{1}{l|}{0.015} &
			\multicolumn{1}{l|}{Feb} &
			\multicolumn{1}{l|}{0.022} &
			May &
			\multicolumn{1}{l|}{-0.0003} &
			\multicolumn{1}{l|}{May} &
			\multicolumn{1}{l|}{0} &
			Jan &
			\multicolumn{1}{l|}{100.2} &
			\multicolumn{1}{l|}{Jun} &
			\multicolumn{1}{l|}{101.61} &
			Jan &
			\multicolumn{1}{l|}{-0.15} &
			\multicolumn{1}{l|}{Dec} &
			\multicolumn{1}{l|}{0.225} &
			Jul &
			\multicolumn{1}{l|}{-0.11} &
			\multicolumn{1}{l|}{Jan} &
			\multicolumn{1}{l|}{0.18} &
			Jul \\ \hline
		\end{tabular}%
	}
\end{table}
\begin{table}[H]
	\centering
	\caption{Tekanan atmosfer dengan nilai ekstrim pada garis lintang $19^\circ$N}
	\label{table:NCEP_19}
	\resizebox{\columnwidth}{!}{%
	\begin{tabular}{|c|llll|llll|llll|llll|llll|llll|}
		\hline
		\multirow{3}{*}{Years} &
		\multicolumn{4}{c|}{AirT ($^\circ$C)} &
		\multicolumn{4}{c|}{SHum (kg/kg)} &
		\multicolumn{4}{c|}{CPrecR ($kg/m^2s$)} &
		\multicolumn{4}{c|}{SLP (kPa)} &
		\multicolumn{4}{c|}{TauX (Pa)} &
		\multicolumn{4}{c|}{TauY (Pa)} \\ \cline{2-25} 
		&
		\multicolumn{2}{c|}{Min} &
		\multicolumn{2}{c|}{Max} &
		\multicolumn{2}{c|}{Min} &
		\multicolumn{2}{c|}{Max} &
		\multicolumn{2}{c|}{Min} &
		\multicolumn{2}{c|}{Max} &
		\multicolumn{2}{c|}{Min} &
		\multicolumn{2}{c|}{Max} &
		\multicolumn{2}{c|}{Min} &
		\multicolumn{2}{c|}{Max} &
		\multicolumn{2}{c|}{Min} &
		\multicolumn{2}{c|}{Max} \\ \cline{2-25} 
		&
		\multicolumn{1}{c|}{Value} &
		\multicolumn{1}{c|}{Time} &
		\multicolumn{1}{c|}{Value} &
		\multicolumn{1}{c|}{Time} &
		\multicolumn{1}{c|}{Value} &
		\multicolumn{1}{c|}{Time} &
		\multicolumn{1}{c|}{Value} &
		\multicolumn{1}{c|}{Time} &
		\multicolumn{1}{c|}{Value} &
		\multicolumn{1}{c|}{Time} &
		\multicolumn{1}{c|}{Value} &
		\multicolumn{1}{c|}{Time} &
		\multicolumn{1}{c|}{Value} &
		\multicolumn{1}{c|}{Time} &
		\multicolumn{1}{c|}{Value} &
		\multicolumn{1}{c|}{Time} &
		\multicolumn{1}{c|}{Value} &
		\multicolumn{1}{c|}{Time} &
		\multicolumn{1}{c|}{Value} &
		\multicolumn{1}{c|}{Time} &
		\multicolumn{1}{c|}{Value} &
		\multicolumn{1}{c|}{Time} &
		\multicolumn{1}{c|}{Value} &
		\multicolumn{1}{c|}{Time} \\ \hline
		2002 &
		\multicolumn{1}{l|}{20.88} &
		\multicolumn{1}{l|}{Feb} &
		\multicolumn{1}{l|}{31.18} &
		May &
		\multicolumn{1}{l|}{0.009} &
		\multicolumn{1}{l|}{Feb} &
		\multicolumn{1}{l|}{0.023} &
		Jul &
		\multicolumn{1}{l|}{-0.0004} &
		\multicolumn{1}{l|}{Aug} &
		\multicolumn{1}{l|}{0} &
		Jan &
		\multicolumn{1}{l|}{99.57} &
		\multicolumn{1}{l|}{May} &
		\multicolumn{1}{l|}{102.08} &
		Feb &
		\multicolumn{1}{l|}{-0.06} &
		\multicolumn{1}{l|}{Nov} &
		\multicolumn{1}{l|}{0.18} &
		Aug &
		\multicolumn{1}{l|}{-0.1} &
		\multicolumn{1}{l|}{Feb} &
		\multicolumn{1}{l|}{0.167} &
		Aug \\ \hline
		2003 &
		\multicolumn{1}{l|}{21.33} &
		\multicolumn{1}{l|}{Dec} &
		\multicolumn{1}{l|}{31.32} &
		May &
		\multicolumn{1}{l|}{0.0094} &
		\multicolumn{1}{l|}{Jan} &
		\multicolumn{1}{l|}{0.0233} &
		Jun &
		\multicolumn{1}{l|}{-0.0004} &
		\multicolumn{1}{l|}{Oct} &
		\multicolumn{1}{l|}{0} &
		Jan &
		\multicolumn{1}{l|}{99.62} &
		\multicolumn{1}{l|}{Jul} &
		\multicolumn{1}{l|}{102.08} &
		Dec &
		\multicolumn{1}{l|}{-0.15} &
		\multicolumn{1}{l|}{May} &
		\multicolumn{1}{l|}{0.184} &
		Jun &
		\multicolumn{1}{l|}{-0.1} &
		\multicolumn{1}{l|}{Dec} &
		\multicolumn{1}{l|}{0.214} &
		Jun \\ \hline
		2004 &
		\multicolumn{1}{l|}{21.41} &
		\multicolumn{1}{l|}{Feb} &
		\multicolumn{1}{l|}{32.04} &
		May &
		\multicolumn{1}{l|}{0.0087} &
		\multicolumn{1}{l|}{Feb} &
		\multicolumn{1}{l|}{0.0228} &
		May &
		\multicolumn{1}{l|}{-0.0006} &
		\multicolumn{1}{l|}{Jun} &
		\multicolumn{1}{l|}{0} &
		Jan &
		\multicolumn{1}{l|}{99.44} &
		\multicolumn{1}{l|}{May} &
		\multicolumn{1}{l|}{102} &
		Jan &
		\multicolumn{1}{l|}{-0.11} &
		\multicolumn{1}{l|}{Sep} &
		\multicolumn{1}{l|}{0.141} &
		Jun &
		\multicolumn{1}{l|}{-0.1} &
		\multicolumn{1}{l|}{Oct} &
		\multicolumn{1}{l|}{0.19} &
		Aug \\ \hline
		2005 &
		\multicolumn{1}{l|}{21.7} &
		\multicolumn{1}{l|}{Jan} &
		\multicolumn{1}{l|}{30.93} &
		Jun &
		\multicolumn{1}{l|}{0.01} &
		\multicolumn{1}{l|}{Jan} &
		\multicolumn{1}{l|}{0.0239} &
		Jun &
		\multicolumn{1}{l|}{-0.0005} &
		\multicolumn{1}{l|}{Jul} &
		\multicolumn{1}{l|}{0} &
		Jan &
		\multicolumn{1}{l|}{99.62} &
		\multicolumn{1}{l|}{Jul} &
		\multicolumn{1}{l|}{102.08} &
		Jan &
		\multicolumn{1}{l|}{-0.19} &
		\multicolumn{1}{l|}{Dec} &
		\multicolumn{1}{l|}{0.185} &
		Aug &
		\multicolumn{1}{l|}{-0.11} &
		\multicolumn{1}{l|}{Dec} &
		\multicolumn{1}{l|}{0.171} &
		Jun \\ \hline
		2006 &
		\multicolumn{1}{l|}{21.56} &
		\multicolumn{1}{l|}{Jan} &
		\multicolumn{1}{l|}{30.55} &
		May &
		\multicolumn{1}{l|}{0.0089} &
		\multicolumn{1}{l|}{Jan} &
		\multicolumn{1}{l|}{0.023} &
		Jul &
		\multicolumn{1}{l|}{-0.0005} &
		\multicolumn{1}{l|}{Sep} &
		\multicolumn{1}{l|}{0} &
		Jan &
		\multicolumn{1}{l|}{99.46} &
		\multicolumn{1}{l|}{Jul} &
		\multicolumn{1}{l|}{101.97} &
		Feb &
		\multicolumn{1}{l|}{-0.07} &
		\multicolumn{1}{l|}{May} &
		\multicolumn{1}{l|}{0.152} &
		Jul &
		\multicolumn{1}{l|}{-0.11} &
		\multicolumn{1}{l|}{Jan} &
		\multicolumn{1}{l|}{0.242} &
		Aug \\ \hline
		2007 &
		\multicolumn{1}{l|}{20.88} &
		\multicolumn{1}{l|}{Jan} &
		\multicolumn{1}{l|}{30.62} &
		Jun &
		\multicolumn{1}{l|}{0.0085} &
		\multicolumn{1}{l|}{Jan} &
		\multicolumn{1}{l|}{0.0232} &
		May &
		\multicolumn{1}{l|}{-0.0006} &
		\multicolumn{1}{l|}{Jun} &
		\multicolumn{1}{l|}{0} &
		Jan &
		\multicolumn{1}{l|}{99.42} &
		\multicolumn{1}{l|}{Jun} &
		\multicolumn{1}{l|}{102} &
		Jan &
		\multicolumn{1}{l|}{-0.13} &
		\multicolumn{1}{l|}{Jun} &
		\multicolumn{1}{l|}{0.221} &
		Jul &
		\multicolumn{1}{l|}{-0.11} &
		\multicolumn{1}{l|}{Dec} &
		\multicolumn{1}{l|}{0.275} &
		Jun \\ \hline
		2008 &
		\multicolumn{1}{l|}{21.44} &
		\multicolumn{1}{l|}{Feb} &
		\multicolumn{1}{l|}{31.2} &
		May &
		\multicolumn{1}{l|}{0.0083} &
		\multicolumn{1}{l|}{Feb} &
		\multicolumn{1}{l|}{0.0235} &
		May &
		\multicolumn{1}{l|}{-0.0004} &
		\multicolumn{1}{l|}{Oct} &
		\multicolumn{1}{l|}{0} &
		Jan &
		\multicolumn{1}{l|}{99.51} &
		\multicolumn{1}{l|}{Aug} &
		\multicolumn{1}{l|}{101.82} &
		Feb &
		\multicolumn{1}{l|}{-0.08} &
		\multicolumn{1}{l|}{Nov} &
		\multicolumn{1}{l|}{0.173} &
		Jun &
		\multicolumn{1}{l|}{-0.11} &
		\multicolumn{1}{l|}{Nov} &
		\multicolumn{1}{l|}{0.195} &
		Aug \\ \hline
		2009 &
		\multicolumn{1}{l|}{21.84} &
		\multicolumn{1}{l|}{Jan} &
		\multicolumn{1}{l|}{31.29} &
		May &
		\multicolumn{1}{l|}{0.0102} &
		\multicolumn{1}{l|}{Jan} &
		\multicolumn{1}{l|}{0.0233} &
		Sep &
		\multicolumn{1}{l|}{-0.0005} &
		\multicolumn{1}{l|}{Jul} &
		\multicolumn{1}{l|}{0} &
		Jan &
		\multicolumn{1}{l|}{99.51} &
		\multicolumn{1}{l|}{May} &
		\multicolumn{1}{l|}{102.2} &
		Jan &
		\multicolumn{1}{l|}{-0.07} &
		\multicolumn{1}{l|}{Nov} &
		\multicolumn{1}{l|}{0.166} &
		Jul &
		\multicolumn{1}{l|}{-0.1} &
		\multicolumn{1}{l|}{Nov} &
		\multicolumn{1}{l|}{0.195} &
		May \\ \hline
		2010 &
		\multicolumn{1}{l|}{21.4} &
		\multicolumn{1}{l|}{Dec} &
		\multicolumn{1}{l|}{31.61} &
		May &
		\multicolumn{1}{l|}{0.009} &
		\multicolumn{1}{l|}{Dec} &
		\multicolumn{1}{l|}{0.0236} &
		May &
		\multicolumn{1}{l|}{-0.0004} &
		\multicolumn{1}{l|}{Jul} &
		\multicolumn{1}{l|}{0} &
		Jan &
		\multicolumn{1}{l|}{99.65} &
		\multicolumn{1}{l|}{Jul} &
		\multicolumn{1}{l|}{102.15} &
		Jan &
		\multicolumn{1}{l|}{-0.13} &
		\multicolumn{1}{l|}{Nov} &
		\multicolumn{1}{l|}{0.151} &
		Jun &
		\multicolumn{1}{l|}{-0.11} &
		\multicolumn{1}{l|}{Dec} &
		\multicolumn{1}{l|}{0.189} &
		May \\ \hline
		2011 &
		\multicolumn{1}{l|}{20.62} &
		\multicolumn{1}{l|}{Dec} &
		\multicolumn{1}{l|}{30.9} &
		Jun &
		\multicolumn{1}{l|}{0.0093} &
		\multicolumn{1}{l|}{Jan} &
		\multicolumn{1}{l|}{0.0229} &
		Jun &
		\multicolumn{1}{l|}{-0.0004} &
		\multicolumn{1}{l|}{Sep} &
		\multicolumn{1}{l|}{0} &
		Jan &
		\multicolumn{1}{l|}{99.53} &
		\multicolumn{1}{l|}{Jun} &
		\multicolumn{1}{l|}{101.83} &
		Jan &
		\multicolumn{1}{l|}{-0.12} &
		\multicolumn{1}{l|}{Dec} &
		\multicolumn{1}{l|}{0.174} &
		Jun &
		\multicolumn{1}{l|}{-0.1} &
		\multicolumn{1}{l|}{Dec} &
		\multicolumn{1}{l|}{0.205} &
		Jun \\ \hline
		2012 &
		\multicolumn{1}{l|}{20.85} &
		\multicolumn{1}{l|}{Jan} &
		\multicolumn{1}{l|}{31.84} &
		Jun &
		\multicolumn{1}{l|}{0.0093} &
		\multicolumn{1}{l|}{Jan} &
		\multicolumn{1}{l|}{0.0237} &
		Jun &
		\multicolumn{1}{l|}{-0.0003} &
		\multicolumn{1}{l|}{Aug} &
		\multicolumn{1}{l|}{0} &
		Jan &
		\multicolumn{1}{l|}{99.75} &
		\multicolumn{1}{l|}{Jun} &
		\multicolumn{1}{l|}{101.85} &
		Feb &
		\multicolumn{1}{l|}{-0.06} &
		\multicolumn{1}{l|}{Nov} &
		\multicolumn{1}{l|}{0.203} &
		Jul &
		\multicolumn{1}{l|}{-0.11} &
		\multicolumn{1}{l|}{Dec} &
		\multicolumn{1}{l|}{0.213} &
		Jun \\ \hline
		2013 &
		\multicolumn{1}{l|}{20.07} &
		\multicolumn{1}{l|}{Jan} &
		\multicolumn{1}{l|}{30.84} &
		May &
		\multicolumn{1}{l|}{0.0078} &
		\multicolumn{1}{l|}{Jan} &
		\multicolumn{1}{l|}{0.0229} &
		Jun &
		\multicolumn{1}{l|}{-0.0005} &
		\multicolumn{1}{l|}{Jul} &
		\multicolumn{1}{l|}{0} &
		Jan &
		\multicolumn{1}{l|}{99.67} &
		\multicolumn{1}{l|}{Jul} &
		\multicolumn{1}{l|}{101.87} &
		Jan &
		\multicolumn{1}{l|}{-0.15} &
		\multicolumn{1}{l|}{Oct} &
		\multicolumn{1}{l|}{0.223} &
		Aug &
		\multicolumn{1}{l|}{-0.09} &
		\multicolumn{1}{l|}{Jan} &
		\multicolumn{1}{l|}{0.314} &
		Oct \\ \hline
		2014 &
		\multicolumn{1}{l|}{21.31} &
		\multicolumn{1}{l|}{Feb} &
		\multicolumn{1}{l|}{31.76} &
		Jun &
		\multicolumn{1}{l|}{0.0084} &
		\multicolumn{1}{l|}{Feb} &
		\multicolumn{1}{l|}{0.0236} &
		Jun &
		\multicolumn{1}{l|}{-0.0006} &
		\multicolumn{1}{l|}{Jul} &
		\multicolumn{1}{l|}{0} &
		Jan &
		\multicolumn{1}{l|}{99.69} &
		\multicolumn{1}{l|}{Jul} &
		\multicolumn{1}{l|}{102.13} &
		Jan &
		\multicolumn{1}{l|}{-0.15} &
		\multicolumn{1}{l|}{Oct} &
		\multicolumn{1}{l|}{0.241} &
		Jul &
		\multicolumn{1}{l|}{-0.13} &
		\multicolumn{1}{l|}{Dec} &
		\multicolumn{1}{l|}{0.259} &
		Jul \\ \hline
		2015 &
		\multicolumn{1}{l|}{20.19} &
		\multicolumn{1}{l|}{Jan} &
		\multicolumn{1}{l|}{31.31} &
		May &
		\multicolumn{1}{l|}{0.0089} &
		\multicolumn{1}{l|}{Feb} &
		\multicolumn{1}{l|}{0.0231} &
		May &
		\multicolumn{1}{l|}{-0.0005} &
		\multicolumn{1}{l|}{Jun} &
		\multicolumn{1}{l|}{0} &
		Jan &
		\multicolumn{1}{l|}{99.73} &
		\multicolumn{1}{l|}{Jun} &
		\multicolumn{1}{l|}{102.08} &
		Feb &
		\multicolumn{1}{l|}{-0.14} &
		\multicolumn{1}{l|}{Jan} &
		\multicolumn{1}{l|}{0.223} &
		Aug &
		\multicolumn{1}{l|}{-0.13} &
		\multicolumn{1}{l|}{Dec} &
		\multicolumn{1}{l|}{0.275} &
		Jun \\ \hline
		2016 &
		\multicolumn{1}{l|}{19.59} &
		\multicolumn{1}{l|}{Jan} &
		\multicolumn{1}{l|}{30.61} &
		Apr &
		\multicolumn{1}{l|}{0.0094} &
		\multicolumn{1}{l|}{Jan} &
		\multicolumn{1}{l|}{0.0229} &
		Jul &
		\multicolumn{1}{l|}{-0.0004} &
		\multicolumn{1}{l|}{Jul} &
		\multicolumn{1}{l|}{0} &
		Jan &
		\multicolumn{1}{l|}{99.61} &
		\multicolumn{1}{l|}{Aug} &
		\multicolumn{1}{l|}{102.06} &
		Jan &
		\multicolumn{1}{l|}{-0.07} &
		\multicolumn{1}{l|}{Oct} &
		\multicolumn{1}{l|}{0.202} &
		Jun &
		\multicolumn{1}{l|}{-0.11} &
		\multicolumn{1}{l|}{Nov} &
		\multicolumn{1}{l|}{0.184} &
		Apr \\ \hline
		2017 &
		\multicolumn{1}{l|}{21.5} &
		\multicolumn{1}{l|}{Jan} &
		\multicolumn{1}{l|}{30.85} &
		May &
		\multicolumn{1}{l|}{0.0098} &
		\multicolumn{1}{l|}{Jan} &
		\multicolumn{1}{l|}{0.0234} &
		Jun &
		\multicolumn{1}{l|}{-0.0004} &
		\multicolumn{1}{l|}{Oct} &
		\multicolumn{1}{l|}{0} &
		Jan &
		\multicolumn{1}{l|}{99.82} &
		\multicolumn{1}{l|}{Jul} &
		\multicolumn{1}{l|}{102.11} &
		Feb &
		\multicolumn{1}{l|}{-0.15} &
		\multicolumn{1}{l|}{Oct} &
		\multicolumn{1}{l|}{0.18} &
		Jun &
		\multicolumn{1}{l|}{-0.1} &
		\multicolumn{1}{l|}{Jan} &
		\multicolumn{1}{l|}{0.242} &
		Jul \\ \hline
		2018 &
		\multicolumn{1}{l|}{20.08} &
		\multicolumn{1}{l|}{Jan} &
		\multicolumn{1}{l|}{30.75} &
		Jun &
		\multicolumn{1}{l|}{0.0085} &
		\multicolumn{1}{l|}{Jan} &
		\multicolumn{1}{l|}{0.023} &
		Jun &
		\multicolumn{1}{l|}{-0.0005} &
		\multicolumn{1}{l|}{Sep} &
		\multicolumn{1}{l|}{0} &
		Jan &
		\multicolumn{1}{l|}{99.57} &
		\multicolumn{1}{l|}{Jul} &
		\multicolumn{1}{l|}{101.98} &
		Feb &
		\multicolumn{1}{l|}{-0.13} &
		\multicolumn{1}{l|}{Oct} &
		\multicolumn{1}{l|}{0.189} &
		Jul &
		\multicolumn{1}{l|}{-0.12} &
		\multicolumn{1}{l|}{Dec} &
		\multicolumn{1}{l|}{0.212} &
		Jul \\ \hline
		2019 &
		\multicolumn{1}{l|}{20.76} &
		\multicolumn{1}{l|}{Jan} &
		\multicolumn{1}{l|}{31.3} &
		Jun &
		\multicolumn{1}{l|}{0.0095} &
		\multicolumn{1}{l|}{Dec} &
		\multicolumn{1}{l|}{0.0237} &
		Jun &
		\multicolumn{1}{l|}{-0.0006} &
		\multicolumn{1}{l|}{Aug} &
		\multicolumn{1}{l|}{0} &
		Jan &
		\multicolumn{1}{l|}{99.46} &
		\multicolumn{1}{l|}{Aug} &
		\multicolumn{1}{l|}{102.1} &
		Feb &
		\multicolumn{1}{l|}{-0.1} &
		\multicolumn{1}{l|}{Nov} &
		\multicolumn{1}{l|}{0.261} &
		Aug &
		\multicolumn{1}{l|}{-0.09} &
		\multicolumn{1}{l|}{Feb} &
		\multicolumn{1}{l|}{0.319} &
		Aug \\ \hline
		2020 &
		\multicolumn{1}{l|}{20.89} &
		\multicolumn{1}{l|}{Jan} &
		\multicolumn{1}{l|}{31.21} &
		May &
		\multicolumn{1}{l|}{0.0096} &
		\multicolumn{1}{l|}{Jan} &
		\multicolumn{1}{l|}{0.0238} &
		Aug &
		\multicolumn{1}{l|}{-0.0005} &
		\multicolumn{1}{l|}{May} &
		\multicolumn{1}{l|}{0} &
		Jan &
		\multicolumn{1}{l|}{99.29} &
		\multicolumn{1}{l|}{Aug} &
		\multicolumn{1}{l|}{102.11} &
		Jan &
		\multicolumn{1}{l|}{-0.15} &
		\multicolumn{1}{l|}{May} &
		\multicolumn{1}{l|}{0.233} &
		Jul &
		\multicolumn{1}{l|}{-0.1} &
		\multicolumn{1}{l|}{Dec} &
		\multicolumn{1}{l|}{0.26} &
		Aug \\ \hline
		2021 &
		\multicolumn{1}{l|}{21.54} &
		\multicolumn{1}{l|}{Feb} &
		\multicolumn{1}{l|}{31.8} &
		May &
		\multicolumn{1}{l|}{0.0094} &
		\multicolumn{1}{l|}{Dec} &
		\multicolumn{1}{l|}{0.0233} &
		Jun &
		\multicolumn{1}{l|}{-0.0005} &
		\multicolumn{1}{l|}{May} &
		\multicolumn{1}{l|}{0} &
		Jan &
		\multicolumn{1}{l|}{99.4} &
		\multicolumn{1}{l|}{May} &
		\multicolumn{1}{l|}{102.03} &
		Dec &
		\multicolumn{1}{l|}{-0.09} &
		\multicolumn{1}{l|}{Dec} &
		\multicolumn{1}{l|}{0.229} &
		Aug &
		\multicolumn{1}{l|}{-0.1} &
		\multicolumn{1}{l|}{Dec} &
		\multicolumn{1}{l|}{0.289} &
		May \\ \hline
		Average &
		\multicolumn{1}{l|}{20.99} &
		\multicolumn{1}{l|}{Jan} &
		\multicolumn{1}{l|}{31.2} &
		May &
		\multicolumn{1}{l|}{0.009} &
		\multicolumn{1}{l|}{Jan} &
		\multicolumn{1}{l|}{0.023} &
		Jun &
		\multicolumn{1}{l|}{-0.0004} &
		\multicolumn{1}{l|}{Jul} &
		\multicolumn{1}{l|}{0} &
		Jan &
		\multicolumn{1}{l|}{99.57} &
		\multicolumn{1}{l|}{Jul} &
		\multicolumn{1}{l|}{102} &
		Jan &
		\multicolumn{1}{l|}{-0.11} &
		\multicolumn{1}{l|}{Nov} &
		\multicolumn{1}{l|}{0.196} &
		Jun &
		\multicolumn{1}{l|}{-0.11} &
		\multicolumn{1}{l|}{Dec} &
		\multicolumn{1}{l|}{0.231} &
		Aug \\ \hline
	\end{tabular}%
}
\end{table}

%\pagebreak
%\newappendix{Lampiran 2. Diskritisasi pada C-grid Arakawa}
%\begin{spacing}{1.5}
	\begin{figure}[H]
		\centering
		\includegraphics[width=1\textwidth]{contents/Arakawa_1.png}		
		\caption{Distribusi variabel pada C-grid Arakawa}
		\label{fig:arakawa_1}
	\end{figure}
	\begin{figure}[H]
		\centering
		\includegraphics[width=1\textwidth]{contents/Arakawa_2.png}	
		\caption{Diskritisasi C-grid Arakawa (posisi bawah kubus, $k$ tetap)}
		\label{fig:arakawa_2}
	\end{figure}
	\begin{figure}[H]
		\centering
		\includegraphics[width=1\textwidth]{contents/Arakawa_3.png}	
		\caption{Diskritisasi C-grid Arakawa (posisi tengah kubus, $k$ tetap)}
		\label{fig:arakawa_3}
	\end{figure}
		\begin{figure}[H]
		\centering
		\includegraphics[width=1\textwidth]{contents/Arakawa_4.png}	
		\caption{Diskritisasi C-grid Arakawa ($j$ tetap)}
		\label{fig:arakawa_4}
	\end{figure}
		\begin{figure}[H]
		\centering
		\includegraphics[width=1\textwidth]{contents/Arakawa_5.png}	
		\caption{Diskritisasi C-grid Arakawa ($i$ tetap)}
		\label{fig:arakawa_5}
	\end{figure}
\end{spacing}
%\pagebreak
%\newappendix{Lampiran 3.}
%\begin{figure}[H]
	\centering
	\includegraphics[width=15cm]{contents/final_figure/SeasonalModel9}
	\caption{\textit{Seasonal model} pada latitude 9$^\circ$C untuk (a) 2m \textit{air temperature},(b) 2m \textit{specific humidity},(c) \textit{Convective precipitation rate},(d) \textit{Sea level pressure},(e) \textit{Wind stress U},(f) \textit{Wind stress V}}
	\label{fig:SM9}
\end{figure}

\begin{figure}[H]
	\centering
	\includegraphics[width=15cm]{contents/final_figure/SeasonalModel19}
	\caption{\textit{Seasonal model} pada latitude 19$^\circ$C untuk (a) 2m \textit{air temperature},(b) 2m \textit{specific humidity},(c) \textit{Convective precipitation rate},(d) \textit{Sea level pressure},(e) \textit{Wind stress U},(f) \textit{Wind stress V}}
	\label{fig:SM19}
\end{figure}
%=====================================================================
% Halaman Biodata
%\newpage
%\chapter*{BIODATA}
%\input{Bio}
%=====================================================================
\end{document}
